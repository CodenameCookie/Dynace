@page

@section String Class
The @code{String} class is used to represent the C language
@code{char *} data type as a Dynace object.  It is a subclass
of the @code{Stream} class and it provides an array of methods
to operate on the string objects.  Unlike C strings this class will
dynamically expand a string buffer as needed so you should never have to
worry about the buffer size.  You can also use the @code{MA_compact}
function to compact the string memory and reduce memory fragmentation.

Most of the @code{String} methods which take string arguments (except
the first argument) will accept a C language string (@code{char *}) or
an instance of the @code{String} class arbitrarily.  The system can tell
the difference between them at runtime.  If it is a valid Dynace object 
the system will assure that it is a kind of @code{String}, otherwise
it is assumed to be a C string.

Instances of this class will also operate like other subclasses of
@code{Stream}.  That is you can write (append), read (copy and remove),
advance (remove leading characters) and obtain its length.  This
comes in handy when you wish to keep a small (and fast) sequential
file in memory.


@subsection String Class Methods
The String class has several class methods which create new @code{String}
instances.  They are provided to lend various levels of flexibility.







@deffn {Build} Build::String
@sp 2
@example
@group
i = vBuild(String, ...);

object  i;
@end group
@end example
This class method creates a new instance of the String class.
The initial value of the new instance will be the concatenation
of all the @code{String} or @code{char *} arguments up to an @code{END}.

Each argument may be a C language @code{char *} or an instance of the
@code{String} class.  Dynace can tell the difference between them at
runtime.  If it is a valid Dynace object the system will assure that it is
a kind of @code{String}, otherwise it is assumed to be a C string.

Note that the variable argument list @emph{MUST} end with an @code{END}.
There is no other way in C to tell when the last argument has been reached.
If you forget the @code{END} the system will probably hang!  (If you
have a problem with the system hanging you might want to check all the
@code{vBuild} generic calls.)

The value returned is the new @code{String} instance.
@example
@group
@exdent Example:

object  x, y;

x = gNewWithStr(String, " World");
y = vBuild(String, "Hello", x, "!", END);
/*  y = "Hello World!"  */
@end group
@end example
@sp 1
See also:  @code{StringValue::String, Dispose::String,}
There is also an instance method @code{Build::String}
@end deffn





@deffn {MaskFunction} MaskFunction::String
@sp 2
@example
@group
fun = gMaskFunction(String, ch);

char    ch;
int     (*fun)();
@end group
@end example
This method is used to return the masking function associated with
the mask character @code{ch}.

Currently there are only two masking characters used in the system,
@code{`#'} (numbers) and @code{`@@'} (letters).  The @code{`#'}
character's associated function is @code{isdigit} while the @code{`@@'}
character's associated function is @code{isalpha}.

The masking function associated with a masking character takes a
character as its only parameter and returns non-zero if the character
is the masking character associated with the function and returns zero
if it is not.

@example
@group
@exdent Example:

int     (*fun)();
char    *str = "A1";
int     r;

fun = gMaskFunction(String, '#');       /*  Uses isdigit  */
r = fun(str[0]);                        /*  r == 0        */
r = fun(str[1]);                        /*  r == 1        */
@end group
@end example
@sp 1
See also:  @code{gSetMaskFiller::String, gApplyMask::String,}
@iftex
@hfil @break @hglue .64in      
@end iftex
@code{gRemoveMask::String}
@end deffn












@deffn {New} New::String
@sp 2
@example
@group
i = gNew(String);

object  i;
@end group
@end example
This class method creates instances of the String class.  The returned
String object is initialized to @code{""}.

The value returned is a Dynace instance object which represents a null string.
@example
@group
@exdent Example:

object  x;

x = gNew(String);
@end group
@end example
@sp 2
See also:  @code{Copy::String, StringValue::String, Dispose::String,}
@iftex
@hfil @break @hglue .64in      
@end iftex
@code{NewWithStr::String, NewWithObj::String}
@end deffn








@deffn {NewWithStr} NewWithStr::String
@sp 2
@example
@group
i = gNewWithStr(String, val);

char *val;  /*  or object  */
object  i;
@end group
@end example
This class method creates instances of the String class.  @code{val}
is the initial value of the string being represented.  

@code{val} may be a C language @code{char *} or an instance of the
@code{String} class.  Dynace can tell the difference between them at
runtime.  If it is a valid Dynace object the system will assure that it is
a kind of @code{String}, otherwise it is assumed to be a C string.

The value returned is a Dynace instance object which represents the string
passed.
@example
@group
@exdent Example:

object  x, y;

x = gNewWithStr(String, "Hello World!");
y = gNewWithStr(String, (char *) x);
@end group
@end example
See also:  @code{Copy::String, StringValue::String, Dispose::String}
@iftex
@hfil @break @hglue .64in     
@end iftex
@code{New::String, NewWithObj::String}
@end deffn












@deffn {NewWithObj} NewWithObj::String
@sp 2
@example
@group
i = gNewWithObj(String, val);

object  val;  /*  or char *  */
object  i;
@end group
@end example
This class method creates instances of the String class.  @code{val}
is the initial value of the string being represented.  

@code{val} may be a C language @code{char *} or an instance of the
@code{String} class.  Dynace can tell the difference between them at
runtime.  If it is a valid Dynace object the system will assure that it is
a kind of @code{String}, otherwise it is assumed to be a C string.

The value returned is a Dynace instance object which represents the string
passed.
@example
@group
@exdent Example:

object  x, y;

x = gNewWithObj(String, (object) "Hello World!");
y = gNewWithObj(String, x);
@end group
@end example
See also:  @code{Copy::String, StringValue::String, Dispose::String}
@iftex
@hfil @break @hglue .64in      
@end iftex
@code{NewWithStr::String, New::String}
@end deffn












@deffn {Sprintf} Sprintf::String
@sp 2
@example
@group
i = vSprintf(String, fmt, ...);

char    *fmt;
object  i;
@end group
@end example
This class method creates a new instances of the String class.  Except
for the first argument and return value, this method takes the same
arguments and performs the same function as the normal @code{sprintf}
contained in your C library.  See that documentation.

Note that except for the first argument, this method does not take
Dynace objects.

The value returned is a Dynace instance object which represents the result
of the operation.

@example
@group
@exdent Example:

object  x;

x = vSprintf(String, "%s is %d years old", "John", 66);
@end group
@end example
@sp 1
See also:  @code{StringValue::String, Dispose::String}
@end deffn





@subsection String Instance Methods
The instance methods associated with the String class provide a
means of changing and obtaining the value associated with an instance of
the String class.  Methods are also included to help the generic
container classes to quickly access members of this class.








@deffn {Append} Append::String
@sp 2
@example
@group
i = gAppend(i, str);

object  i;
char    *str;  /*  or object  */
@end group
@end example
This method is used to append the string represented by @code{str}
to the end of the string represented by @code{i}.

@code{str} may be a C language @code{char *} or an instance of the
@code{String} class.  Dynace can tell the difference between them at
runtime.  If it is a valid Dynace object the system will assure that it is
a kind of @code{String}, otherwise it is assumed to be a C string.

The value returned is the @code{String} instance passed (@code{i}).
@example
@group
@exdent Example:

object  x;

x = gNewWithStr(String, "Hello");
gAppend(x, " World!");
/*  x = "Hello World!"  */
@end group
@end example
@sp 1
See also:  @code{ChangeValue::String, Build::String}
@end deffn












@deffn {ApplyMask} ApplyMask::String
@sp 2
@example
@group
i = gApplyMask(i, mask, text);

object  i;
char    *mask;
char    *text;
@end group
@end example
This method is used to apply @code{mask} to @code{text} and set the
resulting value to @code{i}.  If @code{NULL} or an empty string
(@code{""}) is passed in to @code{mask} then @code{i} has its string
value changed to @code{text}.  If @code{NULL} or an empty string
(@code{""}) is passed in to @code{text} then @code{mask} is applied
directly to the current string value in @code{i}.

Mask definitions contain any combination of mask characters and literal
characters.  The first character of a mask can optionally be either `>'
or `<' indicating the direction which the mask is applied to the string.
If the first character is `<' then the mask is applied from right to
left.  If the first character is `>' or is neither of the directional
characters, then the mask is applied from left to right.  The mask
character `#' indicates that the character in this position must be a
number while the mask character `@@' indicates that the character in
this position must be a letter.  Any other character in the mask is a
literal character and is left in the resulting string in its position
unchanged.

The value returned is the @code{String} instance passed (@code{i}).
@example
@group
@exdent Example:

object  x;
char    mask = "<(###) ###-####";

x = gNewWithStr(String, "1234567890");
gApplyMask(x, mask, NULL);         /* x = "(123) 456-7890" */
gApplyMask(x, mask, "9876543210"); /* x = "(987) 654-3210" */
gApplyMask(x, NULL, "1234567890"); /* x = "1234567890"     */
@end group
@end example
@sp 1
See also:  @code{gMaskFunction::String, gSetMaskFiller::String,}
@iftex
@hfil @break @hglue .64in      
@end iftex
@code{gRemoveMask::String}
@end deffn















@deffn {Build} Build::String
@sp 2
@example
@group
i = vBuild(str, first, ...);

object  str;
char    *first;  /*  or object  */
object  i;
@end group
@end example
This method is used to modify the string associated with instance @code{str}
by optionally overwriting and optionally appending a group of C strings or
@code{String} objects.

@code{first} may be an instance of the @code{String} class, a C string
@code{char *} or a @code{NULL}.  If @code{first} is @code{NULL} the initial
value of @code{str} will remain intact, otherwise, it will be replaced
with the string represented by @code{first}.

After the @code{first} argument has been processed the remaining
@code{String} or @code{char *} arguments, up to an @code{END}, 
will be concatenated to the end of @code{str}.

Each argument may be a C language @code{char *} or an instance of the
@code{String} class.  Dynace can tell the difference between them at
runtime.  If it is a valid Dynace object the system will assure that it is
a kind of @code{String}, otherwise it is assumed to be a C string.

Note that the variable argument list @emph{MUST} end with an @code{END}.
There is no other way in C to tell when the last argument has been reached.
If you forget the @code{END} the system will probably hang!  (If you
have a problem with the system hanging you might want to check all the
@code{vBuild} generic calls.)

The value returned is the @code{String} instance passed (@code{str}).
@example
@group
@exdent Example:

object  x;

x = gNewWithStr(String, "Hello");
vBuild(x, NULL, " World", gNewWithStr(String, "!"), END);
/*  x = "Hello World!"  */

vBuild(x, "Goodbye", " World", "!", END);
/*  x = "Goodbye World!"  */
@end group
@end example
@sp 1
See also:  @code{ChangeValue::String, Append::String,}
There is also a class method @code{Build::String}
@end deffn












@deffn {ChangeCharAt} ChangeCharAt::String
@sp 2
@example
@group
i = gChangeCharAt(i, n, c);

object  i;
int     n;
int     c;
@end group
@end example
This method is used to change the character at position @code{n} within
@code{String} instance @code{i} to character @code{c}.  The index origin
is zero.

If @code{n} is equal to the number of characters in instance @code{i}
(that is one greater than the number of indexable characters), string
@code{i} will be extended by one character (@code{c}).  However, if any
larger index is used Dynace will issue an error message and abort the
program.
@example
@group
@exdent Example:

object  x;

x = gNewWithStr(String, "Hello");
gChangeCharAt(x, 1, 'E');
/*  x = "HEllo"  */
@end group
@end example
@sp 1
See also:  @code{CharValueAt::String}
@end deffn










@deffn {ChangeStrValue} ChangeStrValue::String
@sp 2
@example
@group
i = gChangeStrValue(i, val);

object  i;
char   *val;
@end group
@end example
This method is used to change the value associated with an instance of
the String class.  The new value may be of any length.

The value returned is the instance @code{i}.
@example
@group
@exdent Example:

object  x, y;

x = gNewWithStr(String, "Hello World");
gChangeStrValue(x, "Are you all using Dynace?");
@end group
@end example
@sp 1
See also:  @code{ChangeValue::String, Append::String, Build::String}
@end deffn














@deffn {ChangeValue} ChangeValue::String
@sp 2
@example
@group
i = gChangeValue(i, val);

object  i;
object  val;  /*  or char *  */
@end group
@end example
This method is used to change the value associated with an instance of
the String class.  The new value may be of any length.

@code{val} may be a C language @code{char *} or an instance of the
@code{String} class.  Dynace can tell the difference between them at
runtime.  If it is a valid Dynace object the system will assure that it is
a kind of @code{String}, otherwise it is assumed to be a C string.

The value returned is the instance @code{i}.
@example
@group
@exdent Example:

object  x, y;

x = gNewWithStr(String, "Hello World");
y = gNewWithStr(String, "Are you all using Dynace?");
gChangeValue(x, y);
@end group
@end example
@sp 1
See also:  @code{ChangeStrValue::String, Append::String, Build::String}
@end deffn











@deffn {CharValueAt} CharValueAt::String
@sp 2
@example
@group
c = gCharValueAt(i, n);

object  i;
int     n;
char    c;
@end group
@end example
This method is used to obtain character @code{n} from @code{String} instance
@code{i}.  The index origin is zero.  If an index greater than the available
characters is used Dynace will issue an error message and abort the program.
@example
@group
@exdent Example:

object  x;
char    c;

x = gNewWithStr(String, "Hello");
c = gCharValueAt(x, 1);
/*  c = 'e'  */
@end group
@end example
@sp 1
See also:  @code{ChangeCharAt::String}
@end deffn









@deffn {Compare} Compare::String
@sp 2
@example
@group
r = gCompare(i, obj);

object  i;
object  obj;  /*  or char *  */
int     r;
@end group
@end example
This method is used to determine the equality of the values represented
by @code{i} and @code{obj}.

@code{obj} may be a C language @code{char *} or an instance of the
@code{String} class.  Dynace can tell the difference between them at
runtime.  If it is a valid Dynace object the system will assure that it is
a kind of @code{String}, otherwise it is assumed to be a C string.

@code{r} is -1 if the value represented by @code{i} is less than the
value represented by @code{obj}, 1 if the value of @code{i} is greater
than @code{obj}, and 0 if they are equal.

This method is also used by the generic container classes to determine
object relationships.
@c @example
@c @group
@c @exdent Example:
@c
@c @end group
@c @end example
@sp 1
See also:  @code{Equal::String, CompareI::String, CompareN::String}
@end deffn








@deffn {CompareI} CompareI::String
@sp 2
@example
@group
r = gCompareI(i, obj);

object  i;
object  obj;  /*  or char *  */
int     r;
@end group
@end example
This method is used to determine the equality of the values represented
by @code{i} and @code{obj} without regard for the case of the letters.
Therefore, strings such as ``abc'' and ``ABC'' will be considered equal.

@code{obj} may be a C language @code{char *} or an instance of the
@code{String} class.  Dynace can tell the difference between them at
runtime.  If it is a valid Dynace object the system will assure that it is
a kind of @code{String}, otherwise it is assumed to be a C string.

@code{r} is -1 if the value represented by @code{i} is less than the
value represented by @code{obj}, 1 if the value of @code{i} is greater
than @code{obj}, and 0 if they are equal.
@c @example
@c @group
@c @exdent Example:
@c
@c @end group
@c @end example
@sp 1
See also:  @code{Compare::String, CompareN::String}
@end deffn






@deffn {CompareN} CompareN::String
@sp 2
@example
@group
r = gCompareN(i, obj, n);

object  i;
object  obj;  /*  or char *  */
int     n;    /*  max number of characters to compare  */
int     r;
@end group
@end example
This method is used to determine the equality of the values represented
by @code{i} and @code{obj}.  No more than @code{n} characters will be
compared. If the two strings only differ past @code{n} characters they
will be considered equal.


@code{obj} may be a C language @code{char *} or an instance of the
@code{String} class.  Dynace can tell the difference between them at
runtime.  If it is a valid Dynace object the system will assure that it is
a kind of @code{String}, otherwise it is assumed to be a C string.

@code{r} is -1 if the value represented by @code{i} is less than the
value represented by @code{obj}, 1 if the value of @code{i} is greater
than @code{obj}, and 0 if they are equal.
@c @example
@c @group
@c @exdent Example:
@c
@c @end group
@c @end example
@sp 1
See also:  @code{Compare::String, CompareNI::String}
@end deffn





@deffn {CompareNI} CompareNI::String
@sp 2
@example
@group
r = gCompareNI(i, obj, n);

object  i;
object  obj;  /*  or char *  */
int     n;    /*  max number of characters to compare  */
int     r;
@end group
@end example
This method is used to determine the equality of the values represented
by @code{i} and @code{obj} without regard for the case of the letters.
Therefore, strings such as ``abc'' and ``ABC'' will be considered equal.
No more than @code{n} characters will be compared.  If the two strings
only differ past @code{n} characters they will be considered equal.

@code{obj} may be a C language @code{char *} or an instance of the
@code{String} class.  Dynace can tell the difference between them at
runtime.  If it is a valid Dynace object the system will assure that it is
a kind of @code{String}, otherwise it is assumed to be a C string.

@code{r} is -1 if the value represented by @code{i} is less than the
value represented by @code{obj}, 1 if the value of @code{i} is greater
than @code{obj}, and 0 if they are equal.
@c @example
@c @group
@c @exdent Example:
@c
@c @end group
@c @end example
@sp 1
See also:  @code{CompareI::String, CompareN::String}
@end deffn














@deffn {Copy} Copy::String
@sp 2
@example
@group
so = gCopy(i);

object  i, so;
@end group
@end example
This method is used to create a new instance of the @code{String} class
with the same string value as string @code{i}.  Or put more simply,
it makes a copy.  The new @code{String} instance is returned.

This method performs the same function as @code{DeepCopy::String}.
@example
@group
@exdent Example:

object  x, y;

x = gNewWithStr(String, "Hello");
y = gCopy(x);
/*  x = "Hello"  */
@end group
@end example
@sp 1
See also:  @code{New::String, DeepCopy::String}
@end deffn













@deffn {DeepCopy} DeepCopy::String
@sp 2
@example
@group
so = gDeepCopy(i);

object  i, so;
@end group
@end example
This method is used to create a new instance of the @code{String} class
with the same string value as string @code{i}.  Or put more simply,
it makes a copy.  The new @code{String} instance is returned.

This method performs the same function as @code{Copy::String}.
@example
@group
@exdent Example:

object  x, y;

x = gNewWithStr(String, "Hello");
y = gDeepCopy(x);
/*  x = "Hello"  */
@end group
@end example
@sp 1
See also:  @code{New::String, Copy::String}
@end deffn







@deffn {DeepDispose} DeepDispose::String
@sp 2
@example
@group
r = gDeepDispose(i);

object  i;
object  r;     /*  NULL  */
@end group
@end example
This method is used to dispose of an instance of the String class.
It performs the same operation as @code{Dispose}.
@example
@group
@exdent Example:

object  x;

x = gNewWithStr(String, "Hello World");
gDeepDispose(x);
@end group
@end example
@c @sp 1
@c See also:  @code{}
@end deffn








@deffn {Dispose} Dispose::String
@sp 2
@example
@group
r = gDispose(i);

object  i;
object  r;     /*  NULL  */
@end group
@end example
This method is used to dispose of an instance of the String class.
It performs the same operation as @code{DeepDispose}.
@example
@group
@exdent Example:

object  x;

x = gNewWithStr(String, "Hello World");
gDispose(x);
@end group
@end example
@c @sp 1
@c See also:  @code{}
@end deffn


















@deffn {Drop} Drop::String
@sp 2
@example
@group
i = gDrop(i, num);

object  i;
int     num;
@end group
@end example
This method is used to drop @code{num} characters from string @code{i}.
If @code{num} is positive characters will be dropped from the beginning
of the string, otherwise, characters will be dropped from the end of the
string.  An attempt to drop more characters than exist will lead to a
zero length string.

The modified @code{String} instance passed is returned.
@example
@group
@exdent Example:

object  x, y;

x = gNewWithStr(String, "Hello World");
gDrop(x, 4);   /*  x = "o World"    */
x = gNewWithStr(String, "Hello World");
gDrop(x, -4);   /*  x = "Hello W"    */
x = gNewWithStr(String, "ABC");
gDrop(x, 5);   /*  x = ""    */
@end group
@end example
@sp 1
See also:  @code{Take::String, SubString::String}
@end deffn










@deffn {Equal} Equal::String
@sp 2
@example
@group
r = gEqual(i, obj);

object  i;
object  obj;  /*  or char *  */
int     r;
@end group
@end example
This method is used to determine the equality of the values represented
by @code{i} and @code{obj}.  If they contain the same string values a
1 is returned and 0 otherwise.

@code{obj} may be a C language @code{char *} or an instance of the
@code{String} class.  Dynace can tell the difference between them at
runtime.  If it is a valid Dynace object the system will assure that it is
a kind of @code{String}, otherwise it is assumed to be a C string.
@example
@group
@exdent Example:

object  x;
int     r;

x = gNewWithStr(String, "Hello");
r = gEqual(x, "Hello");                /*  r = 1  */
r = gEqual(x, gNewWithStr(String, "Hello"));  /*  r = 1  */
r = gEqual(x, "World");                /*  r = 0  */
@end group
@end example
@sp 1
See also:  @code{Compare::String}
@end deffn














@deffn {Hash} Hash::String
@sp 2
@example
@group
val = gHash(i);

object  i;
int     val;
@end group
@end example
This method is used by the generic container classes to obtain hash
values for the object.  @code{val} is a hash value between 0 and a large
integer value.
@c @example
@c @group
@c @exdent Example:
@c
@c @end group
@c @end example
@sp 1
See also:  @code{Compare::String}
@end deffn








@deffn {JustifyCenter} JustifyCenter::String
@sp 2
@example
@group
i = gJustifyCenter(i);

object  i;
@end group
@end example
This method is used to center justify the text within a string.  The length
of the string is not modified, only the position of the characters within
the string.  It is often used in conjunction with @code{Take::String} in
order to format text.

The modified @code{String} instance passed is returned.
@example
@group
@exdent Example:

object  x;

x = gNewWithStr(String, "ABD  DEF  DD      ");
gJustifyCenter(x);   /*  x = "   ABD  DEF  DD   "    */
x = gNewWithStr(String, "ABD  DEF");
gJustifyCenter(gTake(x, 14));   /*  x = "   ABD  DEF   "    */
@end group
@end example
@sp 1
See also:  @code{Take::String, JustifyRight::String,}
@iftex
@hfil @break @hglue .64in       
@end iftex
@code{JustifyLeft::String}
@end deffn






@deffn {JustifyLeft} JustifyLeft::String
@sp 2
@example
@group
i = gJustifyLeft(i);

object  i;
@end group
@end example
This method is used to left justify the text within a string.  The length
of the string is not modified, only the position of the characters within
the string.  It is often used in conjunction with @code{Take::String} in
order to format text.

The modified @code{String} instance passed is returned.
@example
@group
@exdent Example:

object  x;

x = gNewWithStr(String, "    ABD  DEF  DD");
gJustifyLeft(x);   /*  x = "ABD  DEF  DD    "    */
x = gNewWithStr(String, "ABD  DEF");
gJustifyLeft(gTake(x, 10));   /*  x = "ABD  DEF  "    */
@end group
@end example
@sp 1
See also:  @code{Take::String, JustifyCenter::String,}
@iftex
@hfil @break @hglue .64in       
@end iftex
@code{JustifyRight::String}
@end deffn











@deffn {JustifyRight} JustifyRight::String
@sp 2
@example
@group
i = gJustifyRight(i);

object  i;
@end group
@end example
This method is used to right justify the text within a string.  The length
of the string is not modified, only the position of the characters within
the string.  It is often used in conjunction with @code{Take::String} in
order to format text.

The modified @code{String} instance passed is returned.
@example
@group
@exdent Example:

object  x;

x = gNewWithStr(String, "ABD  DEF  DD    ");
gJustifyRight(x);   /*  x = "    ABD  DEF  DD"    */
x = gNewWithStr(String, "ABD  DEF");
gJustifyRight(gTake(x, 10));   /*  x = "  ABD  DEF"    */
@end group
@end example
@sp 1
See also:  @code{Take::String, JustifyCenter::String,}
@iftex
@hfil @break @hglue .64in       
@end iftex
@code{JustifyLeft::String}
@end deffn













@deffn {NumbPieces} NumbPieces::String
@sp 2
@example
@group
n = gNumbPieces(i, d);

object  i;
char    d;  /*  delimiter         */
int     n;  /*  number of pieces  */
@end group
@end example
This method is used to determine the number of sub-strings located in
@code{i}.  It is assumed that @code{i} is a string which contains the
delimiter character @code{d}.  That delimiter character is used to
logically split string @code{i} into several sub-strings.
@example
@group
@exdent Example:

object  x;
int     n;

x = gNewWithStr(String, "ABD,DEF,DD");
n = gNumbPieces(x, ',');  /*  n = 3  */
@end group
@end example
@sp 1
See also:  @code{Piece::String}
@end deffn


























@deffn {Piece} Piece::String
@sp 2
@example
@group
r = gPiece(i, d, n);

object  i;
char    d;  /*  delimiter     */
int     n;  /*  piece wanted  */
object  r;  /*  sub-string    */
@end group
@end example
This method is used to create a new string object which will contain a
copy of a sub-string of @code{i}.  It is assumed that @code{i} is a
string which contains the delimiter character @code{d}.  That delimiter
character is used to logically split string @code{i} into several
sub-strings.  @code{n} is used to specify which sub-string is desired.

This method returns a new string object which will contain a copy of the
desired sub-string.  A @code{NULL} is returned if @code{n} is greater
than the number of sub-strings in @code{i}.
@example
@group
@exdent Example:

object  x, y;

x = gNewWithStr(String, "ABD,DEF,DD");
y = gPiece(x, ',', 0);  /*  y="ABD"  */
y = gPiece(x, ',', 1);  /*  y="DEF"  */
y = gPiece(x, ',', 2);  /*  y="DD"   */
@end group
@end example
@sp 1
See also:  @code{NumbPieces::String}
@end deffn























@deffn {PrintLength} PrintLength::String
@sp 2
@example
@group
sz = gPrintLength(i, ts);

object  i;
int     ts;
int     sz;
@end group
@end example
This method is used to obtain the length of the string associated with
the instance of the String class passed when it is printed.  This
method takes into account tabs, backspaces and return characters in
its calculations.

@code{ts} is the tab stop size that the method should use.

Note that this is one of the few generics which doesn't return a Dynace
object.  It returns an integer.
@example
@group
@exdent Example:

object  x;
int     sz;

x = gNewWithStr(String, "Hello\tWorld\bx");
sz = gPrintLength(x, 8);
/*  sz = 13   */
@end group
@end example
@sp 1
See also:  @code{Size::String}
@end deffn











@deffn {Process} Process::String
@sp 2
@example
@group
i = gProcess(i);

object  i;
@end group
@end example
This method is used to process a string with embedded escape sequences.
The escape sequences will be processed and converted into the appropriate
control characters.  For example @code{"\r"} will be converted in to a return
control character.  All the ANSI standard escape sequences are supported.
In addition, @code{"\e"} gets converted into the escape character.
@code{"\^L"} (where L may be any letter - upper or lower case) gets
converted into the corresponding control character, for example
@code{"\^G"} will get converted to the control-G character.
@code{"\dnnn"} works like @code{"\xnnn"} except that @code{nnn} is
treated as a decimal number.
@example
@group
@exdent Example:

object  x;

x = gNewWithStr(String, "Hello\\tWorld\\bx");
gProcess(x);
/*  x will now contain the real control characters  */
@end group
@end example
@sp 1
See also:  @code{PrintLength::String}
@end deffn













@deffn {RemoveMask} RemoveMask::String
@sp 2
@example
@group
i = gRemoveMask(i, mask, text);

object  i;
char    *mask;
char    *text;
@end group
@end example
This method is used to remove @code{mask} from @code{text} and set the
resulting value to @code{i}.  If @code{NULL} or an empty string
(@code{""}) is passed in to @code{mask} then @code{i} has its string
value changed to @code{text}.  If @code{NULL} or an empty string
(@code{""}) is passed in to @code{text} then @code{mask} is removed from
the current string value in @code{i}.

The value returned is the @code{String} instance passed (@code{i}).
@example
@group
@exdent Example:

object  x;
char    mask = "<(###) ###-####";

x = gNewWithStr(String, "(123) 456-7890");
gRemoveMask(x, mask, NULL);            /*x="1234567890"    */
gRemoveMask(x, mask, "(987)654-3210"); /*x="9876543210"    */
gRemoveMask(x, NULL, "(123) 456-7890");/*x="(123) 456-7890"*/
@end group
@end example
@sp 1
See also:  @code{gMaskFunction::String, gSetMaskFiller::String,}
@iftex
@hfil @break @hglue .64in      
@end iftex
@code{gApplyMask::String}
@end deffn













@deffn {SetMaskFiller} SetMaskFiller::String
@sp 2
@example
@group
r = gSetMaskFiller(i, ch);

object  i;
char    ch;
char    r;
@end group
@end example
This method is used to set the character to be used to fill the unused
space in a masked string.  This character defaults to be a space but is
frequently set to be the underscore @code{(`_')} character.  This method
returns the previous mask filler character.

@example
@group
@exdent Example:

object  x;
char    r;

x = gNewWithStr(String, "1234567");
gApplyMask(x, "<(###) ###-####", NULL);
      /* x = "(   ) 123-4567" */
r = gSetMaskFiller(x, '_');
      /* r = ' ' (space)      */
gApplyMask(x, "<(###) ###-####", NULL);
      /* x = "(___) 123-4567" */
@end group
@end example
@sp 1
See also:  @code{gMaskFunction::String, gApplyMask::String,}
@iftex
@hfil @break @hglue .64in      
@end iftex
@code{gRemoveMask::String}
@end deffn












@deffn {Size} Size::String
@sp 2
@example
@group
sz = gSize(i);

object  i;
int     sz;
@end group
@end example
This method is used to obtain the length of the string associated with
the instance of the String class passed.

Note that this is one of the few generics which doesn't return a Dynace
object.  It returns an integer.
@example
@group
@exdent Example:

object  x;
int     sz;

x = gNewWithStr(String, "Hello World");
sz = gSize(x);
/*  sz = 11   */
@end group
@end example
@sp 1
See also:  @code{PrintLength::String}
@end deffn










@deffn {StringRepValue} StringRepValue::String
@sp 2
@example
@group
s = gStringRepValue(i);

object  i;
object  s;
@end group
@end example
This method is used to generate an instance of the @code{String} class
which represents the value associated with @code{i}.  This is often
used to print or display the value.  It is also used by
@code{PrintValue::Object} and indirectly by @code{Print::Object}
(two methods useful during the debugging phase of a project)
in order to directly print an object's value.
@example
@group
@exdent Example:

object  x;
object  s;

x = gNewWithStr(String, "Hello");
s = gStringRepValue(x);
  /* s represents "Hello" (with quotes)  */
@end group
@end example
@sp 1
See also:  @code{StringValue::String, PrintValue::Object, Print::Object}
@end deffn











@deffn {StringValue} StringValue::String
@sp 2
@example
@group
val = gStringValue(i);

object  i;
char    *val;
@end group
@end example
This method is used to obtain the @code{char *} value associated with an
instance of the String class.  Note that this is one of the few
generics which doesn't return a Dynace object.  It returns a char * string.
@example
@group
@exdent Example:

object  x;
char    *val;

x = gNewWithStr(String, "Hello World");
val = gStringValue(x);
@end group
@end example
@sp 1
See also:  @code{New::String, ChangeValue::String}
@end deffn






@deffn {StripCenter} StripCenter::String
@sp 2
@example
@group
i = gStripCenter(i);

object  i;
@end group
@end example
This method is used to strip off blank characters (space, tab, new line,
carriage return) from both sides of a string.  The return value is the
modified @code{String} instance passed.
@example
@group
@exdent Example:

object  x, y;

x = gNewWithStr(String, "    ABD  DEF  DD    ");
gStripCenter(x);   /*  x = "ABD  DEF  DD"    */
@end group
@end example
@sp 1
See also:  @code{StripRight::String, StripLeft::String}
@end deffn












@deffn {StripLeft} StripLeft::String
@sp 2
@example
@group
i = gStripLeft(i);

object  i;
@end group
@end example
This method is used to strip off blank characters (space, tab, new line,
carriage return) from the left side of a string.  The return value is the
modified @code{String} instance passed.
@example
@group
@exdent Example:

object  x, y;

x = gNewWithStr(String, "    ABD  DEF  DD    ");
gStripLeft(x);   /*  x = "ABD  DEF  DD    "    */
@end group
@end example
@sp 1
See also:  @code{StripRight::String, StripCenter::String}
@end deffn










@deffn {StripRight} StripRight::String
@sp 2
@example
@group
i = gStripRight(i);

object  i;
@end group
@end example
This method is used to strip off blank characters (space, tab, new line,
carriage return) from the right side of a string.  The return value is the
modified @code{String} instance passed.
@example
@group
@exdent Example:

object  x, y;

x = gNewWithStr(String, "    ABD  DEF  DD    ");
gStripRight(x);   /*  x = "    ABD  DEF  DD"    */
@end group
@end example
@sp 1
See also:  @code{StripLeft::String, StripCenter::String}
@end deffn























@deffn {SubString} SubString::String
@sp 2
@example
@group
so = gSubString(i, beg, num);

object  i;
int     beg, num;
object  so;
@end group
@end example
This method is used to create a new instance of the @code{String} class
which is initialized to a copy of a sub-string of instance @code{i}.
@code{beg} indicates the beginning position number of the string to be copied,
index origin zero.  @code{num} is the number of characters to copy.

If @code{beg} is negative, @code{beg} will be converted to a beginning point
relative to the end of the string.  Therefore, a @code{-1} would indicate
a starting point of the last character in the string and @code{-2} would
indicate a starting point of the second to the last character in the string.
Zero indicates to start at the beginning of the string.

If @code{num} is negative, @code{-num} characters @emph{before} @code{beg}
will be taken.

The new @code{String} instance is returned.
@example
@group
@exdent Example:

object  x, y;

x = gNewWithStr(String, "Hello World");
y = gSubString(x, 1, 3);   /*  y = "ell"    */
y = gSubString(x, -5, 5);  /*  y = "World"  */
@end group
@end example
@sp 1
See also:  @code{Take::String, Drop::String}
@end deffn











@deffn {Take} Take::String
@sp 2
@example
@group
i = gTake(i, num);

object  i;
int     num;
@end group
@end example
This method is used to change the length of the string associated with
the @code{String} instance @code{i}.  @code{num} is the new length of
the string.  If @code{num} is less than the previous length of the
string, the string will be terminated to the shorter length.  If
@code{num} is greater than the previous length, the string will be padded
with spaces until it is @code{num} characters long.

If @code{num} is negative, characters will be taken from the end of the
string and padding will be performed at the beginning of the string.

The modified @code{String} instance passed is returned.
@example
@group
@exdent Example:

object  x, y;

x = gNewWithStr(String, "Hello World");
gTake(x, 4);   /*  x = "Hello"    */
x = gNewWithStr(String, "Hello World");
gTake(x, -4);   /*  x = "orld"    */
x = gNewWithStr(String, "ABC");
gTake(x, 5);   /*  x = "ABC  "    */
x = gNewWithStr(String, "ABC");
gTake(x, -5);   /*  x = "  ABC"    */
@end group
@end example
@sp 1
See also:  @code{Drop::String, SubString::String}
@end deffn















@deffn {ToLower} ToLower::String
@sp 2
@example
@group
i = gToLower(i);

object  i;
@end group
@end example
This method is used to convert all the characters in @code{String}
instance @code{i} to lower case.  Non-alphabetic characters are left
unchanged.  The instance passed is returned.
@example
@group
@exdent Example:

object  x;

x = gNewWithStr(String, "Hello World");
gToLower(x);
/*  x = "hello world"  */
@end group
@end example
@sp 1
See also:  @code{ToUpper::String}
@end deffn












@deffn {ToUpper} ToUpper::String
@sp 2
@example
@group
i = gToUpper(i);

object  i;
@end group
@end example
This method is used to convert all the characters in @code{String}
instance @code{i} to upper case.  Non-alphabetic characters are left
unchanged.  The instance passed is returned.
@example
@group
@exdent Example:

object  x;

x = gNewWithStr(String, "Hello World");
gToUpper(x);
/*  x = "HELLO WORLD"  */
@end group
@end example
@sp 1
See also:  @code{ToLower::String}
@end deffn




