@c -*-texinfo-*-

@c  Copyright (c) 1996 Blake McBride
@c  All rights reserved.
@c
@c  Redistribution and use in source and binary forms, with or without
@c  modification, are permitted provided that the following conditions are
@c  met:
@c
@c  1. Redistributions of source code must retain the above copyright
@c  notice, this list of conditions and the following disclaimer.
@c
@c  2. Redistributions in binary form must reproduce the above copyright
@c  notice, this list of conditions and the following disclaimer in the
@c  documentation and/or other materials provided with the distribution.
@c
@c  THIS SOFTWARE IS PROVIDED BY THE COPYRIGHT HOLDERS AND CONTRIBUTORS
@c  "AS IS" AND ANY EXPRESS OR IMPLIED WARRANTIES, INCLUDING, BUT NOT
@c  LIMITED TO, THE IMPLIED WARRANTIES OF MERCHANTABILITY AND FITNESS FOR
@c  A PARTICULAR PURPOSE ARE DISCLAIMED. IN NO EVENT SHALL THE COPYRIGHT
@c  HOLDER OR CONTRIBUTORS BE LIABLE FOR ANY DIRECT, INDIRECT, INCIDENTAL,
@c  SPECIAL, EXEMPLARY, OR CONSEQUENTIAL DAMAGES (INCLUDING, BUT NOT
@c  LIMITED TO, PROCUREMENT OF SUBSTITUTE GOODS OR SERVICES; LOSS OF USE,
@c  DATA, OR PROFITS; OR BUSINESS INTERRUPTION) HOWEVER CAUSED AND ON ANY
@c  THEORY OF LIABILITY, WHETHER IN CONTRACT, STRICT LIABILITY, OR TORT
@c  (INCLUDING NEGLIGENCE OR OTHERWISE) ARISING IN ANY WAY OUT OF THE USE
@c  OF THIS SOFTWARE, EVEN IF ADVISED OF THE POSSIBILITY OF SUCH DAMAGE.

@chapter Mechanics
This chapter describes the actual steps necessary to use build and use
Dynace.  First the steps necessary to configure, compile and link Dynace
and Dynace applications is described, then the steps necessary to use
existing classes is described.  Once this is understood, the creation of
new classes is described.  Finally, Dynace customization and special
techniques are discussed.


@section Setup, Configuration & Build Procedures
This section, in concert with the examples, describes the tools and
actual steps necessary to build Dynace and create Dynace application
programs.  The best method of learning is to go through the example
programs located under the @code{\DYNACE\EXAMPLES} directory in the
order in which they appear.

@subsection Build Utilities
All the utilities used to build Dynace applications are contained within
the Dynace package and the command line utilities (compiler, linker,
etc.) which come with your compiler.  Dynace utilizes special utilities
and build procedures designed to intelligently minimize the need for
re-compilation.  These procedures are too complex for IDEs which come
with the various compilers.  For this reason the IDE environments which
come with your compiler are not used.  Instead, the included
@code{dmake} make utility is used.

Having said that, however, you may still use the IDE for editing and
debugging.  You may also be able to find a way of integrating the
supplied make files and procedures into your favorite IDE.

@subsection DMAKE Build Setup
This section lists environment variables which must be set for the
DMAKE build.

The @code{\DYANCE\BIN} directory should be part of the search path used
by your system for executable programs (the @code{PATH} environment
variable).

The following environment variables should be set as follows (modify as
appropriate):
@example
@group
set MAKESTARTUP=c:\dynace\utils\startup.mk
set TMPDIR=c:\tmp
set DOS4G=quiet
set DOS16M=:4M
@end group
@end example

The @code{DOS4G} and @code{DOS16M} environment variables are only necessary
when running the 32 bit DOS version of @code{dpp.exe} which has been
compiled with the WATCOM compiler.  Win32 or other versions of @code{dpp}
do not require these variables.

Although not required by the example programs, Microsoft users may find
it convenient to add @code{\DYNACE\INCLUDE} and @code{\DYNACE\LIB} to
your compiler's include and library search paths respectively.

Borland users will need to adjust the @code{BORLAND_HOME} path setting
in the @code{B16.DM} and @code{B32.DM} makefiles located in all the
example and Dynace source code directories.


@subsection Building Dynace From Scratch
The procedure used to build Dynace from scratch is fully described
in 
@iftex
@hfil @break 
@end iftex
@code{\DYNACE\DOCS\BUILD.txt}.  See that file for build instructions.

New users will want to read all the files in the @code{docs} directory.


@subsection Examples Setup
All examples are buildable under all supported environments and build
procedures including 16 and 32 bit environments.  The only difference is
the makefile used.



@subsection Example Files
This section documents the files which are contained in each example
program.

@table @asis
@item README
This file describes the object of the example.  It should be read
first.

@item MAIN.C
This file is the complete source code to the example program.  It is
fully commented and describes all aspects of the current example which
are unique.

@end table


In addition to the above, the following build/system specific make 
files are used.

@table @asis
@item M32.DM
DMAKE makefile for Microsoft Visual C 32 bit

@item MAKEFILE.UNX
GNU MAKE makefile for Unix or Linux systems

@item MKFILE
Makefile for the Plan 9 operating system

@item M16.DM
DMAKE makefile for Microsoft Visual C 16 bit

@item B16.DM
DMAKE makefile for Borland C 16 bit

@item B32.DM
DMAKE makefile for Borland C 32 bit

@item W32.DM
DMAKE makefile for WATCOM C 32 bit

@item S32.DM
DMAKE makefile for Symantec C 32 bit
@end table


Once an application is built, the only file needed to run it is
@code{MAIN.EXE}.



@subsection Building The Examples
In order to build with the DMAKE command line utility execute the
following command.

@example
dmake -f MAKEFILE.DM
@end example

where @code{MAKEFILE.DM} is one of the DMAKE makefiles listed above.

Alternatively, you can use the following command to create a debug
version of the example:

@example
dmake -f MAKEFILE.DM  DEBUG=1
@end example



@subsection Debugging
When @code{dpp} strategy 1 is selected Dynace uses a small amount of
assembler.  This is not true for the default strategy of 2, however,
when strategy 1 is used there are a couple of procedures which, when
followed, will easy the debugging process.

First compile your application with debugging information.  Then when you
wish to debug the application, set a break point at the beginning of any
methods you wish to debug.  You may then step through your program as
normal and when you enter a generic, the system will stop at the
methods in which you placed the break points.

Once you are fully into the method, you may access the instances
variables through the ``@code{iv}'' structure pointer and the
class variables through the ``@code{cv}'' structure pointer.



@subsection Building Your Own Application
The best method of building your own application would be to start
with one of the example programs and proceed from there.  This way
all the compiler, linker and other options will be preset.  The best
examples to start with would be one of the ones which define a new
class.  Their makefiles are designed to handle class files in
a very convenient and automated way.

@subsection DMAKE
DMAKE is a very powerful, portable and enhanced make utility developed
by Dennis Vadura.  It is freely distributable and not owned by
Algorithms Corporation or Blake McBride.  Complete documentation for this utility
is contained in @code{\DYNACE\DOCS\DMAKE.MAN}





@page
@section Using Dynace Classes
Dynace usage can be divided into two categories: @emph{class user} or
@emph{class creation}.  All a class user has to do to use Dynace classes
is include a single file in the source code and call a function to
initialize the system.  After these two easy steps the class user has full
access to the entire set of existing classes.  A class user need not learn
any special syntax or procedures.  Learning to use the existing classes
is the same as learning to use any other library of routines.

A @emph{class creator} will have to understand all aspects of being a
class user first.  In addition, a class creator will have to understand
the steps necessary to define classes, methods and generics.  These
steps are well defined and simple to use.

This section describes the actual steps necessary to @emph{use} a
pre-existing class.  The next section will discuss the steps necessary
to create new classes.

@subsection External Naming Conventions
External naming conventions refers to those names that the programmer
defines such as a new class name, an instance variable name or a method
or generic name.  By contrast, internal naming conventions refers to
those names which are generated by the Dynace preprocessor (@code{dpp}).
These internal names are often made up from modifications or
concatenations of the external names.

In Dynace, generic function names, by convention only, always begin with
a lower case ``@code{g}'' or ``@code{v}'', followed by an upper case
letter, and then optionally followed by other letters (of any case),
numbers, or underscores.  This convention allows a user to immediately
know which functions may be used polymorphically or overloaded.

Generics which begin with ``@code{g}'', by convention only, have compile
time (as well as runtime) argument checking turned on and often, but not
always, have a fixed number of arguments.

Generics which begin with ``@code{v}'', by convention only, have compile
time argument checking turned off and are often, but not always,
associated with generics which take a variable number of arguments.  Use
of generics with argument checking turned off allows them to be
associated with methods with arbitrary argument signatures.  It is up to
the system and the programmer to assure that the correct arguments are
passed.  Dynace provides ample tools necessary to assure correct
arguments at runtime.  In fact, the runtime checking facilities are much
more reliable and fool proof than compile time argument checking.  The
only disadvantage is that you don't know a problem exists until you run
the system.  If, however, an error is discovered it can be handled many
different ways and never crashes the system.

Dynace uses the ``@code{g}'' type of generic as much as possible.
However, some methods, such as indexing into a variable dimension
array or @code{printf} like functionality, are implemented as
``@code{v}'' generics.


It's all right if several methods (associated with different classes)
have the same name because methods are implemented as static C
functions.  So there is no name conflict.

Classes normally begin with a capital letter.  Instance variables, by
convention only, begin with ``@code{i}'' and are followed by an
upper case letter.  Likewise, class variables, by convention only, begin
with ``@code{c}'' and are followed by an upper case letter.

In the index the notation will specify a generic (minus the normal
leading ``@code{g}'' or ``@code{v}''), two colons and its associated
class.  For example @code{ChangeValue::Number} specifies a generic
called @code{gChangeValue} which is associated with the @code{Number}
class.

The above naming scheme is used by the Dynace system, however, it
is not enforced.  You may use any naming scheme you wish.

@subsection Internal Naming Conventions
External naming conventions refers to those names that the programmer
defines such as a new class name, an instance variable name or a method
or generic name.  By contrast, internal naming conventions refers to
those names which are generated by the Dynace preprocessor (@code{dpp}).
These internal names are often made up from modifications or
concatenations of the external names.

The following table lists the internal name structure used by Dynace.
The associated source external names for example purposes are the class
@code{String} and the generic @code{gNew}.

@table @code
@item String
a macro which evaluates to the class object but also automatically
initializes the class if it hasn't already been initialized
@item String_c
the actual variable which contains the class object (@code{NULL} if the
class hasn't been initialized yet
@item String_t
a typedef defining a String class (always object), used for documentation
purposes only
@item String_cv
pointer to the actual locally defined class variable structure
@item String_cv_t
a typedef which defines the locally defined class variable structure
@item String_iv_t
a typedef which defines the locally defined instance variable structure
@item String_initialize
the name of the class initialization function generated by @code{dpp}
@item gNew_t
a typedef defining a pointer to a function with the return type and argument
signature of the generic
@item gNew_g
a Dynace object representing the generic
@item String_im_gNew
default instance method name for methods with argument checking turned on
(imeth)
@item String_cm_gNew
default class method name for methods with argument checking turned on
(cmeth)
@item String_ivm_gNew
default instance method name for methods with argument checking turned off
(ivmeth)
@item String_cvm_gNew
default class method name for methods with argument checking turned off
(cvmeth)
@end table


@subsection Class Hierarchy
The only class in Dynace which has no superclasses is called
@code{Object}.  All classes in Dynace eventually inherit from the class
@code{Object}.  The Object class defines those features that are common
to all objects.  The class @code{Behavior} inherits from Object and
defines the functionality common to classes and metaclasses.  The class
@code{Class} inherits from Behavior, and describes the functionality
particular to all classes.  The class @code{MetaClass} also inherits
from Behavior, and defines the operations particular to metaclasses.
There is a class called @code{Dynace} which inherits from Object, and is
used to effect various aspects of the Dynace kernel.  There are also
classes called @code{Method} and @code{GenericFunction}, which both inherit
from Object, and perform the expected functions.

The above class hierarchy can be depicted as follows:

@example
@group
     Object
          Behavior
               Class
               MetaClass
          Dynace
          Method
          GenericFunction
@end group
@end example

These classes comprise the entire Dynace kernel.  Other facilities of Dynace,
such as threads, pipes, dictionaries, are supplied as Dynace classes.
Although an in depth understanding of each of the above classes is
unnecessary, it is important to fully understand the relationship between
the inheritance of the above classes, and the indentation of the diagram.

@subsection Example Code
The following example illustrates a complete program which uses Dynace.
Each line is explained in the following sub-sections.

@example
@group
#include "generics.h"

main(int argc, char *argv[])
@{
        /*  declare some variables which will hold objects  */
        object  name, pi, age;

        /*  initialize the Dynace kernel and class library  */
        InitDynace(&argc);

        /*  optional:  start automatic garbage collector  */
        gSetMemoryBufferArea(Dynace, 40000L);


        /*  set name to an instance of the String class  */
        name = gNewWithStr(String, "Tom Swift");

        /*  set pi to an instance of the DoubleFloat class  */
        pi   = gNewWithDouble(DoubleFloat, 3.14159265358979);

        /*  set age to an instance of the ShortInteger class  */
        age  = gNewWithInt(ShortInteger, 32);

        /*  tell some objects to print themselves  */
        gPrint(name, stdoutStream);
        gPrint(age, stdoutStream);

        /*  change the age  */
        gChangeShortValue(age, 33);
        


        /*  manually dispose of the objects (not necessary if the
            garbage collector is enabled)  */
        gDispose(name);
        gDispose(pi);
        gDispose(age);
@}
@end group
@end example

@subsection Include File
Any source module which uses any Dynace objects must include @code{generics.h}.
This may be accomplished with the following line:
@example
#include "generics.h"
@end example

@subsection Declarations
All objects (classes, instances, methods...) in Dynace are @emph{objects}.
The declaration used in Dynace is @code{object}.  Therefore, if you wanted
to declare three objects in Dynace you could use the following:
@example
        object   a, b, c;
@end example
This declaration may be used anywhere you can define any other type of
variable in the C language.

@subsection Initializing Dynace
The first thing which must be done prior to using any classes in Dynace is
to initialize the Dynace kernel and system classes.

This initialization @emph{must} be called from the @code{main()}
function.  The initialization function @emph{must} also be passed the
address of the @code{argc} argument to the @code{main()} function.  This
information is required by the Dynace garbage collector.

Use of the following is required to initialize the Dynace system:
@example
        InitDynace(&argc);
@end example


@subsection Generic Functions
In Dynace, generic function names, by convention only, always begin with
a lower case ``@code{g}'' or ``@code{v}'', followed by an upper case
letter, and then optionally followed by other letters (of any case),
numbers, or underscores.  This convention allows a user to immediately
know which functions may be used polymorphically or overloaded.

Generic functions are used like any other C language function.  However,
generic functions always have at least one argument.  This first argument,
is the object which will be performing the requested method.  Any remaining
arguments will be passed to the appropriate method.

When using variable argument generics (those beginning with
``@code{v}'') the number and types of arguments may be different from
one call to a particular generic function call to another.  This is
because different methods, with different argument requirements, may be
called.

@subsection Creating Objects
By convention, the generic function used to create new objects is called
@code{gNew()} or @code{vNew()}.  The first argument to either should be
the class of the new object (or instance) to be created.  Any, possible,
remaining arguments (in the case of @code{vNew}) will be the ones
required by a particular class.

If compile time argument checking is used then the @code{gNew} function
will possibly be renamed to avoid argument conflicts.

The following are examples of some instances being created:
@example
@group
        object   name, pi, age;

        name = gNewWithStr(String, "Tom Swift");
        pi   = gNewWithDouble(DoubleFloat, 3.14159265358979);
        age  = gNewWithInt(ShortInteger, 32);
@end group
@end example

There are a number of interesting things about this example.  First of
all, the syntax is normal C language syntax.  Second, the three objects
created, each of a different type, had the same declaration.  Note also,
the standard naming of the generic function @code{gNewXXXX()}.

@subsection Using Objects
Once an object is created it may be used as arguments to generic functions.
For example:
@example
@group
        object   myInt;

        myInt = gNewWithInt(ShortInteger, 42);
        gPrint(myInt, stdoutStream);
        gChangeShortValue(myInt, 68);
        gPrint(myInt, stdoutStream);
@end group
@end example
When the @code{gPrint()} generic function is called, as in the above
example, Dynace checks the class of @code{myInt}, and finds that it is of type
@code{ShortInteger}.  It then finds a method which is associated with the
@code{gPrint()} generic function / @code{ShortInteger} class combination,
and evokes it.  This association is defined by the person who wrote the
@code{ShortInteger} class.

@subsection Disposing of Objects
All objects in Dynace are allocated from the heap.  Therefore, when an object
is no longer needed, it must be returned to the heap.  If this is not
done you would eventually run out of heap space.  Dynace has two methods
of accomplishing this task, disposal generics and the garbage collector.

@subsubsection Disposal Generics

One method Dynace uses to release unused objects is through an explicit
call to a disposal generic.  There are two common disposal generics.
One is called @code{gDispose} and the other is @code{gDeepDispose}.

@code{gDispose} is typically used to dispose of a single object and any
associated allocated space it may reference.

The @code{gDeepDispose} generic, in addition to performing the same
functions as the @code{gDispose} generic, is typically used to dispose
of whole group of related objects at the same time.  This method will
typically recursively call the @code{gDeepDispose} generic on all
objects being referenced within the object being disposed.

Both @code{gDispose} and @code{gDeepDispose} always return @code{NULL}.
This is done in order to allow the disposal generic to null out the
variable which holds the object in the same expression which disposes
of it.

The following is an example of the @code{gDispose()} generic:
@example
@group
        object    age;

        age = gNewWithInt(ShortInteger, 32);
        gPrint(age, stdoutStream);
        age = gDispose(age);
@end group
@end example

@subsubsection Garbage Collection
The second, and preferred, method Dynace has for disposing of objects is
through a @emph{garbage collector}.  The garbage collector is an
integral part of Dynace which constantly monitors the applications memory
use.  Whenever certain conditions arise, such as low memory conditions,
the garbage collector automatically, and transparently kicks in.  The
garbage collector marks all objects which are directly or indirectly
accessible via the @emph{automatic} variables in the program.  It also
marks any objects which are referenced by global or static variables
which have been registered with Dynace.  Dynace also automatically marks all
classes, methods, and generic function objects.  After this marking
phase, the Dynace garbage collector goes through all objects and frees
those that are not marked.

The simple explanation is that by using the Dynace garbage collector, you
never have to worry about freeing any objects which are no longer
needed.  The garbage collection process requires no additional coding
on the developers part.  It runs entirely automatically and
transparently to the application program.

A few things to remember, though, the Dynace garbage collector may be used
with the @code{gDispose} generic function.  The garbage collector is off
by default, and must be turned on (as shown below).  Only those objects
which are referenced by automatic variables, and those globals and
statics which have been registered with the garbage collector (via the
gRegisterMemory generic), will be marked.  Therefore, any objects which
are only referenced by unregistered globals or statics will be disposed
of by the garbage collector.

Note that under Win32, Dynace has special code and can automatically
detect and mark unregistered globals or statics.  Therefore manual
registration of these memory locations is not needed under Win32.

The garbage collector may be turned on with the following line (after
the Dynace system has been initialized):
@example
        gSetMemoryBufferArea(Dynace, 40000L);
@end example

Dynace offers an alternative garbage collection scheme which has important
tradeoffs.  See the sub-section on the Boehm Garbage Collector under the
Dynace Customization & Special Techniques section.

@subsection Generics Files
In any given program, only a single source file should contain all the
generic functions used throughout the entire application.  This includes
the generics used by the Dynace kernel, the class library, and any
application specific generics.  The file used to define all the generics
is called @code{generics.c}.  There is also a file called @code{generics.h}
which provides for the external declarations for all the classes and
generics.  Both of these files are automatically generated by a program
called @code{dpp}.

See the section documenting the @code{dpp} program for further
details.


@subsection Compiling & Linking
This section describes the necessary tools and steps necessary to create
an application which uses the Dynace system.

@subsubsection Files

In order to compile and link programs which use the Dynace kernel and class
library you will need the following files:
@table @code

@item dynl.h
This file is included with the Dynace system and located in the
include directory.

@item generics.h
If your application creates its own classes, you will need a custom
@code{generics.h} file located in your application directory.
This file is automatically generated by the @code{dpp} program.
Otherwise, the @code{generics.h} file included with Dynace may
be used.

@item generics.c
This is a file generated by @code{dpp} from the application specific
@code{generics.h} file.  It should be created and kept in the
applications source directory.  It is only needed if your application
creates its own classes.

@item dpp.exe
This program is used to generate the @code{generics.c} and
@code{generics.h} files in the applications source directory.
@code{dpp.exe} should be kept somewhere in your normal path for
executable programs.

@item *.lib
The Dynace library files are located under the @code{lib} directory.
These libraries are specific to each supported compiler and whether
they are for 16 or 32 bit versions of the compiler.  For 16 bit
compilers, only the large memory model is supported.

@end table

@subsubsection Compiling & Linking
The best way to compile and link your application is to start with one
of the makefiles included with the example programs included with Dynace.
There are specific make files for each supported compiler.

@subsection Learning Dynace Classes
Once the basics of using Dynace objects, and the mechanics of creating a
program are understood, a study of the Dynace library, including what
facilities it has to offer, and how to use it, are strongly recommended.
Nothing will get you up and running with the system faster.  The true
value of Dynace cannot be appreciated until a full understanding of the Dynace
class library, and what facilities it has to offer, as well as the
understanding of the simplicity and power offered by creating your own
classes has been achieved.

@page
@section Defining Dynace Classes
This section describes the syntax and procedure necessary to create
Dynace classes.  It is recommended that all the information in the
previous sections be understood prior to continuing in this section.  It
is also recommended that some experience with the class library be
accomplished.

Each new class in Dynace is defined in its own unique source file.
Multiple classes should never be defined in a single source file.
However, it is possible to define a large class by several source files.
This process is illustrated in one of the example programs.

Dynace uses the static function ability of the C language to encapsulate
the methods.  This is why it is best to define a class in a single
source file.

Classes are defined in class definition files which have a
``@code{.d}'' file extension instead of the normal ``@code{.c}''
file extension.  These class definition files may contain normal
C code as well as the additional syntax elements described in the
next subsection.

When building the application, class definition files are processed
by the Dynace pre-processor (@code{dpp}) which produces corresponding
C source files (with a ``@code{.c}'' file extension) which may be
compiled with your normal C compiler.  At the same time @code{dpp}
also performs argument checking by comparing arguments to methods
and their associated generics.  @code{dpp} also uses this information
to produce the appropriate generic files.


The following sub-sections describe the exact syntax
necessary to define your own Dynace classes.

@iftex
@break
@end iftex

@subsection Class Definition File Syntax
Class definition files may contain normal C code without restriction.
In addition, the following syntax elements may be used to define a class,
methods and generics.


@example

Legend:

        Items between [ ] are optional
        ... indicates any number (including 0) of the preceding
        | indicates a choice
        Identifiers in caps should be replaced by the appropriate name
        Lines beginning with // are comments
        Lower case identifiers and other symbols are typed as-is

@iftex
@hrule
@end iftex

// Class definition

[CLASS-SCOPE]  defclass  CLASS-NAME
        [[:]  SUPERCLASS-NAME  [[,] SUPERCLASS-NAME]...]
[@{
        [[instance:]
                INSTANCE-VARIABLE-DECLARATIONS]
        [class:
                CLASS-VARIABLE-DECLARATIONS]
        [init:
                SPECIAL-CLASS-INITIALIZATION-FUNCTION-NAME ;]
@}];


CLASS-SCOPE = public | private

VARIABLE-DECLARATION = [TYPE]  variable ;

TYPE defaults to object

@iftex
@c @hrule
@break
@end iftex
//  Public method definition 

[public]  METHOD-TYPE [RETURN-TYPE]  GENERIC [,GENERIC]...
        [,<GENERIC>]... [: METHOD-NAME]  ( ARGUMENT-LIST )
@{
        METHOD-BODY
@}

METHOD-TYPE = imeth | cmeth | ivmeth | cvmeth

RETURN-TYPE defaults to object

Undeclared method names default to CLASS_Xm_GENERIC

ARGUMENT-LIST - types default to object,
                first arg (object self) defaults

@iftex
@hrule
@end iftex

//  Private method definition 

private METHOD-TYPE [RETURN-TYPE]  METHOD-NAME
        ( ARGUMENT-LIST )
@{
        METHOD-BODY
@}

METHOD-TYPE = imeth | cmeth | ivmeth | cvmeth

RETURN-TYPE defaults to object

ARGUMENT-LIST - types default to object,
                first arg (object self) defaults

@iftex
@hrule
@end iftex

//  Super syntax

GENERIC(super [ARG]  [, ARGN]... )

@iftex
@hrule
@end iftex

//  Comments

/*  COMMENT  */

//  COMMENT-TO-END-OF-LINE
@c @hrule
@end example

@subsubsection Class Definition
This syntax element allows for a central location for all elements
associated with the new class.  The new class's name, an optional
list of superclasses, instance variables, class variables, and
class initialization function are all defined by this structure.

All variables default to type @code{object} if not otherwise
declared.

Since most elements of a class definition are optional, the minimum
that is requires is as follows:

@example
@group
defclass   MyClass;
@end group
@end example


This defines a class called @code{MyClass} which has no specific class
or instance variables and has a single superclass (@code{Object}).

The next example defines a class which is a subclass of @code{SomeClass}.

@example
@group
defclass   MyClass : SomeClass;
@end group
@end example


The next example defines a class which is a subclass of @code{SomeClass}
and @code{SomeOtherClass}.

@example
@group
defclass   MyClass : SomeClass, SomeOtherClass;
@end group
@end example


The next example defines a class which has a couple of specific
instance variables associated with it.  This class defaults to
the single @code{Object} superclass.

@example
@group
defclass   MyClass  @{
        int     iAge;
        object  iName;
@};
@end group
@end example

The next example is the same as the previous except that since
variables default to the @code{object} type, this type declaration
may be omitted.

@example
@group
defclass   MyClass  @{
        int     iAge;
        iName;
@};
@end group
@end example


The next example adds to the previous example by adding a class variable.
Note that since there is no type explicitly declared it defaults to
@code{object}.

@example
@group
defclass   MyClass  @{
        int     iAge;
        iName;
 class:
        cPeople;
@};
@end group
@end example


The next example adds to the previous example with the addition of
a specific superclass (as shown above).

@example
@group
defclass   MyClass  :  SomeClass  @{
        int     iAge;
        iName;
 class:
        cPeople;
@};
@end group
@end example


The next example adds to the previous example with the addition of
a class initialization function.  This feature is rarely needed but allows
a specific function to be executed by Dynace when the class is being
initialized.  This is typically used to initialize class variables.

@example
@group
defclass   MyClass  :  SomeClass  @{
        int     iAge;
        iName;
 class:
        cPeople;
 init:  my_init_function;
@};
@end group
@end example


The next example illustrates the fact that the various sections
may be in any order, however, if the instance section is not first, it
must be explicitly declared.

@example
@group
defclass   MyClass  :  SomeClass  @{
 init:  my_init_function;
 class:
        cPeople;
 instance:
        int     iAge;
        iName;
@};
@end group
@end example

The default scope of class definitions is @code{private}.  Therefore
the previous example could just as well have been written as follows.

@example
@group
private  defclass   MyClass  :  SomeClass  @{
 init:  my_init_function;
 class:
        cPeople;
 instance:
        int     iAge;
        iName;
@};
@end group
@end example

Use of the @code{private} keyword is strongly discouraged.  It was only
added to complement the @code{public} scoping which is also strongly
discouraged but needed in certain circumstances.

The @code{public defclass} declaration is used to cause the Dynace
preprocessor (@code{dpp}) to create additional files which contain the
class and instance variable structures.  This is the same as the
@code{-iv} and @code{-cv} @code{dpp} options.  See the @code{dpp}
section for further explanation of this feature.



@subsubsection Public Method Definition
Public methods are only accessible through generic functions.
The body of public methods is identical with normal C functions.
The main difference between a C function definition and a public
method is that a method of associating a method with generics
and determining the method type is required.

The following table lists the available method types:

@table @code
@item imeth
This indicates an instance method with compile time argument checking
enabled.  Use of this option requires that all methods associated
with a specific generic function all have the exact same argument
signature.

@item cmeth
This indicates a class method with compile time argument checking
enabled.  Use of this option requires that all methods associated with a
specific generic function all have the exact same argument signature.

@item ivmeth
This indicates an instance method with compile time argument checking
turned off.  Use of this feature allows methods with totally arbitrary
argument signatures to be associated with the same generic function.

@item cvmeth
This indicates a class method with compile time argument checking
turned off.  Use of this feature allows methods with totally arbitrary
argument signatures to be associated with the same generic function.


@end table

The following example illustrates a minimum method declaration.  Note
that a generic name is specified instead of the method's name.  This is
done since all external access to a method is through its associated
generic, the actual method name is rarely important.  Dynace will
default a method's name to be the associated generic name prefixed with
the class name, a code indicating the method type and the generic name.
Each part is separated with an underscore.  So in the following example
the actual method name will be @code{MyClass_im_gMyGeneric}.  You can,
however, explicitly name a generic.  This will be shown in other
examples.

@example
@group
public  imeth   object  gMyGeneric(object self)
@{
        return self;
@}
@end group
@end example


Since @emph{all} methods take @code{object self} as their first argument,
Dynace defaults this information.  Therefore, the above example may also
be written as follows.  


@example
@group
public  imeth   object  gMyGeneric()
@{
        return self;
@}
@end group
@end example


If the method took an additional arguments, the first argument may,
optionally, be skipped.  For example:

@example
public  imeth   object  gSomeGeneric(object self, int age, char *name)
@end example

may also be written as:

@example
public  imeth   object  gSomeGeneric(int age, char *name)
@end example


In addition, all argument types default to @code{object}, therefore,
any arguments which are of type object may exclude their type
declaration.  For example:

@example
public  imeth   object  gSomeGeneric(object self, int age, object name)
@end example

may also be written as:

@example
public  imeth   object  gSomeGeneric(int age, name)
@end example


Also, since @code{public} and the return type of @code{object} are
Dynace's defaults the first example may also be written as follows.
Explicit use of the @code{public} declaration is provided in order to
be consistent with the @code{private} declaration and not especially
recommended.

@example
@group
imeth   gMyGeneric()
@{
        return self;
@}
@end group
@end example




It is often convenient to associate one method with several generics.
This can be done as follows.

@example
@group
imeth   gMyGeneric, gGeneric2, gGeneric3 ()
@{
        return self;
@}
@end group
@end example


Explicit method naming is sometimes needed if it is to be directly
evoked within the same source file.  This can be done as follows.


@example
@group
imeth   gMyGeneric : SomeMethod ()
@{
        return self;
@}
@end group
@end example



Explicit method naming can also be used with multiple generic
association as follows.

@example
@group
imeth   gMyGeneric, gAnotherGeneric : SomeMethod ()
@{
        return self;
@}
@end group
@end example





The following example is a duplicate of the previous example, except
that it returns an @code{char *}.

@example
@group
imeth   char * gMyGeneric, gAnotherGeneric : SomeMethod ()
@{
        return "String";
@}
@end group
@end example


The following example illustrates the syntax used to associate a single
method with generics which have argument checking turned on (imeth's)
and generics which have them turned off (ivmeth's).  This syntax is
only recognized by the @code{dpp} program when its @code{-X} option is enabled,
otherwise this special syntax is just ignored.


@example
@group
imeth   gMyGeneric, <vAnotherGeneric> ()
@{
        return "String";
@}
@end group
@end example


To use this feature the method must be declared as an @code{imeth} or
@code{cmeth}.  You then surround the generic name which is to be
treated as an @code{ivmeth} or @code{cvmeth} in angled brackets as
shown.  All the previously described syntactical elements may be
used without restriction when this option is used.  And again,
generic names in angled brackets will be ignored by the system unless
the @code{-X} @code{dpp} option is used.












@subsubsection Private Method Definition
Private methods are methods who are not associated with any generic
and are therefore only available for use within the file they are
declared it - similar to static functions.  However, like public
methods, private methods always take @code{object self} as their
first argument.

Private methods are mainly used as a way of creating method like
functionality in cases where they are only used internally and no
dynamic binding is needed.  They are more efficient.

The following declares a typical private method.  Unlike public
methods, private method declarations name the method and have
no generics associated with them.  Once declared, private methods may
be used like regular functions.


@example
@group
private imeth   object  MyMethod(object self)
@{
        return self;
@}
@end group
@end example


Like public methods, private methods default their return type and
argument list.  Therefore, the above declaration may also be written as
follows.

@example
@group
private imeth   MyMethod()
@{
        return self;
@}
@end group
@end example


@subsubsection Super Messages
Super messages are a way of evoking methods associated with a superclass
of an object, as opposed to the method which is directly associated
with the class.  Typically, this is used to allow a method to wrap
functionality around functionality already provided by a superclass.
It is often used in the creation of new instances.

Typically, if you wanted to evoke the functionality associated with an
instance, you would use the following syntax.

@example
        gSomeGeneric(someInstance, someArg);
@end example

However, if you wanted to instead evoke the method associated with the
superclass if your instance, you would use the following syntax.

@example
        gSomeGeneric(super  someInstance, someArg);
@end example

As a more complete example, lets say you have a class containing an
instance variable which you wish to be initialized to some value
whenever a new instance is created.  The following illustrates the
point.

@example
@group
defclass  MyClass  @{
        iSomeObject;
@};

private imeth init()
@{
        iSomeObject = gNew(String);
        return self;
@}

cmeth  gNew()
@{
        object  obj = gNew(super self);  // line 1
        return init(obj);
@}
@end group
@end example

In this example the instance variable is being initialized to a null
string object for every new instance created.  Remember that since
this is a class method, @code{self} will be a class object.  Since
only one of the kernel classes knows how to actually create a new
(blank) object, the @code{gNew} method must evoke the one above it
in order to finally get to the one which can actually create the
bare new object.  Once the new object is obtained, it may be initialized
and returned.

Note that you couldn't have just called @code{gNew(self)}
because that would call the same method you're in to be called
recursively and indefinitely.

In addition, since sending the @code{super} message to @code{self}
is the most common use of @code{super} the @code{self} is
actually optional.  Therefore, line 1 above could be written as follows:

@example
        object  obj = gNew(super);  // line 1
@end example


@subsubsection Comments
In addition to normal C comments the @code{//} is used to signal comments
to the end of the line.


@subsection Example Code
The following example illustrates a complete program which defines a new
Dynace class.  Each line is explained in the following sub-sections.

@example
@group
defclass  MyNewClass  @{
        char    *iName;
        int     iAge;
        iSomeOtherObject;
 class:
        long    cTotalInstances;
        int     cAverageAge;
@};

private imeth init(char *name, int age)
@{
        iName = (char *) malloc(strlen(name)+1);
        strcpy(iName, name);
        iAge = age;
        return self;
@}

cmeth  gNewInstance(char *name, int age)
@{
        return init(gNew(super), name, age);
@}

imeth   object    gDispose()
@{
        free(iName);
        return gDispose(super);
@}

imeth   object   gDeepDispose()
@{
        free(iName);
        if (iSomeOtherObject)
                gDeepDispose(iSomeOtherObject);
        return gDispose(super);
@}

imeth   gSetObject(obj)
@{
        iSomeOtherObject = obj;
        return self;
@}

imeth   gPrint(FILE *fp)
@{
        fprintf(fp, "%s is %d years old\n", iName, iAge);
        return self;
@}

@end group
@end example


@subsection Class Definition
The first step in defining a new class would be to create a @code{.d}
file with a @code{defclass} specification.  This is discussed in
detail in the section which describes the @code{defclasss} syntax.

Note the naming convention used.  Class names begin with an upper case
letter and are followed with upper and lower case letters.  Instance
variables start with an ``@code{i}'' and are followed with an upper case
letter.  Likewise, class variables start with a ``@code{c}'' and are
followed with an upper case letter.

This naming convention is just that -- a convention.  Dynace does
not enforce or necessarily encourage this convention.  However,
since Dynace allows direct access to instance and class variables
from within methods, some naming convention is desirable, otherwise
it would be difficult to distinguish instance variables from locally
declared variables.

@subsection Defining Methods
Once the @code{defclass} specification is given, the methods may be
declared.  The syntax for this, as well as several examples are discussed
in the section which described method syntax.

It is important to understand that @code{self} will always refer to the
instance object which called the generic, if it is an instance method.
And @code{self} will always refer to the class object if this is a
class method.

The naming convention used is that generics with fixed number and
type of arguments in which compile time argument checking is enabled
start with ``@code{g}'' and are followed be an upper case letter.

Generics without compile time argument checking enabled may be passed
arbitrary arguments.  These generics start with ``@code{v}'' and are
followed be an upper case letter.

This naming convention is not enforced or necessarily encourage by
Dynace.  However, it is intended to help a programmer distinguish normal
C functions from various points where Dynamic binding is occurring.


@subsubsection Accessing Instance And Class Variables
An object's instance and class variables may only be accessed from
within a routine located in the source file which defines the class.
There are several ways available to access these variables.  This
subsection will discuss all the ways from the most frequently used
methods to the least.

The most frequent use of instance variables is from within public or
private instance (not class) methods.  These methods may directly access
the variables as they would access any local variable, by just using
them by name.

Class variables are also directly accessible by name in @emph{all}
methods or regular C functions within the module.

There are four places where instance variable access gets a little tricky.
Dynace, however, provides mechanisms to conveniently access instance
variables in all circumstances.

The first difficulty is when attempting to access a newly created instance's
instance variables from within a class method which had just created the
new object.  This is a common problem within @code{New} generics.  There
are two solutions to this problem.  One solution is clean but a little
less efficient than a more dirty solution.  Understanding of the dirty
solution, however, is helpful when debugging code (regardless of which
solution is used).

The clean solution is depicted in the following example.

@example
@group

private imeth  init(char *name, int age)
@{
        iName = (char *) malloc(strlen(name)+1);
        strcpy(iName, name);
        iAge = age;
        return self;
@}

cmeth   gNewInstance(char *name, int age)
@{
        return init(gNew(super), name, age);
@}
@end group
@end example

In the above example @code{gNewInstance} first calls @code{gNew} to
create a new instance.  It then passes the new instance along with the
initialization arguments to an instance method (@code{imeth}) because
instance methods can access the new instance's instance variables.
Since the @code{init} method returns the object passed, @code{gNewInstance}
can simply return it.

One interesting point about the above example is that private methods,
unlike public ones, must either be defined or declared prior to use.

With the dirty solution it is important to understand that Dynace
provides all instance variable access through a structure pointed to
by a locally declared pointer named @code{iv} (which stands for
``instance variables'').  In the case of instance methods, this
procedure is automatically handled by the Dynace preprocessor.
However, it must be done explicitly, when needed, in class methods.

Dynace also declares a typedef which is a structure mirroring the
instance variable definition contained with the @code{defclass}
statement.  Finally, Dynace provides macros for obtaining
a pointer to the instance variables associated with a given object.

The following example depicts the solution to the above mentioned problem.

@example
@group
cmeth   gNewInstance(char *name, int age)
@{
        object  obj;                    /*  line 1  */
        ivType  *iv;                    /*  line 2  */

        obj = gNew(super);              /*  line 3  */
        iv = ivPtr(obj);                /*  line 4  */
        iName = (char *) malloc(strlen(name)+1);
        strcpy(iName, name);
        iAge = age;
        return obj;
@}
@end group
@end example


This example demonstrates all the above procedures.  Line 1 declares a
variable which will hold the new instance object to be created.  Line 2
declares the @code{iv} variable which Dynace needs in order to access
the object's instance variables.

Line 3 calls the super @code{gNew} in order to create a new and
uninitialized instance object and assign it to @code{obj}.  Line 4
gives Dynace access to the instance variables associated with the
@code{obj} variable.

After line 4, the code may directly access the instance variables
as would be done in an instance method.

The second problem associated with accessing instance variables has
to do with normal C functions contained within the same module.
Since C functions don't necessarily take the self argument, it is
impossible for Dynace to know how to access the instance variables.
However, a simple method of achieving this is provided.

As stated above, Dynace uses the @code{iv} variable in order to
access instance variables, therefore, all that has to be done is
to pass the @code{iv} pointer to the C function.  For example:

@example
@group
static  void    MyFunc(ivType *iv, int age)     /*  Line 1  */
@{
        iAge = age;                             /*  Line 2  */
@}

imeth   gChange(int age, char *name)
@{
        MyFunc(iv, age);                        /*  Line 3  */
        iName = name;
        return self;
@}
@end group
@end example

In the above example, a normal C function is given access to the
instance variables by passing it the noted @code{iv} pointer.
Since Dynace automatically gives instance methods access to an
@code{iv} pointer, it may be simply passed to the function
(as shown in line 3).

Line 1 shows how the @code{iv} variable would be declared.  It doesn't
necessarily have to be the first argument.  Line 2 shows how the
instance variables are now directly accessible within the function.

The third problem associated with accessing instance variables has to do
with accessing the instance variables of more than one object within the
same module at the same time.  The problem can be resolved (in a relatively
inefficient, but clean manner) with instance variable accessing methods.
However, Dynace provide a very efficient way to accommodate this situation.

Since Dynace gives a method of accessing the instance variable
structure pointer associated with an object (@code{ivPtr} macro), these
pointers may then be used as normal C structures.  For example:

@example
@group
imeth   gCopy()
@{
        object  obj2;                   /*  Line 1  */
        ivType  iv2;                    /*  Line 2  */

        obj2 = gNew(super MyClass);     /*  Line 3  */
        iv2  = ivPtr(obj2);             /*  Line 4  */
        iv2->iAge = iAge;               /*  Line 5  */
        return obj2;                    /*  Line 6  */
@}
@end group
@end example

The above example defines an instance method which produces a new
instance of the class and initializes it to have the same age as
the instance object passed.  Line 1 declares the variable which will hold
the new object being created.

Line 2 declares a pointer which will point to the instance variables contained
within @code{obj2}.  Note that you couldn't use the name @code{iv} because
that would conflict with Dynace's use of that variable relative to the
implied @code{self} argument.

Line 3 creates a new blank instance of the class.  Line 4 gives access
to the instance variables contained within the new object.

Line 5 is the neat part.  The left side of the assignment references
an instance variable associated with @code{obj2} and the right side
references an instance variable associates with the object passed
(self).  This line could have also been written as follows:

@example
        iv2->iAge = iv->iAge;
@end example

since Dynace automatically creates the @code{iv} structure pointer and
associates it with @code{self}.  The key point here is that
all direct instance variable usage is always relative to the @code{iv}
variable.

The final difficulty is associated with instance variable access has to do
with the C preprocessor.  The following example will cause problems.

@example
@group
#define ASSIGN  iAge = age

imeth   gChangeAge(int age)
@{
        ASSIGN;
        return self;
@}
@end group
@end example

The above example will cause problems because Dynace will scan the
method and determine that since it can't find any instance variable
names referenced in the method, there is no need to gain access to them.
Therefore, Dynace will not generate the @code{iv} variable for this
method and you will receive a compile error.

This problem may be corrected by explicitly requesting Dynace to give
access to the instance variables by use of the @code{accessIVs} macro.
The following example corrects the previous example as discussed.

@example
@group
#define ASSIGN  iAge = age

imeth   gChangeAge(int age)
@{
        accessIVs;
        ASSIGN;
        return self;
@}
@end group
@end example

This macro must be placed in the variable declaration portion of the
instance method's definition.  The @code{accessIVs} macro actually declares
and initializes a variable called @code{iv} from the @code{self} argument
to the method.  This @code{iv} variable will point to a normal C language
structure which will look like the structure you defined when defining
the instance variables.



@subsubsection Determining the Class of an Object
There is a macro in Dynace which will return the class of any object.  This
macro is called @code{ClassOf}.  The following example illustrates its usage:
@example
@group
        object  o1, o2;

        o1 = ClassOf(o2);
@end group
@end example
Now @code{o1} will contain the class of @code{o2}.

@subsubsection Sending Super Messages
The process of evoking a method which is located in a superclass is
sometimes necessary when executing a method within a particular class.
Normally, when a generic function is evoked, the search for an applicable
method starts at the class of the object which is the first argument to
the generic.  If no method is found, the superclasses are then
searched.  When sending a super message, the superclasses of the object
are searched first.  If no method is found then the superclasses of the
superclasses are searched, and so on.  The class of the object is bypassed.

An example will clarify this concept.  Lets say you have a class defined
as follows:

@example
@group
defclass  MyClass  @{
        int     iA;
@};

imeth   gPrint()
@{
        printf("iA = %d\n", iA);
        return self;
@}
@end group
@end example

This is a simple class which just holds the integer @code{iA}, and has
a method to print it.  Now let's define a second class which inherits
from the above class.  The second class will look as follows:

@example
@group
defclass  AnotherClass : MyClass  @{
        int     iB;
@};

imeth   gPrint()
@{
        gPrint(super self);
        printf("iB = %d\n", iB);
        return self;
@}
@end group
@end example

Not that both class's print methods share the same generic function,
@code{gPrint}.

Since the second class inherits from the first class, instances of the
second class will actually contain both the integer @code{iB}, and also
the integer @code{iA}, defined in the first class.  Since the second
class's print method has no direct access to the @code{iA} instance
variable, because of strict encapsulation, there is no way for it to
print the @code{iA} instance variable's value.  When the @code{self}
argument to the @code{gPrint} generic function is of the second class's
type, the second class's print method would always be evoked.  There needs
to be a way to evoke the print method associated with the first class from
the print method in the second class.

In the second class's print method, the super syntax shown would
accomplish the desired effect.  This syntax works like evoking the
@code{gPrint} method on an instance of the current class, except that it
starts its search with the superclasses.

The end result of all this is that when a @code{gPrint} generic function is
called with an instance of the second class, the second class's print method
will be evoked.  Then the second class's print method will evoke the first
class's print method, printing the instance variable @code{iA}.  When the
first class's print method is through, it returns to the second class's
print method, which called it.  After that, the second class's print
method prints the @code{iB} instance variable, and returns.

This super message concept is very important to understand.  It is one
of the key features which allows you to take a pre-existing class and
modify its behavior to suit your needs without having to re-write the
pre-existing code.

@subsection Class Initialization
During startup, when your application executes @code{InitDynace},
Dynace initializes the Dynace kernel, the generic objects and
the Dynace kernel classes.  The remaining classes get initialized
when they are used.  If you use a class which is a subclass of
other classes, Dynace automatically initializes the parent classes
first.

Explicit class initialization may be accomplished by simply referencing
the class.  The following example, which should exist inside a function,
explicitly initializes the @code{MyClass} class.

@example
        MyClass;
@end example

@subsubsection Class Initialization Function
Sometimes it is desirable to perform some additional procedures at the
time the class is initialized.  For this case it is possible to have
Dynace evoke class specific initialization functions declared by the
programmer.  Most often this feature is used to initialize class
variables.

A class specific initialization should be a static function which
take no arguments and returns no results.  This function will
get automatically evoked by Dynace immediately after the creation
of the class it's associated with.  Therefore, it will have full
access to all the class variables.

In order to inform Dynace of the existence of a class initialization
function, it should be declared in the @code{defclass} statement as
described in the @code{defclass} specification described in a previous
section.


@subsubsection An Example

The following code fragment depicts a class initialization function
being used.

@example
@group
static  void    class_init(void);

defclass  MyClass  @{
        int     iVar;
 class:
        object  cVar;
 init:  class_init;
@};

static  void    class_init()
@{
        cVar = gNew(Set);
@}
@end group
@end example


@subsection What Gets Linked
Regardless of what classes exist in a particular library or what
classes are mentioned in the @code{generics.h} or @code{generics.c}
files Dynace is designed in such a fashion so that at link time
only the classes which your application actually uses will get linked
in.


@subsection The @code{gNew} Class Method
Most classes will have an instance creation method.  This is the method
which is responsible for creating and initializing instances of the new
class, also called the @emph{constructor}.  If no local instance
variable initializing is necessary, then no constructor method need be
defined.  One of the superclasses, or the Dynace kernel will automatically
create and initialize the instance.  In the absence of any explicitly
initialized instance variables in a new instance, the Dynace kernel will
zero out all instance variables of a new object.

The @code{gNew} class method works like any other class method.  It
receives, as its first argument, the object representing the class.
If the constructor needs to take additional arguments you can either
use the @code{vNew} generic (which has no compile time argument checking
so the method can have any signature) or you can use a different generic name
(such as @code{gNewInstance}) which will have the desired argument
signature.

The first thing the @code{gNew} method normally does is create an
instance of the class.  This can be done by sending a super message to
@code{gNew}.  It must then pass the new object to an instance method
so that the instance variables can be accessed and initialized.

@example
@group
defclass  MyClass  @{
        char    *iName;
        int     iAge;
@};

private imeth  init(char *name, int age)
@{
        iName = (char *) malloc(strlen(name)+1);
        strcpy(iName, name);
        iAge = age;
        return self;
@}

cmeth  gNewInstance(char *name, int age)
@{
        return init(gNew(super), name, age);
@}
@end group
@end example

Notice how the above example was also able to allocate additional storage
for @code{iName}.


@subsection The @code{Dispose} Instance Method
Any class whose instances must do something more when being disposed of
than just freeing themselves, must define an instance disposal method.
This method, called the @emph{destructor}, is usually associated with
the @code{gDispose} and @code{gDeepDispose} generic functions.

In the example in the previous section one of the instance variables
(@code{iName}) contained a pointer to allocated storage.  If the instance
was freed without regard for this allocated storage, the system would
soon run out of heap space.  Even the garbage collector couldn't help.
So what we do is define a destructor which will free the instance
appropriately.  The following code will illustrate this procedure:
@example
@group
imeth   object    gDispose()
@{
        free(iName);
        return gDispose(super);
@}
@end group
@end example
There are several things which may be noted from this example.

First, the method returns what is returned from the @code{gDispose}
super call.   By convention the disposal generics return @code{NULL}.
This is often used to null out a variable which references an object
which is being disposed and hence no longer valid.

Second, the call to free, frees the storage allocated and assigned to
the @code{iName} instance variable.  It does not free the entire
instance.

Third, the call to the super method is used to request any
superclasses to perform their freeing.  Eventually, the super call will
terminate at the Dynace @code{Object} class which will do the actual
freeing of the entire instance.

As discussed before, there are two ways to dispose of an object.  The
first, is by explicitly evoking the @code{gDispose} generic function on
an object.  The second, and preferred, way is to allow the garbage
collector to dispose of the object when necessary.

If no special processes need to be performed to free an object, no
destructor need be defined.  Dynace has a default disposal procedure
which simply disposes of the individual object.


@subsection Creating Generic Files
The @code{generics.h} and @code{generics.c} files are created by the
@code{dpp} program.  @code{dpp} reads in the pre-defined classes and
generics from a pre-existing @code{generics.h} files as well as user
added classes and generic information from class definition (@code{.d}
files) files.  @code{dpp} is able to parse the necessary files and merge
the information in order to obtain sufficient information to generate
the correct generic files for the application.   See the section describing
@code{dpp} for a complete description of it.


@page
@section Dynace Pre-Processor (DPP)
The @code{dpp} program is the heart of the compile time functionality of
Dynace.  @code{Dpp} is capable of reading in pre-existing generics
declaration files (normally @code{generics.h}) and class
definition files (@code{*.d} files) and using this information to
perform five specific tasks.

First, @code{dpp} is able to assimilate all the information read
in and produce a merged generics declaration file (normally
@code{generics.h}) which can be included by the application.
This file includes all necessary declarations for the entire application,
including class and generic declarations and system includes.

Second, @code{dpp} is also able to use the information read in
to produce an application specific @code{generics.c} file.
This is the file which contains all the actual generics as well
as the generic and class initialization procedures.

Third, @code{dpp} is able to read in a class definition file
(@code{*.d} file) and produce a normal C file (@code{*.c})
for compilation by your compiler.

Fourth, @code{dpp} will validate the argument and return type
declarations between methods and pre-existing generics to make sure
there are no inconsistencies.  This will only be done for methods which
have argument checking turned on.  If an inconsistency is encountered,
@code{dpp} issues an error message, but does not stop its processing.
It will, however, optionally return an error code.

Fifth, it provides facilities so that the application specific
@code{generics.c} file only need be generated and compiled when
absolutely necessary.  It does this by comparing any pre-existing
@code{generics.h} file with one it's about to create.  If they are the
same (i.e. no new generics or classes have actually been added), then
instead of generating the file, the @code{generics.h}, @code{generics.c}
and @code{generics.obj} files are just touched.


@subsection Controlling DPP
@code{Dpp} operates by first reading in all input files (regardless
of their order on the command line) and then using this information
to check the input consistency and produce any requested output files.

@code{Dpp} is non-interactive, all operational requests and parameters
are given at the command line.  However, these command line arguments
may also be located in files which @code{dpp} can read as an extension
of the command line.  These files, called command files, may contain
any number of parameters in the same format as the normal command line.
Parameters may be spread across any number of lines because @code{dpp}
ignores new lines, tabs and multiple spaces located in one of these
command files.

Command files are identified by a ``@@'' followed by a file name with
no intervening spaces.  Command files and normal command line
parameters may be freely intermixed.  For example the following
is quite legal.

@example
        dpp -g @@cmdfile -h
@end example

Unlike the DOS standard, @code{dpp} only supports a hyphen or dash
as an option switch.  All options are case sensitive.

A complete list of options may obtained by evoking @code{dpp} without
any arguments.  It will display a complete list of all options and
exit.

@subsection DPP Status Messages
During @code{dpp} operation, it will display the name of each file
it is reading or writing, in addition to any warning or error messages.
At the end of its run, @code{dpp} will display a status message similar
to the following.

@example
@group
Classes  = 48, 3, 51
Generics = 193, 15, 208
@end group
@end example

The first line indicates that @code{dpp} read in 48 class definitions
via generic header files (i.e. @code{generics.h}).  It then encountered
3 new class definitions while processing source files
(i.e. @code{*.c, *.d}).  This totaled 51 classes.

The second line indicates the same thing for generics.


@subsection Input Options (-g & -G)
Input options are those which cause @code{dpp} to read a file.

@subsubsection Generic Declaration Input
It is important that @code{dpp} have complete access to all related
class and generic declarations when processing any files.  This is
the only way @code{dpp} can have sufficient information to adequately
verify system consistency and produce complete generic files.

Typically, the single @code{generics.h} file associated with the application
is all that is needed.  The main exception to this rule is when initially
building the application specific @code{generics.h} file or when wanting
to build a new, clean one.  In that case you should read in all the
@code{generics.h} files associated with all the subsystems you intend
to use.

As shipped, Dynace only includes two such files in the include directory.
One called @code{generics.h} includes all the classes and generics
defined by the Dynace kernel and class library.  A second, called
@code{wingens.h}, includes all that's in @code{generics.h} file plus
all the classes and generics associated with the Dynace Window
Development System.  You would use one or the other, not both.

The command line argument used for this purpose is @code{-g}.
Immediately following this option may be a list of zero or more,
space separated, generic file names to be read in.  There is always
a space after the @code{-g}.  If no file names follow the option,
the single @code{generics.h} file is assumed.

There is also a similar @code{-G} option which works the same except
that it won't return an error code if one of the files can't be
found.

@subsubsection Source File Input (-s & -p)
@code{Dpp} includes two options for reading in source files.  One
command is for reading only and the other is for reading and
pre-processing.

The @code{-s} option is for reading only.  It reads in the specified
files and uses this information for consistency checks and producing
complete generic files.  It is capable of reading in the new
class definition files (@code{*.d}) as well as the old Dynace 2.x
class definition files (@code{*.c}).

The @code{-p} option is for reading and pre-processing class
definition files (@code{*.d}), thus producing associated
C source files (@code{*.c}) with the same file name prefix.
The information read in is also used as in the @code{-s} option,
therefore, it would never be necessary to include the same
file for processing under the @code{-s} and @code{-p} options.

In addition to recognizing @code{*.d} files as class definition files,
@code{dpp} will likewise recognize @code{*.dd, *.n, *.nn, *.i} and
@code{*.ii} in upper or lower case.  Input file names ending in
@code{.dd, .nn} or @code{.ii} will produce corresponding @code{.cc} or
@code{.cpp} files depending on whether you are on Unix or DOS
respectively.

@subsection Output Options
Output options are those options which cause @code{dpp} to create
a file.

@subsubsection Generic Declaration File Creation (-h)
Once all the pre-existing generic declaration files (typically
@code{generics.h}) and class definition files have been read in,
@code{dpp} is then able to produce a merged class and generic
declaration file for inclusion by application code.

The @code{-h} option is used for this purpose.  The @code{-h} option
may, optionally, be followed by a space and file name.  If the file name
is given, @code{dpp} will use that file for the generic class definition
file.  Otherwise, the default of @code{generics.h} will be used.

@subsubsection Generic Source File Creation (-c)
Use of the @code{-c} option causes @code{dpp} to generate the
application specific C source file which contains all the
generics used by the application as well as generic and class
initialization code.  Typically, the application specific
generics declaration file (@code{generics.h}) contains
all the information needed for creation of this file.

The @code{-c} option may, optionally, be followed by a space and file
name.  If the file name is given, @code{dpp} will use that file for the
generic file.  Otherwise, the default of @code{generics.c} will be used.

If the @code{generics.c} file gets too large for your compiler environment
see the Splitting the Generics File (-M) @code{dpp} option.

@subsubsection Processing Class Definition Files (-p)
@code{Dpp} is capable, after reading in the appropriate generics
declaration file, of reading a class definition file, verifying
its consistency with the system and generating a corresponding
C source file for compilation.

The @code{-p} option is for reading and pre-processing class
definition files (@code{*.d}), thus producing associated
C source files (@code{*.c}) with the same file name prefix.
The information read in is also used as in the @code{-s} option,
therefore, it would never be necessary to include the same
file for processing under the @code{-s} and @code{-p} options.

Input files ending in @code{.dd} will produce corresponding
@code{.cpp} or @code{.cc} files depending on whether you are
on Unix or DOS.

@subsubsection Java Interface Files (-j)
Dynace has the ability to call Java code and have Java code
call Dynace generics.  The @code{-j} option to @code{dpp}
causes @code{dpp} to generate files used to interface with Java.
This option also takes an optional package name argument.
The files generated are named @code{object.java},
@code{DynaceClassRef.java}, and @code{LoadDynClasses.c}.
These Java interface files, in concert with the Dynace / Java interface DLL
named @code{JavaDynace.dll} makes the interface possible.

@subsubsection Scheme Interface Files (-L1 & -L2)
@code{Dpp} has the ability to generate interface code to allow Dynace
generic functions to be accessible from within a Scheme language
environment.  Once the @code{generics.h} file is read in (via the -g
option) these options cause @code{dpp} to generate the necessary
interface files for two supported Scheme environments, LibScheme and
MzScheme.  This interface file may be named after the option and
will default to @code{scminter.c}.  This file must be compiled and
linked with your application.

MzScheme is available at
@code{http://www.plt-scheme.org}

@subsection Misc. Options
@code{Dpp} supports a variety of other options documented in this
section.  Some options, however, are not documented.  This is because
some features are either experimental and not ready for use or only used
to support old versions of Dynace which are no longer supported in any
other fashion.

@subsubsection Preventing Unnecessary Compiles (-t)
Generic files (both .c and .h) files change whenever classes or generics
are added or deleted, or arguments to existing generics change.
Whenever a class definition file is changed it must be pre-processed
to produce a compilable C file.  Also, when class definition files
change, it may or may not be the case that new classes or generics
have been added or existing generics have changed argument types.
In fact, except for a new development project, adding classes and
generics is relatively rare when compared to other code changes.

Given this fact, @code{dpp} is able to determine whether a significant
change has occurred or not.  It does this by comparing the pre-existing
application specific generics declaration file (@code{generics.h})
with one it's about to create.  If they are the same, @code{dpp}
does not create the new generic file and instead remembers the fact
that it was not necessary.

If it wasn't necessary to generate the generic declaration file
and the @code{-t} option is given, @code{dpp} will touch the specified
files in order to prevent the make utility from unnecessarily building
and compiling the corresponding @code{generics.c} file.  This
can lead to a significant savings in time.


The @code{-t} option may, optionally, be followed by a space and space
delimited list of file names to touch.  If the file list is given,
@code{dpp} will touch the given files in the order given.  Otherwise, the
default of @code{generics.h generics.c generics.obj} will be used.
On unix systems @code{generics.o} is used instead.  If any of the
files don't already exist, @code{dpp} will just skip it.

@subsubsection Removing Classes or Generics (-r)
It is sometimes necessary to cause @code{dpp} to remove classes or
generics from an existing @code{generics.h} file.  This is necessary
when an old class or generic is no longer used, or when the return type
or arguments to a generic has been changed.  If the return type or
arguments to a generic has been changed and you don't want to rebuild
the @code{generics.h} file from scratch, it's often easier to remove the
old generic from the @code{generics.h} file and then just re-scan the
source file with the new generic definition to re-add the changed
generic.  Once the @code{generics.h} file has been corrected a new
@code{generics.c} file can be generated from the correct
@code{generics.h} file (with the @code{-g -c} options).

To use this option when generating generics files simply follow the
@code{-r} option with a list of classes and/or generics to remove (each
separated by a space).

@subsubsection Splitting the Generics File (-M)
As a system becomes increasingly complex the number of generic functions
increases.  @code{Dpp} typically generates a single @code{generics.c}
file which contains all the generic functions defined within the system.
The problem with this is that in some limited environments (DOS or 16
bit Windows) there is a limit to the size a single source (@code{.c})
file can be so @code{generics.c} gets too big to be compiled.  The
solution is to split @code{generics.c} into multiple smaller files. Note
that there isn't any problem associated with the size of the
@code{generics.h} file in these environments.

The @code{-M} option causes @code{dpp} to split the @code{generics.c}
file at the time it is created (using the @code{-c} option).  As part
of the @code{-M} option you must include the number of generic functions
to split at.  This option is not separated with a space.  For example,
using @code{-M900} will cause the first 900 generic functions to be
defined in @code{generics.c} and the remaining generic functions will
be placed in a file names @code{gens1.c}.  @code{Gens1.c} will have to
be compiled and linked into the application just as @code{generics.c}.

@subsubsection Including Additional Header Files (-Isc, -Iac, -Ish & -Iah)
It is sometimes necessary to cause additional header files to be
included in the @code{dpp} generated @code{generics.c} or
@code{generics.h} files.  This occurs when generic return or argument
types are defined in other include files.  Therefore, these files must
be included prior to declaring these generics.  @code{Dpp} has the
@code{-I} series of options in order to support this feature.

Any particular @code{-I} type option should be followed by a space and
space delimited list of file names to include.  @code{Dpp} will cause
the specified files to be included by the generated @code{generics}
file in the specified order.

Use @code{-Isc} for system include files to be added to the
@code{generics.c} file.  Use @code{-Iac} for application include files
to be added to the @code{generics.c} file.  Use @code{-Ish} for system
include files to be added to the @code{generics.h} file. And use
@code{-Iah} for application include files to be added to the
@code{generics.h} file.

@subsubsection Ignoring Errors (-i)
Normally, while processing the input files, if @code{dpp} encounters
some kind of an error, it displays the error message and continues
processing the input files.  However, if an error occurs, @code{dpp}
does not normally produce any output files.  The @code{-i} option
causes @code{dpp} to produce the output files requested anyway.

@subsubsection Quiet Operation (-q)
Normally, @code{dpp} displays continuous status messages indicating
what files it is reading and writing in order to give some idea
of where it is in its process.  The @code{-q} option may be
used to cause all unnecessary messages not to be displayed.
This option, however, will not inhibit error messages.

@subsubsection Return Code (-z)
Under normal operation @code{dpp} returns 0 if no error occurs and
non-zero otherwise.  This is typically used to signal the make
facility that an error occurred.  The @code{-z} option is used to cause
@code{dpp} to return 0 regardless of whether or not an error occurs.

@subsubsection Compile Time Argument Checking (-N)
The @code{-N} option is used to cause @code{dpp} to treat all generics
as if they were defined with compile time argument checking turned off
and generate code accordingly.  In this case all generics will take
arbitrary arguments and depend entirely on runtime argument validation.

This was the way Dynace versions 1 and 2 worked.  The capability is kept
for two reasons.  First for backward compatibility.  And second, to
support environment which require complete flexibility with respect to
generic arguments.

@subsubsection Disabling #line directives (-nld)
Normally @code{dpp} generates line directives when preprocessing
a class definition file into a C source file.  This is done in order
to allow debugging on the original class definition file instead
of the generated C file.

If this feature is not wanted use the @code{-nld} option.

@subsubsection Extra #line directives (-eld)
Normally @code{dpp} generates line directives only when necessary
(when the files get out of line sync).  This option causes @code{dpp}
to (in addition to it's normal #line directive output) output a
#line directive prior to each method.

@subsubsection Preprocessing Strategies (-S)
@code{Dpp} is capable of supporting four different code generation
strategies.  The various strategies refer to different linkage
conventions between generics and methods.  The various alternatives
offer choices which determine the generated code's portability,
efficiency, and use of C++ specific features.

Class definition files will work the same regardless of which strategy
is selected.  However, it is absolutely critical that the entire system
be compiled for one and the same strategy.  This includes generic files,
class definition files and the Dynace system and class libraries as well.

To use this option you would use @code{-Sn} where @code{n} is a number
between 1 and 4.  There is no space between the @code{-S} and the number.
The various strategies are as follows:

Strategy 1 incorporates a small, system and compiler specific, piece of
assembler code.  This strategy is the most efficient and the most
recommended.  This assembler piece is supplied for many of the popular
platforms.  Writing the assembler piece for a new platform is very easy
for an assembler expert and documented in the next section of this manual.
The only down side of this strategy is lack of portability (if the
assembler piece is unavailable).

Strategy 2 uses a total C based approach.  This strategy maximizes
portability and for that reason it is the default strategy and the one
used to generate the Dynace system as shipped.  The down side to this
strategy is that it is less efficient in terms of execution speed and
stack space use than strategy 1.

Strategy 3 attempts to reduce the runtime overhead associated with
strategy 2 by the use of C++'s inline function facility.  The resulting
code will run faster but will incur a heavy penalty in code size due
to the inline expansions.  This strategy is not typically recommended.

Strategy 4 produces code similar to strategy 3 except that generics
which have variable arguments are treated like strategy 2 instead.
This was done in order to accommodate compilers such as GCC which
cannot inline variable argument functions.

Remember, Dynace is shipped using strategy 2.  If you wish to use a
different strategy you must rebuild the entire system with the selected
strategy.

@subsubsection Generic Overloading (-X)
Dynace allows many methods to share the same generic function.  This
option has no effect on that facility of Dynace.

Dynace also has the ability to simultaneously associate a single method
with a fixed argument generic which has compile time argument checking
and a variable argument generic which has no compile time argument
checking using the '<>' method declaration syntax.  Under normal
operation @code{dpp} ignores the '<>' syntax and does not perform
this association.  The @code{-X} option is used to enable @code{dpp}'s
support for this feature.

This facility is used to enable Dynace to support users who like
maximum compile time argument checking and those who prefer maximum
flexibility with regard to generic overloading with the same code
base.

@subsubsection Force Generics.h File Generation (-f)
Under normal operation, when generating a @code{generics.h} file
@code{dpp} will first read in any existing @code{generics.h} file
and not generate a new file if the existing one is correct.  The
@code{-f} option is used to force a new @code{generics.h} file
regardless of its status.

This feature is useful when changing strategies.  If none of the generics
or classes have changed Dynace would not normally generate a new generics
file.  This option will force a new one, with the correct strategy, to
be created.

@subsubsection Generating External Structures (-iv & -cv)
These options cause Dynace to generate external declarations for
a class's instance and class variables respectively when preprocessing
a class definition file.  The file names used will be the same as
the source file except that the file will end in @code{.iv} or
@code{.cv} respectively.

@subsubsection Class Initialization (-ni)
Under normal operation @code{dpp} generates the necessary class
initialization file when preprocessing a class definition file.
This option is used to cause @code{dpp} not to generate this
function.  The programmer will then be responsible for this
function.

@subsubsection Auto Include Generation (-nai)
Under normal operation @code{dpp} automatically adds code to include the
@code{generics.h} file in generated @code{generics.c} and C source
files.  This option causes @code{dpp} not to perform this auto inclusion.

This option along with the @code{-mg} option are used to support
users who prefer a single header per class.

@subsubsection Macro Guard Typedefs (-mg)
This option causes @code{dpp} to macro guard all typedefs and inline
generics in the @code{generics.h} file.  This is used in conjunction
with the @code{-nai} option to support one header per class.  This is an
unrecommended configuration.


@page
@section Dynace Customization & Special Techniques
The Dynace kernel includes a class called @code{Dynace}.  This class is used
to query and set various parameters which effect the global operation
of Dynace.  The following subsections detail the various parameters which
may be effected in Dynace as well as other special issues.

@subsection Makefile / Generics.h File Dependencies
All source files are technically dependent upon the application specific
@code{generics.h} file.  However, it is important that the make utility
not be made aware of this fact.  This is important because one of the main
goals and advantages of Dynace is the fact that rarely do changes to a class
require other modules to have to be recompiled - even when the changes
cause changes to the @code{generics.h} file.  If the makefile were
made aware of this dependency, there would be many, many totally
unnecessary compiles performed.

Most changes to a class definition file, including changes in the
instances or class variables associated with the class, changes in the
class hierarchy or algorithmic changes, have no effect on the
@code{generics.h} file.  The only things which could potentially cause a
change to the @code{generics.h} file are the addition or removal of an
entire class, the addition or removal of a generic, or the changing of
the argument list or return type associated with a generic.

With a little programmer cooperation and the disassociation (for make
utility purposes) of the @code{generics.h} file from other source
modules the need for re-compiles can be very drastically reduced.  The
make utility DMAKE and the example programs (the ones which create
classes) were designed with these concepts in mind.  Emulation of
these approaches is highly advised.

The following subsections detail all the things which could change
a @code{generics.h} file and what effect it has on other modules.
Advice on how to handle each situation will be discussed.

@subsubsection Adding New Classes
The addition of a new class never requires automatic recompilation
of any other modules.  Since the class was just added, the only modules
which could even use the new class must have also been edited in
order to reference the new class object.  In that case, those modules
will already be automatically recompiled anyway.  Other modules
don't directly reference the new class object and don't require
recompilation.

Remember that because of Dynace's design, any previously compiled code
which doesn't explicitly access a class by name, may still process
and use arbitrary objects without having knowledge of those objects
prior to being compiled.

@subsubsection Removal Of Classes
Again, the removal of a class should never cause the automatic
recompilation of all modules.  For starters, any module which
references that class object will have to be edited, causing it to
be recompiled.  You can't possible forget to edit a file because
the system won't link until all references to the removed class
are eliminated.

The best approach is to use @code{grep} or some other file search utility
to find all references to the removed class and make the appropriate
changes.

In addition to the above, it should be noted that @code{dpp} never
removes a class declaration from a @code{generics.h} file.  Of
course it really doesn't matter if an unused declaration exists,
however, periodic rebuilding of the @code{generics.h} file
from scratch will remove the accumulated junk.

@subsubsection Adding New Generics
Adding new generics never causes the need for mass recompilation for the
exact same reasons as the addition of new classes.  The only modules
which could possible use it have to be edited to put the new generic in
and that will cause only the appropriate modules to be recompiled.
No other module is effected in any way.

@subsubsection Generic Removal
The removal of generics is a little more complex.  First of all,
if you remove a generic from a class definition file, Dynace
doesn't automatically remove it from the @code{generics.h} file.
In fact, the removal of a generic from a class definition file
has no effect at all on the @code{generics.h} file.  This is because
without scanning all the class definition files and building the
@code{generics.h} file from scratch, Dynace has no way of knowing
whether or not any other module makes use of that generic.
Therefore, Dynace leaves it in.

Because of the above, there is no advantage to associating the
@code{generics.h} file to other code.  The bottom line is that
if a generic is removed, you'll have to search the code for
use of that generic and make appropriate changes.  This will
cause the appropriate modules to be recompiled.

Another interesting thing about these facts is that rebuilding
@code{generics.h} from scratch on a periodic basis is a good idea in
order to get rid of any unnecessary generics.

@subsubsection Generic Argument Or Return Type Changes
Of all the possible changes to a @code{generics.h} file this is the
most troublesome.  It's also the change which is performed
the least.  Once an application or class library reaches any level
of maturity, the existing protocols tend to change little.  New
features are implemented with new generics.

It's also difficult to change generic arguments because of the fact that
the person making the change must take into account the effect it will
have on any other classes which use the same generic.  Generics with
argument checking turned on must agree with all methods which they are
associated with.  Generics with argument checking turned off are not
effected with this issue.

Having said that, however, there are only two ways of dealing with
this situation.  First of all, if working on a team project, it should
be agreed that no generics will be changed without mutual agreement.
It's more common to just create a new generic.

The best way to deal with the problem is to search through all the
code for references to the generic and make the appropriate changes.
This will cause only the effected files to be recompiled.

Another, and poor, answer is to recompile everything and fix the modules
which get a compiler error.

This issue is not bad if you think about it.  What would you normally
have to do if you change the arguments to a regular C function?  Answer -
the same thing.

@subsection Compile Time Argument Checking
Dynace supports the simultaneous use of generics which have compile
time argument checking enabled and those that have the option
turned off.  By convention only, generics which have compile time
argument checking turned on begin with ``@code{g}'' and those that
have compile time argument checking turned off begin with ``@code{v}''. 

The programmer determines which generics have compile time argument checking
turned on or off at the time the generics are defined.  Those introduced
with @code{imeth} or @code{cmeth} have argument checking turned on and
those introduced with @code{ivmeth} or @code{cvmeth} have argument checking
turned off.

For those generics which have compile time argument checking enabled,
the Dynace preprocessor (@code{dpp}) will validate the argument lists
between generics and their associated methods during preprocessing time.
All discrepancies between generics and their associated methods will be
reported at this time.  Code which uses generics which have compile time
argument checking enabled will be validated by your normal C compiler by
use of the @code{dpp} generated declarations which exist in the
@code{generics.h} file.

For generics with compile time argument checking turned off none of
these checks will be performed at compile time.  The runtime system
will be used to perform the necessary checks.

@subsubsection Compile Time vs. Runtime Argument Checking
Before jumping to the conclusion that it's always best to have compile
time argument checking turned on a few things should be mentioned.
First of all Dynace has ample facilities to perform argument validation
at runtime.  Second, runtime-only argument checking is much more accurate
and safe than compile-only time argument checking.  Third, enabling
compile time argument checking on a generic causes inconvenient
restrictions placed on that generic as follows.

All methods associated with a generic which has compile time argument
checking turned on must have the exact same argument signature.  On the
other hand, a generic with compile time argument checking turned off
may be associated with methods with arbitrary argument signatures.
This adds a great deal of flexibility and convenience to the system
at no loss of safety except that you'll not discover a problem
until runtime instead of compile time.


The problem with compile time argument checking shows itself most with
the @code{gNew} generic.  Instead of being able to use @code{gNew}
for creation of instances of all classes, we have to use things
like @code{gNewWithInt} and @code{gNewWithStr} to differentiate
the various argument signatures.  Not only is this ugly but it also
causes the programmer to constantly ask ``now, which @code{gNew} do
I use for this class?''.

The next sub-section details an alternative approach.

@subsubsection Sharing Methods
In order to accommodate programmers with different demands Dynace has the
ability to simultaneous support programmers who prefer compile time
argument checking turned on all the time and those who prefer more
flexibility and better esthetics at the cost of not detecting argument 
errors until runtime.  Dynace has the ability to associate a single
method with both compile time argument checking generics and those
without compile time argument checking within a single code base.  The
programmer may enable this feature when building the system via the
@code{-X} @code{dpp} option (see the appropriate section).

As shipped, Dynace includes several generics which are used to provide
non-compile time argument checking versions of conveniently overloaded
generic functions including @code{vNew}, @code{vFind}, @code{vRemove},
@code{vFormat} and @code{vAdd}.  The entire system must be compiled with
the @code{dpp} @code{-X} option in order to enable them however.

@subsection Runtime Argument Validation
Whenever a generic function is evoked in Dynace, the system checks the
validity of the first argument.  It makes sure that the first argument
is in fact a valid Dynace object.  If it is not, Dynace will report that fact
and abort the program.  Otherwise, the continuation of the program could reek
havoc with the entire system.

This @emph{object checking} comes at a runtime cost, however.  This cost
is incurred at every evocation of any generic function.  While the cost
is relatively minimal, programs which are well debugged may turn the
checking off to maximize performance.  The following code fragment
illustrates your control over object checking:
@example
@group
gObjectChecking(Dynace, 0);  /* This line turns off object checking */
gObjectChecking(Dynace, 1);  /* This line turns on  object checking */
@end group
@end example

The default is for object checking to be turned on.


@subsubsection Additional Argument Validation
If runtime argument checking is turned on (the default) Dynace
automatically checks the validity and type of the first argument to all
generics.  It then dispatches to the appropriate method.  Since
different methods associated with a particular generic may contain a
different number and types of arguments, it is the responsibility of the
individual methods to validate their remaining arguments.

Dynace provides several mechanisms to accomplish and simplify this task.
The most convenient mechanism is the use of several macros called
@code{ChkArg}, @code{ChkArgTyp}, @code{ChkArgNul} and
@code{ChkArgTypNul}.  These macros provide a very simple way of
validating an argument as being a valid Dynace object and, optionally,
of being of a particular type.  If the argument is invalid an
appropriate and meaningful error message will appear.  These macros are
also used internally by Dynace where appropriate and may be controlled
by the object checking mode described in the previous section.

Other mechanisms which may be used are the @code{IsObj} function, the
@code{gIsKindOf} generic, the @code{ClassOf} macro, the
@code{gInvalidObject} generic, the @code{gInvalidType} generic and the
@code{gAbort} generic, as well as others.

The following code fragment illustrates the use of the @code{ChkArgTyp}
macro:
@example
@group
/*  Validate that arg3 (the 3rd argument) is an instance of String
    (or one of its sub-types)  */
ChkArgTyp(arg3, 3, String);

/*  Validate that arg4 (the 4th argument) is any Dynace object  */
ChkArg(arg4, 4);
@end group
@end example


Note that the @code{ChkArg} set of macros may only be used as the first
executable statements in a method.

@subsection Methods With A Variable Number Of Arguments
Due to the way Dynace deals with the various method / generic interface
strategies (described under @code{dpp} subsection ``Preprocessing
Strategies'') it was necessary to construct a standard way of dealing
with methods which implement variable arguments.  Use of these
conventions or standards assures the portability of method definitions
across all support strategies without loss of functionality or
convenience.

The conventions used to deal with variable argument lists are fully
described in the section on the @code{MAKE_REST} and @code{GetArg}
Dynace macros.

@subsection Tracing Facility
Dynace includes a powerful and flexible tracing facility to assist
in the debugging of applications.  The facility traces generic calls
of arbitrary combinations of classes, generics and methods.
Each traced generic prints out the generic name, the type of the first
argument to the generic, the name of the method found and the class
of the method found.

The tracing facility has three global modes controlled by
@code{Trace::Dynace}.  It can be turned off (the default)
to disable all generic tracing.  It can be turned on in order to
trace generic calls associated with classes, generics or methods
which have their tracing options activated (via @code{Trace::Behavior},
@code{Trace::GenericFunction} and @code{Trace::Method}) and none
deactivated.  The third mode is to turn tracing on all.  This traces
all generic calls which do not have their tracing specifically
deactivated or are not associated with classes or methods which
are specifically turned off.

Note that when referring to classes in the context of the tracing
facility, it refers to both the class of the first argument to
the generic and the class associated to the method found.

The tracing facility can be turned on and modified at any number of
arbitrary points of the application.

Trace output is sent to the stream @code{traceStream} which defaults
to @code{stdout}.

See @code{Trace::Dynace}, @code{Trace::Behavior},
@code{Trace::GenericFunction}, @code{Trace::Method},
@iftex 
@hfil @break 
@end iftex
@code{TracePrint::Dynace} and @code{traceStream} for more information.




@subsection Memory Management and Object Disposal
As stated earlier, all objects in Dynace are allocated from the heap.  In
fact the @code{object} declaration actually declares a variable to be a
pointer to this allocated memory.  Therefore, when an object is no
longer needed, it must be returned to the heap.  If this is not done you
would eventually run out of heap space.  Dynace has two methods of
accomplishing this task, disposal generics and the garbage collector.

Any class whose instances must do something more when
being disposed of than just freeing themselves must define an instance
disposal method.  The types of additional things this disposal method
would normally do is free any additionally allocated storage which is
local to the object being freed or perform any processing which needs
to be performed at object disposal time.

The disposal methods are associated with the @code{gDispose}
and @code{gDeepDispose} generic functions.

The @code{gDispose} generic is typically used to dispose of a single
object and any associated allocated space it may reference.  The system
default @code{gDispose} will handle getting rid of the single object so
the only time a specific method is needed is when the object
allocates its own memory from the heap or other processing needs to
occur at the disposal time.  A programmer defined @code{gDispose} method
would typically free any allocated space associated with the object and
then call its super @code{gDispose} method.

The @code{gDeepDispose} generic, in addition to performing the same
functions as the @code{gDispose} generic, is typically used to dispose
of whole group of related objects at the same time.  This method will
typically recursively call the @code{gDeepDispose} generic on all
objects being referenced within the object being disposed.  The system
default @code{gDeepDispose} method performs the same function as the
system default @code{gDispose} method.

@anchor{Speeding-Up-IsObj}
@subsection Speeding Up IsObj

@code{IsObj} is a function which validates a Dynace object pointer at
runtime.  This function is a small and fast function but gets called
quite a lot and often accounts for much of the overhead associated
with Dynace.  This overhead is significantly exasperated in the
Windows environment where the heap is non-contiguous.

Dynace's allocation scheme is designed to be memory efficient.
However, this scheme is at odds, under Windows, with the speed of the
system. Dynace has facilities to eliminate this overhead under
Windows.  If you build and run a Dynace based application, you can
call @code{gMAxMemUsed(Dynace)} after running through a representative
run of the application and before exiting to determine the amount of
Dynace memory the application used.  You can then use the
@code{Dynace_GetInitialPageSize()} function to cause Dynace to
allocate that amount as a single chunk of memory thus significantly
speeding up the application.  @code{Dynace_GetPageSize()} can also
be used to set reasonable values.  See the relevant documentation
for those functions in this manual.

Note that this issue and its corresponding solution are only relevant
in a Windows environment.  Unix / Linux don't suffer from these problems.

@subsection Garbage Collection
The way the garbage collector works is that it keeps track of your
maximum real memory requirements, after a garbage collection.  Whenever
your current memory usage exceeds a user defined number of bytes over
the maximum real memory requirements, a garbage collection is evoked.
Thus, the automatic aspects of the garbage collection process may be
controlled by specifying this user defined buffer area.  If the buffer
area is negative, no garbage collection process will take place.  You
will just have to explicitly dispose of any unneeded objects.  The
buffer area may be defined as follows:
@example
        gSetMemoryBufferArea(Dynace, 40000L);
@end example
Notice that the argument must be a long type.  If passed a negative long value,
garbage collection will be turned off.  The default setting is @code{-1L}.

The smaller the buffer area is, the more often the garbage collector will
have to run, and the slower the system will operate.  However, the memory
requirements of the system will be as small as possible.  The larger the
buffer area the less frequently the garbage collector will have to run, and
therefore, the faster the system will run.  However, if the buffer area is
too large, the system will experience delays when the very infrequently
called garbage collector has to collect a large volume of garbage.  The
best setting should be determined experimentally.  Just remember that
the memory requirements of the system will be equal to the real memory
requirements of the system plus the buffer area.  A starting value
of about @code{40000L} is recommended.

The garbage collector may be manually evoked with the following line:
@example
        gGC(Dynace);
@end example

Any class whose instances allocate memory from the heap must define an
instance method which is associated with the @code{gGCDispose} generic.
The garbage collector automatically calls this method when it frees an
object.  It must free the allocated space and call the @code{gGCDispose}
super method just like a @code{gDispose} method.  The only difference
between the @code{gDispose} and @code{gGCDispose} methods is that the
@code{gGCDispose} method should free allocated memory but must
@emph{never} dispose of any other Dynace object.  Although it may perform
any other necessary disposal type processing.  Typically the
@code{gDispose} method is adequate as a @code{gGCDispose} method and is
used as such.

The garbage collector in Dynace uses an algorithm which does not require a
lot of stack space even while collecting highly recursive data
structures.  The stack requirements are constant regardless of the
objects being collected.

Dynace offers an alternative garbage collection scheme which has important
tradeoffs.  See the sub-section on the Boehm Garbage Collector.



@subsection Global or Static Variables
The Dynace garbage collector only marks objects which are accessible, either
directly or indirectly, via automatic variables (those on the stack).
Therefore, if you create an object and assign it to a global or static
variable (a variable not on the stack) only, the garbage collector will
not mark those objects.  If those objects are not marked they will be
freed.  This is probably not what you want.

Note that under Win32, Dynace has special code and can automatically
detect and mark unregistered globals or statics.  Therefore manual
registration of these memory locations is not needed under Win32.

Dynace has the ability to register a memory range with the garbage collector
so that any valid objects referenced in this range will be marked.  This
will prevent the freeing of these needed objects.  The following example
illustrates the registration of a single global variable, @code{GV}:
@example
        RegisterVariable(GV);
@end example
This will cause the garbage collector to mark all objects which @code{GV}
refers to, either directly or indirectly.  It is OK for @code{GV} to not
refer to a valid object.

Dynace also includes a method of marking a range of addresses.  The following
example illustrates its usage:
@example
        gRegisterMemory(Dynace, begRange, size);
@end example
The @code{RegisterVariable} macro actually calls @code{gRegisterMemory}.




@subsection Boehm Garbage Collector
In addition to the garbage collector (GC) which is part of Dynace,
Dynace supports another GC which is freely available and widely used
throughout the industry.  This GC is not owned or supported by
Algorithms Corporation or Blake McBride.  It is called the Boehm GC and is
copyrighted by Hans-J. Boehm, Alan J. Demers and Xerox Corporation.

The Dynace GC is only capable of collecting Dynace objects and requires
a bit of programmer effort when dealing with non-Dynace objects such as
application malloc'ed memory (which the Dynace GC will not collect) or
global variables.  The Boehm collector has been integrated with Dynace
such that it will collect Dynace objects as well as application
malloc'ed memory.  In addition, the Boehm GC requires somewhat less
programmer effort in order to work correctly.  The main advantage of the
Dynace GC is that it is more portable and comes with vanilla Dynace.
The Boehm GC will not work in 16 bit environments such as DOS or 16 bit
Windows.

In order to use the Boehm GC you must re-compile the entire Dynace
system with the macro @code{BOEHM_GC} defined.  In addition, the
application must also be compiled with that macro defined and linked
with the Boehm GC library.  Both collectors cannot be used at the same
time, one or the other.  When using the Boehm GC there is no need to
enable it, set the buffer area, or register memory regions.  It's on all
the time and functions automatically.  See the Boehm GC documentation.

If your Dynace package didn't come with the Boehm GC source code you
may obtain it via one of the following internet or e-mail sources:

@example
@group
http://www.hpl.hp.com/personal/Hans_Boehm/gc/
http://reality.sgi.com/employees/boehm_mti
ftp://parcftp.xerox.com/pub/gc
Hans_Boehm@@hp.com
boehm@@acm.org
@end group
@end example



@subsection Native Thread Support
Dynace has been designed to support a truly multi-threading
environment.  This is not the simulated threads supplied with Dynace
but the true threads provided by the underlying system.  Dynace
utilizes critical sections or semaphores to ensure safe operation
in a true multi-threading environment.

To enable this facility Dynace must be built from scratch with
@code{NATIVE_THREADS=1} set on the makefile line.  The same is
true for all applications which intend to use native threads.
@code{Dpp} generates class files with the structures defined
and used by Dynace already protected.

Additionally, the interface to the underlying system's critical
section or semaphore facility will have to be provided.  With
knowledge of this area it is simple to provide.  Dynace comes
shipped with native thread support for Win32 only at this time.
See @code{include/dynl.h} and the @code{*.c} files generated by
@code{dpp}.



@subsection Method Cache
As stated earlier, the Dynace system caches method dispatching.  This
caching speeds up the system tremendously.  The larger the cache the
better.  The default cache size is rather small, and may be user
controlled.  The following line may be used to set the cache size:
@example
        gResizeMethodCache(Dynace, 51, 101);
@end example
This example sets the cache size to 51 classes and 101 generics.  These
numbers do not have to reflect the actual number of classes and generics
you have.  They are just used to setup the cache.  Regardless of the
numbers used, the system can handle any number of classes and generic
functions.

If this function is used, it should be used soon after initialization
of the Dynace system.


@subsection Memory Compaction
Dynace includes a memory allocator which is capable of performing memory
compaction.  This is used by the @code{String} class (as well as
possibly other classes) to reduce or eliminate memory fragmentation.
Since the compactor (@code{MA_compact()}) may relocate memory, it is
never called automatically by Dynace.  After a call to @code{MA_compact}
all pointers previously returned by Dynace generics such as
@code{gStringValue} may be invalid.  All object pointers, however, will
always be valid.

The compactor would typically be called, manually, after a number of
operations which would typically fragment memory.


@subsection Avoiding Runtime Method Lookup Costs
There is a small runtime overhead associated with the runtime binding
mechanism of Dynace (or any system which has runtime binding).  This
overhead is kept to a minimum by the Dynace method cache and other techniques
employed by the Dynace system.  It is possible, however, to locally cache
a method so as to reduce the future runtime overhead to zero.  This
technique is normally only used in tight loops where there is no chance
of using the wrong method.  Overuse of this technique is strongly
discouraged.

See the macros @code{cmcPointer}, @code{cmiPointer}, @code{imcPointer} and
@code{imiPointer} for further details and examples.  There is also an
example program which demonstrates the technique.



@subsection Generic Functions As First Class C Objects
In C, variables are treated as first class objects.  That is they may
be defined, initialized, assigned to (value change) and localized in
any arbitrary block.  C functions, on the other hand, are sort of
second class (or static) objects.  They may be defined but they
can't be changed (at runtime) or localized in an arbitrary block.

In Dynace generic functions are implemented in such a way that they
are first class objects, just like variables.  That is they may
be assigned to, passed and localized just like variables.

See the relevant example for more details.


@subsection Static Binding
It is possible to perform static binding with the Dynace system.  Doing so,
however, is highly discouraged because it eliminates much of the
flexibility and encapsulation properties of Dynace.  The only reason
someone might want static binding is for efficiency reasons.  These
needs, however, may be met by local method caching described above.
(See the sub-section on Avoiding Runtime Method Lookup Costs.)

In order to perform static binding you declare your method as a normal
C function instead of using the method declaration syntax.  You must
also declare its first argument to be @code{object self}, and explicitly use
the @code{accessIVs} macro.

From that point on it will act as other methods and may then be called
directly, without going through the generic.


When calling a method directly it is the programmers responsibility to
validate the validity and type of the first argument.


@subsection Allocating Objects From The Stack
All objects in Dynace are normally allocated from the heap.  This
allows for the maximum flexibility in terms of their life time
and scope.  However, Dynace does have a limited capability to
allocate objects from the stack.

The two advantages of allocating objects from the stack are that
the allocation process, and hence the creation of the object, is very
fast.  The second advantage is that no explicit object disposal is
necessary. Stack based objects simply disappear when the code goes
out of scope in a fashion similar to automatic variables.  Thus,
there is zero overhead associated with the object's disposal.

The two disadvantages associated with stack based objects are that stack
based objects are limited to the stack size and there is no provision
for auto execution of an object destructor.  In addition, object
creation is a bit clumsy.

Dynace provides two ways to dispose of a heap based object.  First it
can be explicitly disposed via the @code{gDispose} or similar generic.
Second, it can be automatically disposed via the garbage collector.
Both mechanisms allow the programmer to specify additional processing
which must be performed when the object is disposed (such as freeing
memory or closing files).  Unfortunately, stack based objects do not have
this ability unless they are explicitly disposed and explicitly disposing
of objects mostly negates the whole value of having stack based objects
in the first place.

Explicit disposal of stack based objects does not free up any stack
space until the scope is exited.  Therefore, if five objects are created
and freed in sequence in the same stack frame the stack space required
will be that of all five objects.

Due to the above facts stack based objects are generally discouraged.
However, in cases where they are needed and no special disposal
is required they can fit the bill.

See documentation on the @code{StackAlloc} for further details.


@page
@section Notes On Compilers, Rebuilding And Porting Dynace

Dynace has been designed and tested to be as portable as possible.  At
the time of the writing of this manual Dynace has been compiled and
tested under MS--DOS, Windows 3.1, Windows 95, Windows NT, Interactive
UNIX, UHC UNIX, Linux and Sun SPARC and the Alpha processor.  Compilers
used were Microsoft Visual C++ 1.0 and above (16 and 32 bit), Borland
C++ 3.1 and above (16 and 32 bit), WATCOM C/C++^32 9.0 and above,
Symantec C++ version 7.21 (32 bit), GNU C 2.4 and above.  In addition
several prior versions of each of the above compilers were also tested.

When @code{dpp} strategy 1 is selected there is one small piece of
assembler (the jumpTo code) used by the Dynace system.  This code is
specific to each compiler/model/OS combination.  All jumpTo assembler
routines which have been developed have been included with the system.
There is no assembler required for the default strategy of 2.

The Dynace threader uses the C library functions setjmp/longjmp.  When using
these functions it is important not to over-optimize a program (as most
compiler documentation attest).  The included makefiles where designed
to maximize the reliability of the Dynace threader.  If the threader is not
going to be used more aggressive optimization may be used.

The generics file must always be compiled with the optimization options
(usually maximum) defined in the makefiles.


@subsection DMAKE
The makefile utility used for all builds, except unix, is DMAKE.
DMAKE is a very powerful, portable and enhanced make utility developed
by Dennis Vadura.  It is freely distributable and not owned by
Algorithms Corporation or Blake McBride.  Complete documentation for this utility
is contained in @code{\DYNACE\DOCS\DMAKE.MAN}

@subsection MS--DOS

All 16 bit MS--DOS libraries were built using the large memory model.  There
is no reason why Dynace could not be run in other models other than the
size restrictions they present.  

Compilation for other memory model may require different jumpTo assembler
code which is not needed when using the default @code{dpp} strategy of 2.

@subsubsection Microsoft Visual C++

The makefiles used for this compiler are all called @code{M16.DM} or
@code{M32.DM} depending on whether it's for the 16 or 32 bit compiler.

Since the Dynace threader makes use of setjmp/longjmp and the Microsoft
Visual C++ documentation recommends not to use most of the optimization
features when setjmp is used, it is recommended to use only the
optimization options used in the included make files.  It has been
demonstrated that the @code{-Oe} option causes problems with the
threader.  If the threader is not going to be used much heavier
optimization may be used.

Maximum optimization with no stack checking must always be used when
compiling the generics file.

Jumto assembler for both the 16 and 32 bit compilers are provided for
this compiler so any @code{dpp} strategy may be used.

@subsubsection WATCOM Compilers

The makefiles used for this compiler are all called
@code{W32.DM}.

All testing was done with WATCOM C/C++^32 version 10.0.
Versions 9.0 and 9.5 were used in the past.

The flat memory model is used with argument passing on the stack.  The
target system is DOS4G.  

When switching between version 9.0 and 9.5 of the compiler two different
jumpTo routines must be used.  @code{jumpto.w} is for version 9.5 and
10.0, @code{jumpto.w90} is for version 9.0.

Maximum optimization with no stack checking must always be used when
compiling the generics file.

@subsubsection Borland C++

The makefiles used for this compiler are all called @code{B16.DM} or
@code{B32.DM} depending on whether the 16 or 32 bit compiler is used.

Since the Dynace threader makes use of setjmp/longjmp and the Borland C++
documentation recommends not to use most of the optimization features
when setjmp is used, it is recommended to use only the optimization
options used in the included make files.  If the threader is not going
to be used much heavier optimization may be used.

The Dynace threader will not run under the huge model with the Borland
compiler due to what is believed to be a problem with the Borland
huge model setjmp/longjmp functions.

Maximum optimization must always be used when compiling the generics
file.

Jumto assembler for both the 16 and 32 bit compilers are provided for
this compiler so any @code{dpp} strategy may be used.

@subsubsection Symantec C++

The makefiles used for this compiler are all called 
@code{S32.DM} for their 32 bit compiler.

Symantec compilers version 7.2 and prior are known not to work due to
compiler bugs.  Version 7.2 plus a patch from Symantec as well as 7.21
did work.

There is no jumpto assembler code for this compiler so @code{dpp}
strategies 2 or above must be used.  The 16 bit version of their
compiler was never tested.

@subsection UNIX

The makefiles associated with the UNIX build are all named makefile.unx.

Only the GNU C compiler has been used.

Because of the linear memory associated with UNIX (or other 32 bit
environments) Dynace runs quite well under it.

Since DOS puts CR/LF pairs at the end of each line and UNIX expects just
a LF there is a problem with several source files.  Since Dynace uses
multi-line macros UNIX expects lines which are continued to end in
backslash/LF combinations.  In DOS they end in backslash/CR/LF
combinations.  The extra CR causes a problem under UNIX.  In order to
fix the source files (get rid of the extra CR) Dynace includes a utility
(in source form) to correct the files.  The utility is called
@code{delcr.c} and is located under the @code{bin} directory.  It may be
compiled and applied to all @code{*.c, *.h, *.s, makefile.unx, readme}
files.  The files which causes the most trouble are
@code{dynl.h} under the @code{include} directory, @code{thread.c}
under the @code{threads} directory, @code{array2.h} under the
@code{class} directory and all the @code{makefile.unx} files.
@code{delcr} may be used by simply naming the files to be converted on
the command line.  Wild characters are ok.

In addition note that the top level makefile is set up to correct
all the files in the system automatically.

The @code{-O} option cannot be used with any GNU C compiler prior to
version 2.5.  This is because functions declared volatile (returning
type object) were assumed not to ever return (by design).  This
feature has been eliminated in GNU C 2.5.

Although we can not support it, Dynace has been ported to the Sun/SPARC
and Alpha chip environments by some of our users.  The jumpTo code is
contained in the file name @code{jumpto.s2}.  All necessary files are
included with the system.  See the makefiles for additional information.

@subsection Using A Debugger

Because of the way Dynace dispatches methods it is often difficult to
trace through a generic call at the C source level.  It has been found
that if a break point is placed at the beginning of the actual method
the debugger will stop there when going through the generic.  So, for
example, if you want to trace into a call to the @code{gNew} generic and
you know you are dealing with the @code{ShortInteger} class, you may put
a break point on the @code{ShortInteger_im_gNew} method in the
@code{shortint.c} file and continue.  The system will stop at the
@code{ShortInteger_im_gNew} method and you should be able to continue as
normal.  If you are not sure what class the object is just but a break
point on all the possible methods the generic will dispatch to.

There is not a problem when debugging a program at the machine instruction
level.  Therefore, it is possible to step at the C level up to a generic
call, switch to the machine instruction level, step through the generic
call until the method is reached, and then switch back to the C source
level.


See the section on the Dynace tracing facility for help debugging
a program.



@subsection Porting To Other Compilers, Memory Models or OSs
There are several issues associated with porting Dynace to a new
environment.  These issues are covered in the following subsections.

Dynace includes a few short programs to assist in the porting process.
They are located under @code{\DYNACE\KERNEL\PORT}.


@subsubsection JumpTo Assembler Code

The Dynace preprocessor (@code{dpp}) is capable of supporting various
method / generic interface strategies which are described on the section
documenting @code{dpp}.  While the default strategy (2) does not require
any assembler code, the more efficient strategy (1) does.  Although
Dynace comes with the necessary assembler code for many platforms, this
section outlines the steps necessary to create the assembler code for a
new environment.

Basically, the assembler must take a pointer to a function as an
argument, pop two stack frames and jump (not jsr) to the function
passed.

The best way to create this code is to compile jumpto.c (provided
with Dynace) into assembler, modify it and then assemble it.  Compiling
generics.c into assembler for reference is also helpful.  Running
the code through a machine instruction debugger is a must.

Basically what you need to do is write down the state of the stack
pointer and registers upon entry of the generic function.  Next, allow
the generic function to calculate the method to execute and call the
jumpTo code.  Finally, (and this is the part that needs to be modified)
the jumpTo code should call the method such that upon entry of the
method the stack pointer and registers look as though they did when
the generic function was entered.

For someone that knows assembler this process should take about an hour.

@subsubsection  GC & CPU Registers
If objects can exist in CPU registers, then the Dynace Garbage collector must
have access to them in order to mark them so as not to incorrectly collect
them.  Dynace has two ways of dealing with this.

The first way is to declare objects to be volatile.  This causes the
compiler not to keep any objects in registers.  While this is a safe
and portable solution it also slows the system down a bit.  It is,
however, required by 16 bit compilers because they tend to split
far pointers and store them on the stack in non-contiguous locations.

The second possibility is to explicitly mark objects which are
referenced via registers.  This is the best solution but requires
a small bit of inline assembler.  This code would be located in
@code{kernel.c} under the @code{GC} function.

@subsubsection Memory Alignment
Memory alignment determines where in memory objects can be stored.
By selecting efficient memory alignments you can maximize the code
and garbage collector efficiency.  Typically, alignment is on
4 byte boundaries for 32 bit systems and 2 byte on 16 bit systems.

Dynace includes support for 2 and 4 byte alignments via the
@code{ALIGN2} and @code{ALIGN4} macros located in @code{kernel.c}.


@subsubsection  Linear vs. Segmented Memory
As might be expected, Dynace runs much more efficiently in 32 bit,
linear memory mode.  Under linear mode Dynace can more quickly determine
object validity, a function which is often performed.

Dynace supports both modes via the @code{SEGMENTED_MEMORY} macro
contained within @code{kernel.c}.


@subsubsection Thread Timer
The Dynace multi-threader requires some mechanism whereby the OS
is able to execute an application defined function on a regular
basis.  The frequency would normally be between 50 and 100 times or
more a second.

@subsubsection Setjmp/longjmp Functionality
Since the multi-threader makes heave use of the setjmp/longjmp
system functions, it is important that those functions save and
restore the registers in addition to the stack frame.  This will
enable maximum reliability of local variables associated with
threads.
