@page

@section NumberArray Class
The @code{NumberArray} class, which is a subclass of @code{Array}, is an
abstract class used to combine common functionality of numeric arrays.
Although most of the functionality is actually implemented in the
subclasses of the @code{NumberArray} class, they are documented here
because of their common interface.


@subsection NumberArray Class Methods
The only class method associated with this class is the @code{New} method
which is used to create new instances of numeric arrays.  This method
is actually implemented by subclasses of @code{NumberArray}.








@deffn {New} New::NumberArray
@sp 2
@example
@group
ary = vNew(cls, rnk, ...)

object    cls, ary;
unsigned  rnk, ...
@end group
@end example
This class method is used to create a new numeric array.  This method
is actually implemented by and should only be used with subclasses of
the @code{NumberArray} class.  It is documented here because it is the
same for all subclasses of the @code{NumberArray} class.

@code{cls} would be one of the subclasses of the @code{NumberArray}
class (such as 
@iftex
@hfil @break 
@end iftex
@code{CharacterArray} @code{ShortArray}
@code{UnsignedShortArray} @code{LongArray} @code{FloatArray} or
@code{DoubleFloatArray}).  This will indicate the type of array to be
created.

@code{rnk} is the number of dimensions the new array should have.
The remaining arguments (each of type unsigned) indicates the size of
each consecutive dimension.  Note that the number of arguments following
@code{rnk} @emph{must} be the same as the value in @code{rnk}.

@code{ary} is the new array object created and will be initialized to all
zeros.
@example
@group
@exdent Example:

object  ary;

ary = vNew(DoubleFloatArray, 2, 5, 4);
/*  ary is a 5x4 double matrix  */
@end group
@end example
@c @sp 1
@c See also:  @code{}
@end deffn






@subsection NumberArray Instance Methods
The instance methods associated with this class are actually implemented
by subclasses of @code{NumberArray}.  They are documented here because
of their common interface with all subclasses of @code{NumberArray}.

Regardless of the array type, each subclass of @code{NumberArray}
implements a full complement of methods necessary to deal with all C
language data types.  In all cases, if the value being stored into an
array is not the same as the array, the value being stored will be
converted to the correct type prior to the assignment.  Also, if the
desired return type is not the same as the array type, the returned
value will be converted.  The type of the array will remain unchanged.
This naturally means that if you store the number 3.141 into an element
of a @code{ShortArray} you will only get 3.0 when attempting to get
back a @code{double} value.










@deffn {ChangeCharValue} ChangeCharValue::NumberArray
@sp 2
@example
@group
ary = vChangeCharValue(ary, val, ...);

object    ary;
int       val;
unsigned  ...
@end group
@end example
This method is used to change the value of one element of
array @code{ary} which should be an instance of a subclass of
@code{NumberArray}.

@code{val} is the value which the element of the array should be changed
to.  The reason it is of type @code{int} is because C promotes @code{char}
to @code{int} when variable arguments are used.

The arguments after the @code{val} argument (each an @code{unsigned})
are used to specify the exact index into each consecutive dimension of
the array.  The number of arguments after the @code{val} argument
@emph{must} be equal to the number of dimensions (or rank) of array
@code{ary}.  See @code{IndexOrigin::Array} for more information.

If @code{ary} and @code{val} do not represent the same types, normal
data type conversion will take place during the assignment without
changing the type of @code{ary}.

The value returned is the modified array passed.
@example
@group
@exdent Example:

object  ary;

ary = vNew(CharacterArray, 2, 5, 4);
vChangeCharValue(ary, 'C', 1, 2);
/*  ary[1][2] <- 'C'  */
@end group
@end example
@sp 1
See also:  @code{ChangeValue::NumberArray, ChangeDoubleValue::NumberArray,}
@iftex
@hfil @break @hglue .64in 
@end iftex
@code{ChangeCharValue::NumberArray}  ...
@end deffn







@deffn {ChangeDoubleValue} ChangeDoubleValue::NumberArray
@sp 2
@example
@group
ary = vChangeDoubleValue(ary, val, ...);

object    ary;
double    val;
unsigned  ...
@end group
@end example
This method is used to change the value of one element of
array @code{ary} which should be an instance of a subclass of
@code{NumberArray}.  This method should also be used for
arrays of type @code{FloatArray}.

@code{val} is the value which the element of the array should be changed
to.  

The arguments after the @code{val} argument (each an @code{unsigned})
are used to specify the exact index into each consecutive dimension of
the array.  The number of arguments after the @code{val} argument
@emph{must} be equal to the number of dimensions (or rank) of array
@code{ary}.  See @code{IndexOrigin::Array} for more information.

If @code{ary} and @code{val} do not represent the same types, normal
data type conversion will take place during the assignment without
changing the type of @code{ary}.

The value returned is the modified array passed.
@example
@group
@exdent Example:

object  ary;

ary = vNew(DoubleFloatArray, 2, 5, 4);
vChangeDoubleValue(ary, 3.141, 1, 2);
/*  ary[1][2] <- 3.141  */
@end group
@end example
@sp 1
See also:  @code{ChangeValue::NumberArray, ChangeCharValue::NumberArray,}
@iftex
@hfil @break @hglue .64in 
@end iftex
@code{ChangeShortValue::NumberArray}  ...
@end deffn









@deffn {ChangeLongValue} ChangeLongValue::NumberArray
@sp 2
@example
@group
ary = vChangeLongValue(ary, val, ...);

object    ary;
long      val;
unsigned  ...
@end group
@end example
This method is used to change the value of one element of
array @code{ary} which should be an instance of a subclass of
@code{NumberArray}.

@code{val} is the value which the element of the array should be changed
to.  

The arguments after the @code{val} argument (each an @code{unsigned})
are used to specify the exact index into each consecutive dimension of
the array.  The number of arguments after the @code{val} argument
@emph{must} be equal to the number of dimensions (or rank) of array
@code{ary}.  See @code{IndexOrigin::Array} for more information.

If @code{ary} and @code{val} do not represent the same types, normal
data type conversion will take place during the assignment without
changing the type of @code{ary}.

The value returned is the modified array passed.
@example
@group
@exdent Example:

object  ary;

ary = vNew(LongArray, 2, 5, 4);
vChangeLongValue(ary, 27L, 1, 2);
/*  ary[1][2] <- 27L  */
@end group
@end example
@sp 1
See also:  @code{ChangeValue::NumberArray, ChangeDoubleValue::NumberArray,}
@iftex
@hfil @break @hglue .64in 
@end iftex
@code{ChangeShortValue::NumberArray}  ...
@end deffn









@deffn {ChangeShortValue} ChangeShortValue::NumberArray
@sp 2
@example
@group
ary = vChangeShortValue(ary, val, ...);

object    ary;
int       val;
unsigned  ...
@end group
@end example
This method is used to change the value of one element of
array @code{ary} which should be an instance of a subclass of
@code{NumberArray}.

@code{val} is the value which the element of the array should be changed
to.  The reason it is of type @code{int} is because C promotes @code{short}
to @code{int} when variable arguments are used.

The arguments after the @code{val} argument (each an @code{unsigned})
are used to specify the exact index into each consecutive dimension of
the array.  The number of arguments after the @code{val} argument
@emph{must} be equal to the number of dimensions (or rank) of array
@code{ary}.  See @code{IndexOrigin::Array} for more information.

If @code{ary} and @code{val} do not represent the same types, normal
data type conversion will take place during the assignment without
changing the type of @code{ary}.

The value returned is the modified array passed.
@example
@group
@exdent Example:

object  ary;

ary = vNew(ShortArray, 2, 5, 4);
vChangeShortValue(ary, 27, 1, 2);
/*  ary[1][2] <- 27  */
@end group
@end example
@sp 1
See also:  @code{ChangeValue::NumberArray, ChangeDoubleValue::NumberArray,}
@iftex
@hfil @break @hglue .64in 
@end iftex
@code{ChangeShortValue::NumberArray}  ...
@end deffn








@deffn {ChangeUShortValue} ChangeUShortValue::NumberArray
@sp 2
@example
@group
ary = vChangeUShortValue(ary, val, ...);

object    ary;
unsigned  val;
unsigned  ...
@end group
@end example
This method is used to change the value of one element of
array @code{ary} which should be an instance of a subclass of
@code{NumberArray}.

@code{val} is the value which the element of the array should be changed
to.  The reason it is of type @code{unsigned} is because C promotes
@code{unsigned short} to @code{unsigned} when variable arguments are used.

The arguments after the @code{val} argument (each an @code{unsigned})
are used to specify the exact index into each consecutive dimension of
the array.  The number of arguments after the @code{val} argument
@emph{must} be equal to the number of dimensions (or rank) of array
@code{ary}.  See @code{IndexOrigin::Array} for more information.

If @code{ary} and @code{val} do not represent the same types, normal
data type conversion will take place during the assignment without
changing the type of @code{ary}.

The value returned is the modified array passed.
@example
@group
@exdent Example:

object  ary;

ary = vNew(UnsignedShortArray, 2, 5, 4);
vChangeUShortValue(ary, 27U, 1, 2);
/*  ary[1][2] <- 27U  */
@end group
@end example
@sp 1
See also:  @code{ChangeValue::NumberArray, ChangeDoubleValue::NumberArray,}
@iftex
@hfil @break @hglue .64in 
@end iftex
@code{ChangeShortValue::NumberArray}  ...
@end deffn








@deffn {ChangeValue} ChangeValue::NumberArray
@sp 2
@example
@group
ary = vChangeValue(ary, val, ...);

object    ary;
object    val;
unsigned  ...
@end group
@end example
This method is used to change the value of one element of array @code{ary}
which should be an instance of a subclass of @code{NumberArray}.

@code{val} should be an instance of the @code{Number} class and is the
value which the element of the array should be changed to.

The arguments after the @code{val} argument (each an @code{unsigned})
are used to specify the exact index into each consecutive dimension of
the array.  The number of arguments after the @code{val} argument
@emph{must} be equal to the number of dimensions (or rank) of array
@code{ary}.  See @code{IndexOrigin::Array} for more information.

If @code{ary} and @code{val} do not represent the same types, normal
data type conversion will take place during the assignment without
changing the types of @code{ary} or @code{val}.

The value returned is the modified array passed.
@example
@group
@exdent Example:

object  ary;
object  v;

ary = vNew(DoubleFloatArray, 2, 5, 4);
v = gNewWithDouble(Double, 3.141);
vChangeValue(ary, v, 1, 2);
/*  ary[1][2] <- 3.141  */
@end group
@end example
@sp 1
See also:  @code{ChangeDoubleValue::NumberArray,}
@iftex
@hfil @break @hglue .64in      
@end iftex
@code{ChangeCharValue::NumberArray  ...}
@end deffn












@deffn {CharValue} CharValue::NumberArray
@sp 2
@example
@group
v = vCharValue(ary, ...);

object    ary;
unsigned  ...
char      v;
@end group
@end example
This method is used to obtain the @code{char} value associated with a
particular element of an instance of a subclass of the
@code{NumberArray} class.

The arguments after the @code{ary} argument (each an @code{unsigned})
are used to specify the exact index into each consecutive dimension of
the array.  The number of arguments after the @code{ary} argument
@emph{must} be equal to the number of dimensions (or rank) of array
@code{ary}.  See @code{IndexOrigin::Array} for more information.

Note that this is one of the few generics which doesn't return a Dynace
object.  It returns a @code{char}.  If the instance does not represent
an instance of the @code{CharacterArray} class, the returned value of
whatever is represented will be converted to a @code{char}.
@example
@group
@exdent Example:

object  ary;
char    v;

ary = vNew(CharacterArray, 2, 5, 4);
v = vCharValue(ary, 1, 2);
/*  v has ary[1][2]  */
@end group
@end example
@sp 1
See also:  @code{ChangeCharValue::NumberArray}
@end deffn









@deffn {DoubleValue} DoubleValue::NumberArray
@sp 2
@example
@group
v = vDoubleValue(ary, ...);

object    ary;
unsigned  ...
double    v;
@end group
@end example
This method is used to obtain the @code{double} value associated with a
particular element of an instance of a subclass of the
@code{NumberArray} class.

The arguments after the @code{ary} argument (each an @code{unsigned})
are used to specify the exact index into each consecutive dimension of
the array.  The number of arguments after the @code{ary} argument
@emph{must} be equal to the number of dimensions (or rank) of array
@code{ary}.  See @code{IndexOrigin::Array} for more information.

Note that this is one of the few generics which doesn't return a Dynace
object.  It returns a @code{double}.  If the instance does not represent
an instance of the @code{DoubleFloatArray} class, the returned value of
whatever is represented will be converted to a @code{double}.
@example
@group
@exdent Example:

object  ary;
double  v;

ary = vNew(DoubleFloatArray, 2, 5, 4);
v = vDoubleValue(ary, 1, 2);
/*  v has ary[1][2]  */
@end group
@end example
@sp 1
See also:  @code{ChangeDoubleValue::NumberArray}
@end deffn











@deffn {LongValue} LongValue::NumberArray
@sp 2
@example
@group
v = vLongValue(ary, ...);

object    ary;
unsigned  ...
long      v;
@end group
@end example
This method is used to obtain the @code{long} value associated with a
particular element of an instance of a subclass of the
@code{NumberArray} class.

The arguments after the @code{ary} argument (each an @code{unsigned})
are used to specify the exact index into each consecutive dimension of
the array.  The number of arguments after the @code{ary} argument
@emph{must} be equal to the number of dimensions (or rank) of array
@code{ary}.  See @code{IndexOrigin::Array} for more information.

Note that this is one of the few generics which doesn't return a Dynace
object.  It returns a @code{long}.  If the instance does not represent
an instance of the @code{LongArray} class, the returned value of
whatever is represented will be converted to a @code{long}.
@example
@group
@exdent Example:

object  ary;
long    v;

ary = vNew(LongArray, 2, 5, 4);
v = vLongValue(ary, 1, 2);
/*  v has ary[1][2]  */
@end group
@end example
@sp 1
See also:  @code{ChangeLongValue::NumberArray}
@end deffn









@deffn {ShortValue} ShortValue::NumberArray
@sp 2
@example
@group
v = vShortValue(ary, ...);

object    ary;
unsigned  ...
short     v;
@end group
@end example
This method is used to obtain the @code{short} value associated with a
particular element of an instance of a subclass of the
@code{NumberArray} class.

The arguments after the @code{ary} argument (each an @code{unsigned})
are used to specify the exact index into each consecutive dimension of
the array.  The number of arguments after the @code{ary} argument
@emph{must} be equal to the number of dimensions (or rank) of array
@code{ary}.  See @code{IndexOrigin::Array} for more information.

Note that this is one of the few generics which doesn't return a Dynace
object.  It returns a @code{short}.  If the instance does not represent
an instance of the @code{ShortArray} class, the returned value of
whatever is represented will be converted to a @code{short}.
@example
@group
@exdent Example:

object  ary;
short   v;

ary = vNew(ShortArray, 2, 5, 4);
v = vShortValue(ary, 1, 2);
/*  v has ary[1][2]  */
@end group
@end example
@sp 1
See also:  @code{ChangeShortValue::NumberArray}
@end deffn






@deffn {UnsignedShortValue} UnsignedShortValue::NumberArray
@sp 2
@example
@group
v = vUnsignedShortValue(ary, ...);

object    ary;
unsigned  ...
unsigned short     v;
@end group
@end example
This method is used to obtain the @code{unsigned short} value associated with a
particular element of an instance of a subclass of the
@code{NumberArray} class.

The arguments after the @code{ary} argument (each an @code{unsigned})
are used to specify the exact index into each consecutive dimension of
the array.  The number of arguments after the @code{ary} argument
@emph{must} be equal to the number of dimensions (or rank) of array
@code{ary}.  See @code{IndexOrigin::Array} for more information.

Note that this is one of the few generics which doesn't return a Dynace
object.  It returns an @code{unsigned short}.  If the instance does not represent
an instance of the @code{UnsignedShortArray} class, the returned value of
whatever is represented will be converted to a @code{unsigned short}.
@example
@group
@exdent Example:

object  ary;
unsigned short   v;

ary = vNew(UnsignedShortArray, 2, 5, 4);
v = vUnsignedShortValue(ary, 1, 2);
/*  v has ary[1][2]  */
@end group
@end example
@sp 1
See also:  @code{ChangeUShortValue::NumberArray}
@end deffn




