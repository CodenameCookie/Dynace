@page

@section Array Class
@pdfsection{Array Class}
The @code{Array} class is used to represent heterogeneous and
non-heterogeneous indexable collections of arbitrary dimensions.  These
arrays are designed to be flexible -- they have the ability to be
dynamically re-dimensioned or reshaped -- and space and speed efficient.
Although most of the actual functionality contained in subclasses of the
@code{Array} class are implemented in the @code{Array} class, most users
will find the subclasses more useful.  The documentation associated with
this class serves to document the functionality which is common to all
its subclasses.



@subsection Array Class Methods
@pdfsubsection{Array Class Methods}
The majority of class methods associated with this class are mainly used
by and documented in subclasses of the @code{Array} class.





@pdfsubsubsection {IndexOrigin}
@deffn {IndexOrigin} IndexOrigin::Array
@sp 2
@example
@group
s = gIndexOrigin(s, n)

object  s;
int     n;
@end group
@end example
This method is used to set the index origin associated with @emph{all}
array indexing.  The default is 0.  This means that a 4 element, 1
dimension array would be indexed by 0, 1, 2, 3.  If the index origin
is set to one that same array would be indexed by 1, 2, 3, 4.

The index origin may be changed any number of times and has no actual effect
on any arrays (either created previously or subsequently).  It only
effects how indexing works.
@c @example
@c @group
@c @exdent Example:
@c 
@c @end group
@c @end example
@c @sp 1
@c See also:  @code{}
@end deffn






@subsection Array Instance Methods
@pdfsubsection{Array Instance Methods}
Instance methods associated with this class provide functionality which
is common to all subclasses of the @code{Array} class.








@pdfsubsubsection {ArrayPointer}
@deffn {ArrayPointer} ArrayPointer::Array
@sp 2
@example
@group
p = gArrayPointer(s)

object  s;
void    *p;
@end group
@end example
This method returns a pointer to the first element of an array represented
be @code{s}.  The return type will be a pointer to the type of array it
is.  For example a @code{ShorArray} will return a pointer to a @code{short}
and a @code{DoubleFloatArray} will return a pointer to a @code{double}.
Consecutive elements may be accessed by incrementing the pointer (although
this is not the preferred way of going through an array).
@c @example
@c @group
@c @exdent Example:
@c 
@c @end group
@c @end example
@c @sp 1
@c See also:  @code{}
@end deffn









@pdfsubsubsection {Copy}
@deffn {Copy} Copy::Array
@sp 2
@example
@group
r = gCopy(i);

object  i;
object  r;
@end group
@end example
This method is used to create a new array object which is an exact copy,
including type, rank, shape and values, of @code{i}.  The value returned
is the new array object.
@c @example
@c @group
@c @exdent Example:
@c 
@c @end group
@c @end example
@sp 1
See also:  @code{DeepCopy::Array}
@end deffn








@pdfsubsubsection {DeepCopy}
@deffn {DeepCopy} DeepCopy::Array
@sp 2
@example
@group
r = gDeepCopy(i);

object  i;
object  r;
@end group
@end example
This method is used to create a new array object which is an exact copy,
including type, rank, shape and values, of @code{i}.  The value returned
is the new array object.

The difference between this method and @code{Copy::Array} is that if the
array being copied is an @code{ObjectArray}, @code{DeepCopy} will also
make copies of the elements.  @code{Copy} will not.
@c @example
@c @group
@c @exdent Example:
@c 
@c @end group
@c @end example
@sp 1
See also:  @code{Copy::Array}
@end deffn









@pdfsubsubsection {DeepDispose}
@deffn {DeepDispose} DeepDispose::Array
@sp 2
@example
@group
r = gDeepDispose(s)

object  s;
object  r;     /*  NULL  */
@end group
@end example
This method disposes and frees all memory associated with an @code{Array}
object.  If it is an instance of the @code{ObjectArray} class each
non-NULL element will also be @code{DeepDispose}'d.  

The value returned is always @code{NULL} and may be used to null out
the variable which contained the object being disposed in order to
avoid future accidental use.
@c @example
@c @group
@c @exdent Example:
@c 
@c @end group
@c @end example
@c @sp 1
@c See also:  @code{}
@end deffn












@pdfsubsubsection {Dispose}
@deffn {Dispose} Dispose::Array
@sp 2
@example
@group
r = gDispose(s)

object  s;
object  r;     /*  NULL  */
@end group
@end example
This method disposes and frees all memory associated with an @code{Array}
object.  

The value returned is always @code{NULL} and may be used to null out
the variable which contained the object being disposed in order to
avoid future accidental use.
@c @example
@c @group
@c @exdent Example:
@c 
@c @end group
@c @end example
@c @sp 1
@c See also:  @code{}
@end deffn





@pdfsubsubsection {Equal}
@deffn {Equal} Equal::Array
@sp 2
@example
@group
r = gEqual(i, a)

object  i, a;
int     r;
@end group
@end example
This method is used to determine the equality of two arrays.  If they
(@code{i} and @code{a}) are instances of the same class, same rank,
same shape and same values than @code{Equal} returns a 1.  @code{Equal}
returns 0 otherwise.
@c @example
@c @group
@c @exdent Example:
@c 
@c @end group
@c @end example
@sp 1
See also:  @code{EQ}
@end deffn








@pdfsubsubsection {Index}
@deffn {Index} Index::Array
@sp 2
@example
@group
p = gIndex(i, idx)

object     i;
va_list  idx;
@end group
@end example
This method is used to obtain a pointer to a single element of an array.
It is only intended to be used internally by the subclasses of
@code{Array}.  @code{i} is the array to be indexed and @code{idx} is a
va_list where each element indicates successive indices into the
array.  The number of elements in @code{idx} @emph{must} equal the rank
of the array.  The pointer returned may be typecast to the appropriate
type.
@c @example
@c @group
@c @exdent Example:
@c 
@c @end group
@c @end example
@sp 1
See also:  @code{ChangeValue::NumberArray, ShortValue::NumberArray}
@end deffn








@pdfsubsubsection {Rank}
@deffn {Rank} Rank::Array
@sp 2
@example
@group
r = gRank(i)

object    i;
unsigned  r;
@end group
@end example
This method is used to obtain the number of dimensions a particular
array (@code{i}) has.
@c @example
@c @group
@c @exdent Example:
@c 
@c @end group
@c @end example
@sp 1
See also:  @code{Shape::Array, Size::Array, Reshape::Array}
@end deffn











@pdfsubsubsection {Reshape}
@deffn {Reshape} Reshape::Array
@sp 2
@example
@group
i = vReshape(i, r, ...)

object     i;
unsigned   r;
@end group
@end example
This method is used to change the rank and shape of an array (@code{i}).
@code{r} is the new rank (number of dimensions) of the array and
the remaining arguments (type @code{unsigned}) indicate the size of
each consecutive dimension.  Not that the number of arguments after
@code{r} @emph{must} be equal to the number @code{r}.

If the new shape has fewer elements than the original, the remaining
elements are discarded.  If, however, the new shape has more elements,
the elements in the original array will be reused (from the beginning)
over and over until the new array is filled.

The value returned is the modified array passed.
@c @example
@c @group
@c @exdent Example:
@c 
@c @end group
@c @end example
@sp 1
See also:  @code{Shape::Array, Rank::Array}
@end deffn







@pdfsubsubsection {Shape}
@deffn {Shape} Shape::Array
@sp 2
@example
@group
s = gShape(i)

object  i;
object  s;
@end group
@end example
This method is used to obtain the shape (or dimensions) of a particular
array (@code{i}).  The value returned (@code{s}) is an instance of the
@code{ShortArray} class, each element of which contains a consecutive
dimension.
@c @example
@c @group
@c @exdent Example:
@c 
@c @end group
@c @end example
@sp 1
See also:  @code{Reshape::Array, Rank::Array, Size::Array}
@end deffn








@pdfsubsubsection {Size}
@deffn {Size} Size::Array
@sp 2
@example
@group
s = gSize(i)

object  i;
int     s;
@end group
@end example
This method is used to obtain the total number of elements in array @code{i}.
@c @example
@c @group
@c @exdent Example:
@c 
@c @end group
@c @end example
@sp 1
See also:  @code{Rank::Array, Shape::Array}
@end deffn







@pdfsubsubsection {StringRep}
@deffn {StringRep} StringRep::Array
@sp 2
@example
@group
s = gStringRep(i);

object  i;
object  s;
@end group
@end example
This method is used to generate an instance of the @code{String} class
which represents the type, rank, shape and values associated with
@code{i}.  This is often used to print or display the value.  It is also
used by @code{PrintValue::Object} and indirectly by @code{Print::Object}
(two methods useful during the debugging phase of a project) in order to
directly print an object's value.
@c @example
@c @group
@c @exdent Example:
@c 
@c @end group
@c @end example
@sp 1
See also:  @code{PrintValue::Object, Print::Object, StringRepValue::Array}
@end deffn






@pdfsubsubsection {StringRepValue}
@deffn {StringRepValue} StringRepValue::Array
@sp 2
@example
@group
s = gStringRepValue(i);

object  i;
object  s;
@end group
@end example
This method is used to generate an instance of the @code{String} class
which represents the values associated with @code{i}.  This is often
used to print or display the value.  It is also used by
@code{PrintValue::Object} and indirectly by @code{Print::Object}
(two methods useful during the debugging phase of a project)
in order to directly print an object's value.
@c @example
@c @group
@c @exdent Example:
@c 
@c @end group
@c @end example
@sp 1
See also:  @code{PrintValue::Object, Print::Object, StringRep::Array}
@end deffn







