@page

@section FindFile Class
@pdfsection{FindFile Class}
The @code{FindFile} class is designed to provide a portable means of
enumerating over the file names contained within a particular file
system directory.  Currently, this class has only been ported to the DOS
and all Windows environments.  There is no current support for Unix or
Macintosh environments.




@subsection FindFile Class Methods
@pdfsubsection{FindFile Class Methods}
The only class method used in this class is one to create instances of itself.





@pdfsubsubsection {NewFindFile}
@deffn {NewFindFile} NewFindFile::FindFile
@sp 2
@example
@group
i = gNewFindFile(FindFile, name, attr);

char    *name;
int     attr;
object  i;
@end group
@end example
This class method creates instances of the @code{FindFile} class.  Each
such instance is initialized to enable the instance to enumerate through
all the file names which meet a particular set of parameters.

The @code{name} parameter is used to specify the pattern of the file name
to be searched for.  This pattern may or may not contain a relative or
absolute directory path but @emph{must} contain some wildcard
characters (such as ``?'' and ``*'').

The @code{attr} parameter is used to narrow the search to file names
which contain certain attributes.  The list of available attribute
type (which should be or'ed together) is as follows:
@example
@group
FF_FILE                 include regular files
FF_DIRECTORY            include directories

FF_READWRITE            include files or directories in which
                        you have read/write permission
FF_READONLY             include files or directories in which
                        you only have read permission

FF_ARCHIVE_ONLY         don't include files/directories which
                        don't have their archive bit set
FF_HIDDEN               include hidden files
FF_SYSTEM               include system files
@end group
@end example

The new instance is returned.  
@example
@group
@exdent Example:
object  ff;
char    *file;

ff = gNewFindFile(FindFile, "*.c", FF_FILE | FF_READWRITE);
while (file = gNextFile(ff))  {
        do something with file
}
gDispose(ff);
@end group
@end example
@sp 1
See also:  @code{NextFile::FindFile}
@end deffn





@subsection FindFile Instance Methods
@pdfsubsection{FindFile Instance Methods}
The instance methods associated with this class provide a means for
sequentially enumerating through the selected file names and obtaining
some degree of information associated with each particular file.




@pdfsubsubsection {Attributes}
@deffn {Attributes} Attributes::FindFile
@sp 2
@example
@group
attr = gAttributes(ff);

object       ff;
unsigned     attr;
@end group
@end example
This method is used to obtain attribute information relating to the last
selected file returned from @code{NextFile}.  The attribute flags are those
defined under @code{NewFindFile}.  If no valid file has been returned from
@code{NextFile} this function simply returns 0.
@example
@group
@exdent Example:
object    ff;
char     *file;
unsigned  attr;

ff = gNewFindFile(FindFile, "*.*",
                  FF_FILE | FF_DIRECTORY | FF_READWRITE);
while (file = gNextFile(ff))  {
        attr = gAttributes(ff);
        if (attr & FF_DIRECTORY)
                handle a directory
        else if (attr & FF_FILE)
                handle a file
}
gDispose(ff);
@end group
@end example
@sp 1
See also:  @code{NewFindFile::FindFile, NextFile::FindFile,}
@hfil @break @hglue .64in      @code{WriteTime::FindFile}
@end deffn








@pdfsubsubsection {Length}
@deffn {Length} Length::FindFile
@sp 2
@example
@group
len = gLength(ff);

object  ff;
long    len;
@end group
@end example
This method is used to obtain the length or size (in bytes) of the file
relating to the last selected file returned from @code{NextFile}.  If
no valid file has been returned from @code{NextFile} this function
simply returns -1.
@example
@group
@exdent Example:
object  ff;
char    *file;
long    len;

ff = gNewFindFile(FindFile, "*.c", FF_FILE | FF_READWRITE);
while (file = gNextFile(ff))  {
        len = gLength(ff);
}
gDispose(ff);
@end group
@end example
@sp 1
See also:  @code{Attributes::FindFile, NextFile::FindFile}
@end deffn









@pdfsubsubsection {Name}
@deffn {Name} Name::FindFile
@sp 2
@example
@group
nam = gName(ff);

object  ff;
char    *nam;
@end group
@end example
This method is used to obtain the name of the last file found via
@code{NextFile}.  It will be the same name as returned from
@code{NextFile}.  If no valid file has been returned from
@code{NextFile} this function simply returns @code{NULL}.
@example
@group
@exdent Example:
object  ff;
char   *file;

ff = gNewFindFile(FindFile, "*.c", FF_FILE | FF_READWRITE);
while (gNextFile(ff))  {
        file = gName(ff);
}
gDispose(ff);
@end group
@end example
@sp 1
See also:  @code{NextFile::FindFile}
@end deffn











@pdfsubsubsection {NextFile}
@deffn {NextFile} NextFile::FindFile
@sp 2
@example
@group
nam = gNextFile(ff);

object  ff;
char   *nam;
@end group
@end example
This method is used to obtain the next file name in the sequential list
specified when using @code{NewFindFile}.  It also changes the context
associated with the object @code{ff} such that all other methods
operated on @code{ff} will refer to the file returned in the last call
to this method.  In spite of the name of this method, it is also used
to obtain the first file name in the list, therefore, this method must
be called prior to any other method on @code{ff} in order to obtain
the first file context.

This method returns the name of the file associated with the file
context this method establishes.  If there is no first file or
additional files this method returns @code{NULL}.
@c @example
@c @group
@c @exdent Example:
@c 
@c @end group
@c @end example
@sp 1
See also:  @code{NewFindFirst::FindFile}
@end deffn











@pdfsubsubsection {WriteTime}
@deffn {WriteTime} WriteTime::FindFile
@sp 2
@example
@group
tim = gWriteTime(ff);

object  ff;
long    tim;
@end group
@end example
This method is used to obtain the date and time the file associated
with the @code{ff} context was last written to.  The value returned
can be typecast to the common @code{time_t} type and represents the
number of seconds past some date like 1970.  If no valid file
context was setup via @code{NextFile}, this method returns -1.
@c @example
@c @group
@c @exdent Example:
@c 
@c @end group
@c @end example
@sp 1
See also:  @code{Attributes::FindFile, NextFile::FindFile}
@end deffn







