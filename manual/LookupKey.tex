@page

@section LookupKey Class
@pdfsection{LookupKey Class}
The @code{LookupKey} class, which is a subclass of @code{Association},
is used to represent an object which contains a key.  It is normally
used in conjunction with the @code{ObjectAssociation}, @code{Set} and
@code{Dictionary} classes in order to manage a group of arbitrary
objects.

Since instances of the @code{LookupKey} class only contain enough
information to describe the key, the @code{LookupKey} class can be
thought of as an abstract class.  It is never used by itself (instances
of it are never created).  Instead a new class would be created which
would have the @code{LookupKey} class as a superclass.  This new class
would describe the data items which would be stored at each link.  The
@code{LookupKey} class would keep track of the key to the object.
Alternatively, a new class with the same methods as this class could be
created and used with the @code{Set} and @code{Dictionary} classes.

The @code{LookupKey} class also has a subclass called
@code{ObjectAssociation} which associates a single arbitrary object with
the key.

See the examples included with the Dynace system for an illustration of the
use of the @code{Set}/@code{Dictionary} related classes.



@subsection LookupKey Class Methods
@pdfsubsection{LookupKey Class Methods}
This class has only a single class method which is used to create and
initialize instances of itself.   Although this class can handle
@code{NULL} key values, their use is discouraged.






@pdfsubsubsection {NewWithObj}
@deffn {NewWithObj} NewWithObj::LookupKey
@sp 2
@example
@group
i = gNewWithObj(LookupKey, key);

object  key;
object  i;
@end group
@end example
This class method creates instances of the @code{LookupKey} class.
@code{key} is used to initialize the key associated with the instance
created.  The new instance is returned.
@c @example
@c @group
@c @exdent Example:
@c 
@c @end group
@c @end example
@c @sp 1
@c See also:  @code{}
@end deffn




@subsection LookupKey Instance Methods
@pdfsubsection{LookupKey Instance Methods}
The instance methods associated with this class are used access, change,
dispose and print the key associated with the instances of this class.
Additional functionality, although implemented by this class, is documented
in @code{LookupKey}'s superclass, @code{Association}.  This is done
because of the common interface these methods share with all subclasses
of @code{Association}.







@pdfsubsubsection {ChangeKey}
@deffn {ChangeKey} ChangeKey::LookupKey
@sp 2
@example
@group
old = gChangeKey(i, key);

object  i;
object  key;
object  old;
@end group
@end example
This method is used to change the key associated with instance @code{i}.
@code{key} is the new key.  The old key is simply replaced, it is not
disposed.  This method returns the old key object.
@c @example
@c @group
@c @exdent Example:
@c 
@c @end group
@c @end example
@sp 1
See also:  @code{Key::LookupKey}
@end deffn













@pdfsubsubsection {DeepCopy}
@deffn {DeepCopy} DeepCopy::LookupKey
@sp 2
@example
@group
c = gDeepCopy(i);

object  i, c;
@end group
@end example
This method is used to create a new instance of the
@code{LookupKey} class which contains a @emph{copy} of the
key value from the original @code{LookupKey}.  @code{DeepCopy} is
used to create the copy of the value.  The new
@code{LookupKey} instance is returned.
@c @example
@c @group
@c @exdent Example:
@c 
@c @end group
@c @end example
@sp 1
See also:  @code{Copy::Object}
@end deffn













@pdfsubsubsection {DeepDispose}
@deffn {DeepDispose} DeepDispose::LookupKey
@sp 2
@example
@group
r = gDeepDispose(i);

object  i;
object  r;     /*  NULL  */
@end group
@end example
This method is used to dispose of instance @code{i} as well as the key
associated with it.  It disposes of the key by calling the
@code{DeepDispose} method on it.

The value returned is always @code{NULL} and may be used to null out
the variable which contained the object being disposed in order to
avoid future accidental use.
@c @example
@c @group
@c @exdent Example:
@c 
@c @end group
@c @end example
@sp 1
See also:  @code{Dispose::Object}
@end deffn








@pdfsubsubsection {Key}
@deffn {Key} Key::LookupKey
@sp 2
@example
@group
key = gKey(i);

object  i;
object  key;
@end group
@end example
This method is used to obtain the key associated with instance @code{i}.
@c @example
@c @group
@c @exdent Example:
@c 
@c @end group
@c @end example
@sp 1
See also:  @code{ChangeKey::LookupKey}
@end deffn










@pdfsubsubsection {StringRepValue}
@deffn {StringRepValue} StringRepValue::LookupKey
@sp 2
@example
@group
s = gStringRepValue(i);

object  i;
object  s;
@end group
@end example
This method is used to generate an instance of the @code{String} class
which represents the value associated with @code{i}.  This is often
used to print or display the value.  It is also used by
@code{PrintValue::Object} and indirectly by @code{Print::Object}
(two methods useful during the debugging phase of a project)
in order to directly print an object's value.
@c @example
@c @group
@c @exdent Example:
@c 
@c object  x;
@c object  s;
@c 
@c x = gNewWithPtr(Pointer, NULL);
@c s = gStringRepValue(x);
@c @end group
@c @end example
@sp 1
See also:  @code{PrintValue::Object, Print::Object}
@end deffn








