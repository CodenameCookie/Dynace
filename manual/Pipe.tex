@page

@section Pipe Class
The @code{Pipe} class is used to provide a mechanism for threads
to transfer information via a uni-directional byte stream.  If
bi-directional communications is needed two pipes may be used.

The @code{Pipe} class is a subclass of @code{Stream} and as such shares
much of its functionality with other subclasses of the @code{Stream}
class.  Although this class implements most of its own functionality, it
is documented as part of the @code{Stream} class because most of the
interface is the same for all subclasses of @code{Stream}.  Differences
are documented in this section.


See the relevant text for a detailed description of pipes.





@subsection Pipe Class Methods
There are only two class methods associated with the @code{Pipe} class.
They are used to create new pipes and find existing ones.








@deffn {FindStr} FindStr::Pipe
@sp 2
@example
@group
p = gFindStr(Pipe, name)

char    *name;
object  p;
@end group
@end example
This method is used to obtain a pipe object from its name.  @code{p}
is the pipe object found or @code{NULL} if not found.
@c @example
@c @group
@c @exdent Example:
@c 
@c @end group
@c @end example
@sp 1
See also:  @code{New::Pipe}
@end deffn





@deffn {New} New::Pipe
@sp 2
@example
@group
p = gNew(Pipe)

object  p;
@end group
@end example
This class method is used to create and initialize a new unnamed pipe
with a default buffer size.

The value returned (@code{p}) is the pipe object created.
@c @example
@c @group
@c @exdent Example:
@c 
@c @end group
@c @end example
@sp 1
See also:  @code{NewWithStrInt::Pipe, FindStr::Pipe, Dispose::Pipe}
@end deffn










@deffn {NewWithStrInt} NewWithStrInt::Pipe
@sp 2
@example
@group
p = gNewWithStrInt(Pipe, name, bufsiz)

char    *name;
int     bufsiz;
object  p;
@end group
@end example
This class method is used to create and initialize a new pipe.

@code{name} is the name of the new pipe and may be @code{NULL} if
no name is to be associated with the pipe.

@code{bufsiz} is the size of the buffer associated with the pipe.  The
size of the buffer determines how many bytes of information may be
buffered before a thread has to held.  If set too small the system will
become inefficient because of the constant switching between threads.
Typical values are 128 or 512.  It depends on the quantity and size of
the data being transferred.

The value returned (@code{p}) is the pipe object created.  Other
threads may obtain the thread object by the use of the @code{FindStr::Pipe}
method.  
@c @example
@c @group
@c @exdent Example:
@c 
@c @end group
@c @end example
@sp 1
See also:  @code{New::Pipe, FindStr::Pipe, Dispose::Pipe}
@end deffn



















@subsection Pipe Instance Methods
The instance methods associated with this class are used to put bytes
on the pipe, take bytes off, dispose of, and obtain information
about particular pipes.

Although this class implements most of its own functionality, it is
documented as part of the @code{Stream} class because most of the
interface is the same for all subclasses of @code{Stream}.  Differences
are documented in this section.













@deffn {DeepDispose} DeepDispose::Pipe
@sp 2
@example
@group
r = gDeepDispose(p)

object  p;
object  r;     /*  NULL  */
@end group
@end example
This method is used to dispose of a pipe instance.  
@iftex
@hfil @break 
@end iftex
It is the same as @code{Dispose::Pipe}.

The value returned is always @code{NULL} and may be used to null out
the variable which contained the object being disposed in order to
avoid future accidental use.
@c @example
@c @group
@c @exdent Example:
@c 
@c @end group
@c @end example
@c @sp 1
@c See also:  @code{}
@end deffn








@deffn {Dispose} Dispose::Pipe
@sp 2
@example
@group
r = gDispose(p)

object  p;
object  r;     /*  NULL  */
@end group
@end example
This method is used to dispose of a pipe instance.  
@iftex
@hfil @break 
@end iftex
It is the same as @code{DeepDispose::Pipe}.

The value returned is always @code{NULL} and may be used to null out
the variable which contained the object being disposed in order to
avoid future accidental use.
@c @example
@c @group
@c @exdent Example:
@c 
@c @end group
@c @end example
@c @sp 1
@c See also:  @code{}
@end deffn








@deffn {Length} Length::Pipe
@sp 2
@example
@group
n = gLength(p)

object  p;
long    n;
@end group
@end example
This method is used to determine the number of bytes currently on
the pipe's buffer.  Note that this method returns a @code{long}.
This is to retain consistency with @code{Length::Stream}.
@c @example
@c @group
@c @exdent Example:
@c 
@c @end group
@c @end example
@sp 1
See also:  @code{Room::Pipe, Size::Pipe}
@end deffn







@deffn {Mode} Mode::Pipe
@sp 2
@example
@group
p = gMode(p, rblock, wblock)

object  p;
int     rblock;   /*  read block   */
int     wblock;   /*  write block  */
@end group
@end example
This method is used to set the blocking attributes of a particular
@code{Pipe}.  A 1 turns on the blocking and a 0 turns it off.

If blocking is turned on when a particular request for reading or
writing occurs and there is insufficient bytes or room available
the thread requesting the read or write will go into a hold state
until another thread can either supply the requested bytes or
make the necessary room available.  If blocking is not turned on
then the particular read or write request will only perform the
operation with as many bytes is available at the time.  There
will be no waiting.

The @code{rblock} flag effects read requests and the @code{wblock}
effects write requests.

This method returns the @code{Pipe} instance passed.
@c @example
@c @group
@c @exdent Example:
@c 
@c @end group
@c @end example
@sp 1
See also:  @code{Room::Pipe, Length::Pipe}
@end deffn









@deffn {Room} Room::Pipe
@sp 2
@example
@group
n = gRoom(p)

object  p;
int     n;
@end group
@end example
This method is used to determine the number of bytes currently available
on the pipe's buffer.

@c @code{block} is a flag used to determine what to do if there is no room
@c available on the pipe's buffer.  If @code{block} is set to 1 and no room
@c is available on the pipe's buffer the thread calling this method will be
@c placed on hold until some room is available.  If @code{block} is 0 and
@c there is no room available on the pipe's buffer a 0 will be returned.
@c This allows the user a convenient way of making a thread wait wait for
@c room or just finding out if any room is available and being able to
@c continue.
@c @example
@c @group
@c @exdent Example:
@c 
@c @end group
@c @end example
@sp 1
See also:  @code{Length::Pipe}
@end deffn












@deffn {Size} Size::Pipe
@sp 2
@example
@group
n = gSize(p)

object  p;
int     n;
@end group
@end example
This method is used to determine the size of the buffer associated with a
@code{Pipe}.
@c @example
@c @group
@c @exdent Example:
@c 
@c @end group
@c @end example
@sp 1
See also:  @code{Length::Pipe}
@end deffn
















