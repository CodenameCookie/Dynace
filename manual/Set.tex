@page

@section Set Class
@pdfsection{Set Class}
The @code{Set} class is used to store a collection of objects which
contain some means of unique identification for quick access.  The
objects stored may be any Dynace object.  @code{gHash} and
@code{gCompare} will be used on the object to be added in order to store
and compare it amongst other objects in the @code{Set}.

Since Dynace contains default methods for @code{Hash} and @code{Compare},
any group of Dynace objects may be stored in any @code{Set} instance.

If values are to be associated with the keys one of the @code{Dictionary}
classes would be preferred.  The @code{Set} class, however, provides
most of the functionality needed by the @code{Dictionary} classes.

See the examples included with the Dynace system for an illustration of the
use of the @code{Set}/@code{Dictionary} related classes.



@subsection Set Class Methods
@pdfsubsection{Set Class Methods}
The only class method used in this class is one to create instances of itself.




@pdfsubsubsection {New}
@deffn {New} New::Set
@sp 2
@example
@group
i = gNew(Set);

object  i;
@end group
@end example
This class method creates instances of the @code{Set} class.  The set
created is designed to accommodate rather small sets of about 30
elements.  If a @code{Set} becomes too full it automatically re-sizes
itself.  See @code{NewWithInt::Set} for a method of creating a set
while specifying its initial size.

The new instance is returned.
@c @example
@c @group
@c @exdent Example:
@c 
@c @end group
@c @end example
@sp 1
See also:  @code{NewWithInt::Set, Resize::Set}
@end deffn







@pdfsubsubsection {NewWithInt}
@deffn {NewWithInt} NewWithInt::Set
@sp 2
@example
@group
i = gNewWithInt(Set, size);

int     size;
object  i;
@end group
@end example
This class method creates instances of the @code{Set} class.
@code{size} is used to give the instance some idea of the initial maximum number
of objects to be stored in the @code{Set}.  The actual number of objects may
exceed this number, however, some loss of efficiency will occur.

Since a @code{Set} is implemented as a hash table it is best if @code{size}
is actually larger than the number of objects to be held.  It should
be an odd number or better yet a prime number.  Keep in mind, however,
that there is storage requirements associated with this number and
the use of a very large number will consume large memory resources.
In reality most reasonable numbers will work fine.

Regardless of the size specified the set object will never overflow.
It will just be less efficient if many more elements then @code{size}
are added.

The new instance is returned.
@c @example
@c @group
@c @exdent Example:
@c 
@c @end group
@c @end example
@sp 1
See also:  @code{New::Set, Resize::Set}
@end deffn





@subsection Set Instance Methods
@pdfsubsection{Set Instance Methods}
The instance methods associated with this class provide all the means
to add, inquire, remove and print instances of the @code{Set} class.












@pdfsubsubsection {Add}
@deffn {Add} Add::Set
@sp 2
@example
@group
r = gAdd(i, luk);

object  i;
object  luk;
object  r;
@end group
@end example
This method is used to add a new object (@code{luk}) to the @code{Set} instance
(@code{i}).  If an object with the same key already exists in the @code{Set} it
will be left as is and @code{Add} will return @code{NULL}.  If
@code{luk} is added it will also be returned.

@code{luk} may be any Dynace object.  The @code{gHash} and
@code{gCompare} generics are used to find and compare the various
objects in the @code{Set}.
@c @example
@c @group
@c @exdent Example:
@c 
@c @end group
@c @end example
@sp 1
See also:  @code{Find::Set, FindAdd::Set, RemoveObj::Set}
@end deffn











@pdfsubsubsection {Copy}
@deffn {Copy} Copy::Set
@sp 2
@example
@group
so = gCopy(i);

object  i, so;
@end group
@end example
This method is used to create a new instance of the @code{Set} class
with the @emph{same} elements as @code{Set} @code{i}.  Or put more simply,
it makes a copy.  The new @code{Set} instance is returned.

Note that this method is designed to also work correctly for the
@code{Dictionary} subclasses of @code{Set}.
@example
@group
@exdent Example:

object  x, y;

x = gNewWithInt(Set, 101);
y = gCopy(x);
@end group
@end example
@sp 1
See also:  @code{NewWithInt::Set, DeepCopy::Set}
@end deffn













@pdfsubsubsection {DeepCopy}
@deffn {DeepCopy} DeepCopy::Set
@sp 2
@example
@group
so = gDeepCopy(i);

object  i, so;
@end group
@end example
This method is used to create a new instance of the @code{Set} class
which contains @emph{copies} of all the elements from the original @code{Set}.
@code{DeepCopy} is used on each element to create the copies.
The new @code{Set} instance is returned.

Note that this method is designed to also work correctly for the
@code{Dictionary} subclasses of @code{Set}.
@example
@group
@exdent Example:

object  x, y;

x = gNewWithInt(Set, 100);
y = gDeepCopy(x);
@end group
@end example
@sp 1
See also:  @code{NewWithInt::Set, Copy::Set}
@end deffn








@pdfsubsubsection {DeepDispose}
@deffn {DeepDispose} DeepDispose::Set
@sp 2
@example
@group
r = gDeepDispose(i);

object  i;
object  r;     /*  NULL  */
@end group
@end example
This method is used to dispose of an entire @code{Set} instance.  It also
disposes of all the objects or associations in the @code{Set} by use of the
@code{DeepDispose} method.  

The value returned is always @code{NULL} and may be used to null out
the variable which contained the object being disposed in order to
avoid future accidental use.
@c @example
@c @group
@c @exdent Example:
@c 
@c @end group
@c @end example
@sp 1
See also:  @code{Dispose::Set, Dispose1::Set, DisposeAllNodes::Set}
@end deffn











@pdfsubsubsection {DeepDisposeAllNodes}
@deffn {DeepDisposeAllNodes} DeepDisposeAllNodes::Set
@sp 2
@example
@group
i = gDeepDisposeAllNodes(i);

object  i;
@end group
@end example
This method is used to remove all objects in a @code{Set} instance
without disposing of the set itself.  Each object or association
is disposed of via @code{DeepDispose}.

The value returned is always the instance passed.
@c @example
@c @group
@c @exdent Example:
@c 
@c @end group
@c @end example
@sp 1
See also:  @code{DisposeAllNodes::Set, Dispose::Set, DisposeAllNodes1::Set}
@end deffn





















@pdfsubsubsection {DeepDisposeGroup}
@deffn {DeepDisposeGroup} DeepDisposeGroup::Set
@sp 2
@example
@group
i = gDeepDisposeGroup(i, fun);

object  i;
int     (*fun)(object);
@end group
@end example
This method is used to remove and dispose of a group of objects from the
@code{Set} @code{i}.  The function, @code{fun}, is executed for each object in
the @code{Set}.  With each call @code{fun} is passed the object at that point
in the @code{Set}.  If @code{fun} returns a 1 that particular object will be
removed, or else if @code{fun} returns a 0 the object will not be
removed.  @code{fun} is used to decide which objects will be removed.

The objects removed are disposed of by use of the @code{DeepDispose} method.
@c @example
@c @group
@c @exdent Example:
@c 
@c @end group
@c @end example
@sp 1
See also:  @code{GroupRemove::Set, DisposeGroup::Set, DisposeAllNodes::Set}
@end deffn











@pdfsubsubsection {DeepDisposeObj}
@deffn {DeepDisposeObj} DeepDisposeObj::Set
@sp 2
@example
@group
r = gDeepDisposeObj(i, luk);

object  i;
object  luk;
object  r;
@end group
@end example
This method is used to remove and dispose of an object (@code{luk}) from
the @code{Set} instance (@code{i}).  If it is not found @code{NULL} is returned,
otherwise @code{i} is returned.  @code{luk} is disposed of by use of the
@code{DeepDispose} method.

@code{luk} may be any Dynace object.  The @code{gHash} and
@code{gCompare} generics are used to find and compare the various
objects in the @code{Set}.
@c @example
@c @group
@c @exdent Example:
@c 
@c @end group
@c @end example
@sp 1
See also:  @code{RemoveObj::Set, DisposeObj::Set}
@end deffn











@pdfsubsubsection {Dispose}
@deffn {Dispose} Dispose::Set
@sp 2
@example
@group
r = gDispose(i);

object  i;
object  r;     /*  NULL  */
@end group
@end example
This method is used to dispose of an entire @code{Set} instance.
It does not dispose of any of the objects or associations in the @code{Set}.

The value returned is always @code{NULL} and may be used to null out
the variable which contained the object being disposed in order to
avoid future accidental use.
@c @example
@c @group
@c @exdent Example:
@c 
@c @end group
@c @end example
@sp 1
See also:  @code{DeepDispose::Set, Dispose1::Set, DisposeAllNodes::Set}
@end deffn










@pdfsubsubsection {Dispose1}
@deffn {Dispose1} Dispose1::Set
@sp 2
@example
@group
gDispose1(i);

object  i;
@end group
@end example
This method is used to dispose of an entire @code{Set} instance.  It also
disposes of all of the objects or associations in the @code{Set} by the use of
the @code{Dispose} method applied to each object.  There is no value
returned.
@c @example
@c @group
@c @exdent Example:
@c 
@c @end group
@c @end example
@sp 1
See also:  @code{DeepDispose::Set, Dispose::Set, DisposeAllNodes1::Set}
@end deffn















@pdfsubsubsection {DisposeAllNodes}
@deffn {DisposeAllNodes} DisposeAllNodes::Set
@sp 2
@example
@group
i = gDisposeAllNodes(i);

object  i;
@end group
@end example
This method is used to remove all objects in a @code{Set} instance
without disposing of the set itself.  It does not dispose of any of the
objects or associations in the @code{Set}, they are simply
disassociated with the @code{Set}.

The value returned is always the instance passed.
@c @example
@c @group
@c @exdent Example:
@c 
@c @end group
@c @end example
@sp 1
See also:  @code{DeepDisposeAllNodes::Set, DisposeAllNodes1::Set,}
@hfil @break @hglue .64in      @code{Dispose::Set}
@end deffn


















@pdfsubsubsection {DisposeAllNodes1}
@deffn {DisposeAllNodes1} DisposeAllNodes1::Set
@sp 2
@example
@group
i = gDisposeAllNodes1(i);

object  i;
@end group
@end example
This method is used to remove all objects in a @code{Set} instance
without disposing of the set itself.  Each object or association in the
set are disposed of via @code{Dispose}.

The value returned is always the instance passed.
@c @example
@c @group
@c @exdent Example:
@c 
@c @end group
@c @end example
@sp 1
See also:  @code{DisposeAllNodes::Set, DeepDisposeAllNodes::Set,}
@hfil @break @hglue .64in      @code{Dispose::Set}
@end deffn














@pdfsubsubsection {DisposeGroup}
@deffn {DisposeGroup} DisposeGroup::Set
@sp 2
@example
@group
i = gDisposeGroup(i, fun);

object  i;
int     (*fun)(object);
@end group
@end example
This method is used to remove and dispose of a group of objects from the
@code{Set} @code{i}.  The function, @code{fun}, is executed for each object in
the @code{Set}.  With each call @code{fun} is passed the object at that point
in the @code{Set}.  If @code{fun} returns a 1 that particular object will be
removed, or else if @code{fun} returns a 0 the object will not be
removed.  @code{fun} is used to decide which objects will be removed.

The objects removed are disposed of by use of the @code{Dispose} method.
@c @example
@c @group
@c @exdent Example:
@c 
@c @end group
@c @end example
@sp 1
See also:  @code{GroupRemove::Set, DeepDisposeGroup::Set}
@end deffn













@pdfsubsubsection {DisposeObj}
@deffn {DisposeObj} DisposeObj::Set
@sp 2
@example
@group
r = gDisposeObj(i, luk);

object  i;
object  luk;
object  r;
@end group
@end example
This method is used to remove and dispose of an object (@code{luk}) from
the @code{Set} instance (@code{i}).  If it is not found @code{NULL} is returned,
otherwise @code{i} is returned.  @code{luk} is disposed of by use of the
@code{Dispose} method.

@code{luk} may be any Dynace object.  The @code{gHash} and
@code{gCompare} generics are used to find and compare the various
objects in the @code{Set}.
@c @example
@c @group
@c @exdent Example:
@c 
@c @end group
@c @end example
@sp 1
See also:  @code{RemoveObj::Set, DeepDisposeObj::Set}
@end deffn



















@pdfsubsubsection {Find}
@deffn {Find} Find::Set
@sp 2
@example
@group
r = gFind(i, luk);

object  i;
object  luk;
object  r;
@end group
@end example
This method is used to find an existing object (@code{luk}) in the
@code{Set} instance (@code{i}).  If the object is found in
the @code{Set} it will be returned otherwise @code{Find} will return @code{NULL}.

@code{luk} may be any Dynace object.  The @code{gHash} and
@code{gCompare} generics are used to find and compare the various
objects in the @code{Set}.
@c @example
@c @group
@c @exdent Example:
@c 
@c @end group
@c @end example
@sp 1
See also:  @code{Add::Set, FindAdd::Set, RemoveObj::Set}
@end deffn










@pdfsubsubsection {FindAdd}
@deffn {FindAdd} FindAdd::Set
@sp 2
@example
@group
r = gFindAdd(i, luk);

object  i;
object  luk;
object  r;
@end group
@end example
This method is used to find an existing object (@code{luk}) in the @code{Set}
instance (@code{i}) and return it.  If the object is not found it will
be added and returned.  

@code{luk} may be any Dynace object.  The @code{gHash} and
@code{gCompare} generics are used to find and compare the various
objects in the @code{Set}.
@c @example
@c @group
@c @exdent Example:
@c 
@c @end group
@c @end example
@sp 1
See also:  @code{Add::Set, Find::Set, RemoveObj::Set}
@end deffn



















@pdfsubsubsection {First}
@deffn {First} First::Set
@sp 2
@example
@group
r = gFirst(i);

object  i;
object  r;
@end group
@end example
This method returns one of the objects in the set.  The returned
object will be an arbitrary one in the set.  This method is useful
in cases where you want to enumerate through each object in the
set where you will perform some operation on the returned object and
then remove or dispose it from the set.  In that case the next call
to @code{gFirst} will simply return a different object from the set.

If no objects are left in the set, @code{gFirst} returns @code{NULL}.
@c @example
@c @group
@c @exdent Example:
@c 
@c @end group
@c @end example
@sp 1
See also:  @code{RemoveObj::Set, Sequence::Set}
@end deffn















@pdfsubsubsection {ForAll}
@deffn {ForAll} ForAll::Set
@sp 2
@example
@group
r = gForAll(i, fun);

object  i;
object  (*fun)(object);
object  r;
@end group
@end example
This method is used to execute function @code{fun} on each object in a @code{Set}.

Each time @code{fun} is called it is passed the next object in the @code{Set}
as its first and only argument.  If @code{fun} returns a non-NULL
object, @code{ForAll} returns immediately with the value returned by
@code{fun}.  As long as @code{fun} returns NULL it will be passed
successive objects in the @code{Set} until there are no more objects.  At
that time @code{ForAll} will just return NULL indicating that all
objects have been enumerated.
@c @example
@c @group
@c @exdent Example:
@c 
@c @end group
@c @end example
@sp 1
See also:  @code{Sequence::Set}
@end deffn



















@pdfsubsubsection {GroupRemove}
@deffn {GroupRemove} GroupRemove::Set
@sp 2
@example
@group
i = gGroupRemove(i, fun);

object  i;
int     (*fun)(object);
@end group
@end example
This method is used to remove a group of objects from the @code{Set}
@code{i}.  The function, @code{fun}, is executed for each
object in the @code{Set}.  With each call @code{fun} is passed the
object at that point in the @code{Set}.  If @code{fun} returns a 1
that particular object will be removed, or else if @code{fun}
returns a 0 the object will not be removed.  @code{fun} is used
to decide which objects will be removed.

The objects removed are not disposed.
@c @example
@c @group
@c @exdent Example:
@c 
@c @end group
@c @end example
@sp 1
See also:  @code{DisposeGroup::Set, DeepDisposeGroup::Set}
@end deffn













@pdfsubsubsection {RemoveObj}
@deffn {RemoveObj} RemoveObj::Set
@sp 2
@example
@group
r = gRemoveObj(i, luk);

object  i;
object  luk;
object  r;
@end group
@end example
This method is used to remove an object (@code{luk}) from the @code{Set}
instance (@code{i}) and return it.  If the object is not found
@code{NULL} will be returned.  @code{luk} is not disposed.

@code{luk} may be any Dynace object.  The @code{gHash} and
@code{gCompare} generics are used to find and compare the various
objects in the @code{Set}.
@c @example
@c @group
@c @exdent Example:
@c 
@c @end group
@c @end example
@sp 1
See also:  @code{DisposeObj::Set, DeepDisposeObj::Set}
@end deffn













@pdfsubsubsection {Resize}
@deffn {Resize} Resize::Set
@sp 2
@example
@group
r = gResize(i, size);

object  i;
int     size;
object  r;
@end group
@end example
This method is used to change the size of the hash table used to
implement the @code{Set}.  It may be used if during runtime it is
discovered that the number of objects stored or to be stored in the
@code{Set} are very different than the number used when the @code{Set}
was created.

Since a @code{Set} is implemented as a hash table it is best if
@code{size} is actually larger than the number of objects to be held.
It should be an odd number or better yet a prime number.  Keep in mind,
however, that there is storage requirements associated with this number
and the use of a very large number will consume large memory resources.
The @code{Set} will support more objects than defined by @code{size} at
a lesser efficiency.  In reality most reasonable numbers will work fine.

@code{r} is the re-sized @code{Set}.

This method is very time consuming and should only be used when
absolutely necessary.
@c @example
@c @group
@c @exdent Example:
@c 
@c @end group
@c @end example
@sp 1
See also:  @code{NewWithInt::Set}
@end deffn











@pdfsubsubsection {Sequence}
@deffn {Sequence} Sequence::Set
@sp 2
@example
@group
s = gSequence(i);

object  i;
object  s;
@end group
@end example
This method takes an instance of the @code{Set} class (@code{i})
and returns an instance of the @code{SetSequence} class
(@code{s}).  The @code{Set} represented by @code{i} is not effected by
this operation.  The @code{Set} sequence item, @code{s}, is used to enumerate
through all the objects in @code{Set} @code{i}.  See the
@code{SetSequence} class for documentation of other methods
needed to use @code{s}.
@example
@group
@exdent Example:

object  set; /*  Set           */
object  s;   /*  Set sequence  */
object  obj; /*  an object     */

/*  set must be initialized previously  */

for (s=gSequence(set) ; obj = gNext(s) ; )  @{
        /*  do something with obj  */
@}
@end group
@end example
@sp 1
See also:  @code{Next::SetSequence, First::Set}
@end deffn




















@pdfsubsubsection {Size}
@deffn {Size} Size::Set
@sp 2
@example
@group
sz = gSize(i);

object  i;
int     sz;
@end group
@end example
This method is used to obtain the number of objects in the @code{Set}.
@c @example
@c @group
@c @exdent Example:
@c 
@c @end group
@c @end example
@c @sp 1
@c See also:  @code{}
@end deffn










@pdfsubsubsection {StringRep}
@deffn {StringRep} StringRep::Set
@sp 2
@example
@group
s = gStringRep(i);

object  i;
object  s;
@end group
@end example
This method is used to generate an instance of the @code{String} class
which represents the object @code{i} and its value.  This is often used
to print or display a representation of an object.  It is also used by
by @code{Print::Object} (a method useful during the debugging phase of
a project) in order to directly print an object to a stream.

This method obtains the values of each object by calling
@code{StringRepValue} on each element.
@c @example
@c @group
@c @exdent Example:
@c 
@c @end group
@c @end example
@sp 1
See also:  @code{Print::Object, PrintValue::Object}
@end deffn












