@page

@section LinkObject Class
The @code{LinkObject} class is a subclass of the @code{LinkList} class.
Since this class inherits much of its functionality from the
@code{LinkList} and @code{Link} classes, see those classes for
additional functionality.  It is used in conjunction with the
@code{LinkValue} class in order to manage arbitrary objects on a doubly
linked list data structure.

The difference between this class and the @code{LinkList} class is that
the @code{LinkList} class deals with instances of the @code{Link} class
(or one of its subclasses) which may contain arbitrary objects and the
@code{LinkObject} class deals with arbitrary objects directly.  There is
no need to deal with link objects at all.  While this simplifies the
usage of a linked lists quite a bit it comes at the expense of
flexibility.  However, since the @code{LinkList} class is a superclass
of the @code{LinkObject} class you can often use the inherited
functionality of the @code{LinkList} class in conjunction with the
@code{LinkObject} object.  It is, therefore, recommended that the
@code{LinkObject} class be used whenever possible to simplify linked
list operations

This class (@code{LinkObject}) inherits much of its functionality from its
superclass (@code{LinkList}), therefore, in addition to the specific
documentation on this class please also refer to the documentation of
its superclass.

See the examples included with the Dynace system for an illustration of the
use of the doubly linked list related classes.



@subsection LinkObject Class Methods
The @code{LinkObject} class has no specific class methods.  It inherits
all of its abilities from its superclass, @code{LinkList}.


@subsection LinkObject Instance Methods
The instance methods associated with this class perform functions to
add, inquire and enumerate objects associated with the linked list.
Since this class is a subclass of the @code{LinkList} class, additional
functionality is inherited from it.  The @code{LinkList} class
documentation should, therefore, be consulted.






@deffn {AddFirst} AddFirst::LinkObject
@sp 2
@example
@group
i = gAddFirst(i, obj);

object  i;
object  obj;
@end group
@end example
This method is used to add a new link to the beginning of the list,
@code{i}, containing the arbitrary object @code{obj}.  The list object,
@code{i}, is returned.
@example
@group
@exdent Example:

object  ll, obj;

ll = gNew(LinkObject);
obj = gNewWithInt(ShortInteger, 44);
gAddFirst(ll, obj);
@end group
@end example
@sp 1
See also:  @code{AddLast::LinkObject, First::LinkObject}
@end deffn







@deffn {AddLast} AddLast::LinkObject
@sp 2
@example
@group
i = gAddLast(i, obj);

object  i;
object  obj;
@end group
@end example
This method is used to add a new link to the end of the list,
@code{i}, containing the arbitrary object @code{obj}.  The list object,
@code{i}, is returned.
@example
@group
@exdent Example:

object  ll, obj;

ll = gNew(LinkObject);
obj = gNewWithInt(ShortInteger, 44);
gAddLast(ll, obj);
@end group
@end example
@sp 1
See also:  @code{AddFirst::LinkObject, Last::LinkObject}
@end deffn











@deffn {First} First::LinkObject
@sp 2
@example
@group
obj = gFirst(i);

object  i;
object  obj;
@end group
@end example
This method is used to obtain the object stored in the first link of the
list @code{i}.  If there are no links on the list this method will
return @code{NULL}.  The list, @code{i}, is not effected by this operation.
@example
@group
@exdent Example:

object  ll, obj, obj2;

ll = gNew(LinkObject);
obj = gNewWithInt(ShortInteger, 44);
gAddFirst(ll, obj);
obj2 = gFirst(ll);
/* obj and obj2 are the same object  */
@end group
@end example
@sp 1
See also:  @code{AddFirst::LinkObject, Last::LinkObject}
@end deffn













@deffn {GetValues} GetValues::LinkObject
@sp 2
@example
@group
ll = vGetValues(ll, ...);

object  ll;
@end group
@end example
This method is used to extract numerous values from a @code{LinkObject} instance.
It does this without modifing the list.  Each argument must be a pointer to a
variable which is to hold one of the values.  The first variable gets the first
object on the list and the second gets the second, etc.  The argument list must
end in @code{NULL} to signify the end.

If there are more variables than links on the list the remaining variables will be
set to @code{NULL}.  If there are more elements on the list than variables the
remiaining list elements are ignored.

The @code{LinkObject} instance passed is returned unaltered.
@example
@group
@exdent Example:

object  ll, obj1, obj2;
object  obj3, obj4;

obj1 = gNewWithInt(ShortInteger, 44);
obj2 = gNewWithInt(ShortInteger, 66);
ll = vMakeList(LinkObject, obj1, obj2, NULL);
vGetValues(ll, &obj3, &obj4, NULL);
@end group
@end example
@sp 1
See also:  @code{MakeList::LinkObject, Nth::LinkObject}
@end deffn











@deffn {Last} Last::LinkObject
@sp 2
@example
@group
obj = gLast(i);

object  i;
object  obj;
@end group
@end example
This method is used to obtain the object stored in the last link of the
list @code{i}.  If there are no links on the list this method will
return @code{NULL}.  The list, @code{i}, is not effected by this operation.
@example
@group
@exdent Example:

object  ll, obj, obj2;

ll = gNew(LinkObject);
obj = gNewWithInt(ShortInteger, 44);
gAddLast(ll, obj);
obj2 = gLast(ll);
/* obj and obj2 are the same object  */
@end group
@end example
@sp 1
See also:  @code{AddLast::LinkObject, First::LinkObject}
@end deffn











@deffn {MakeList} MakeList::LinkObject
@sp 2
@example
@group
obj = vMakeList(LinkObject, ...);

object  obj;
@end group
@end example
This method is used to create a new @code{LinkObject} containing all the
objects listed in the method call.  For example, if there are five objects listed, the
returned @code{LinkObject} will be a list containing five links, each with
an object associated with it.

Since this is a variable argument method you must tell the system when there are no
more arguments by using a last argument of @code{NULL}.  @code{GetValues} can be used to
take the list back apart.  All the other @code{LinkObject} methods may be
used as well.  The value returned is the new
@code{LinkObject} instance.
@example
@group
@exdent Example:

object  ll, obj1, obj2;

obj1 = gNewWithInt(ShortInteger, 44);
obj2 = gNewWithInt(ShortInteger, 66);
ll = vMakeList(LinkObject, obj1, obj2, NULL);
@end group
@end example
@sp 1
See also:  @code{GetValues::LinkObject}
@end deffn









@deffn {Nth} Nth::LinkObject
@sp 2
@xref{Nth,,Nth::Link}
@end deffn
















@deffn {Pop} Pop::LinkObject
@sp 2
@example
@group
obj = gPop(lst);

object  lst;
object  obj;
@end group
@end example
This method is used to obtain the first element of a list while
removing it from the list at the same time.  It works in conjunction
with @code{gPush} to utilize a list as a stack.  @code{obj} is either
the first object on the list or @code{NULL} if the list was empty.
@example
@group
@exdent Example:

object  ll, obj, obj2;

ll = gNew(LinkObject);
obj = gNewWithInt(ShortInteger, 44);
gPush(ll, obj);
obj2 = gPop(ll);
@end group
@end example
@sp 1
See also:  @code{Push::LinkObject}
@end deffn










@deffn {Push} Push::LinkObject
@sp 2
@example
@group
lst = gPush(lst, obj);

object  lst;
object  obj;
@end group
@end example
This method is used to utilize a linklist as a stack.  Pushing is the
same as adding a new link to the beginning of the list.  Object
@code{obj} is added (pushed) on list @code{lst}.  The list object,
@code{lst}, is returned.  This method is the same as @code{gAddFirst}.
@example
@group
@exdent Example:

object  ll, obj;

ll = gNew(LinkObject);
obj = gNewWithInt(ShortInteger, 44);
gPush(ll, obj);
@end group
@end example
@sp 1
See also:  @code{Pop::LinkObject}
@end deffn









@deffn {Sequence} Sequence::LinkObject
@sp 2
@example
@group
s = gSequence(i);

object  i;
object  s;
@end group
@end example
This method takes an instance of the @code{LinkObject} class (@code{i})
and returns an instance of the @code{LinkObjectSequence} class
(@code{s}).  The link list represented by @code{i} is not effected by
this operation.  The link sequence item, @code{s}, is used to enumerate
through all the objects in link list @code{i}.  See the
@code{LinkObjectSequence} class for documentation of other methods
needed to use @code{s}.

The difference between the @code{Sequence} and @code{SequenceLinks}
methods in this class is that the @code{Sequence} method provides a
means to enumerate over the objects in the linked list and the
@code{SequenceLinks} provides a means to enumerate over the links
which hold the objects in the list.
@example
@group
@exdent Example:

object  ll;  /*  linked list           */
object  s;   /*  link object sequence  */
object  obj; /*  an object             */

/*  ll must be initialized previously  */

for (s=gSequence(ll) ; obj = gNext(s) ; )  @{
        /*  do something with obj  */
@}
@end group
@end example
@sp 1
See also:  @code{Next::LinkObjectSequence, SequenceLinks::LinkObject}
@end deffn














@deffn {SequenceLinks} SequenceLinks::LinkObject
@sp 2
@example
@group
s = gSequenceLinks(i);

object  i;
object  s;
@end group
@end example
This method takes an instance of the @code{LinkObject} class (@code{i})
and returns an instance of the @code{LinkSequence} class
(@code{s}).  The link list represented by @code{i} is not effected by
this operation.  The link sequence item, @code{s}, is used to enumerate
through all the links in link list @code{i}.  See the
@code{LinkSequence} class for documentation of other methods
needed to use @code{s}.

The difference between the @code{Sequence} and @code{SequenceLinks}
methods in this class is that the @code{Sequence} method provides a
means to enumerate over the objects in the linked list and the
@code{SequenceLinks} provides a means to enumerate over the links
which hold the objects in the list.
@example
@group
@exdent Example:

object  ll;  /*  linked list    */
object  s;   /*  link sequence  */
object  lnk; /*  a link         */

/*  ll must be initialized previously  */

for (s=gSequenceLinks(ll) ; lnk = gNext(s) ; )  @{
        /*  do something with link lnk  */
@}
@end group
@end example
@sp 1
See also:  @code{Next::LinkSequence, Sequence::LinkObject}
@end deffn






