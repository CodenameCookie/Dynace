@page

@section  BitArray Class
This class, which is a subclass of @code{Array}, is used to represent
arbitrary shaped arrays of on/off--yes/no information in an efficient
manner.  Much of the functionality of this class is implemented and
documented in the @code{Array} class.  Differences are documented in
this section.



@subsection BitArray Class Methods
The only class method implemented by this class is one used to create
new @code{BitArray} instances.






@deffn {New} New::BitArray
@sp 2
@example
@group
ary = vNew(BitArray, rnk, ...)

unsigned  rnk, ...
object    ary;
@end group
@end example
This class method is used to create a new bit array.

@code{rnk} is the number of dimensions the new array should have.
The remaining arguments (each of type unsigned) indicates the size of
each consecutive dimension.  Note that the number of arguments following
@code{rnk} @emph{must} be the same as the value in @code{rnk}.

@code{ary} is the new array object created and will be initialized to
all zeros.
@example
@group
@exdent Example:

object  ary;

ary = vNew(BitArray, 2, 5, 4);
/*  ary is a 5x4 matrix  */
@end group
@end example
@c @sp 1
@c See also:  @code{}
@end deffn



@subsection BitArray Instance Methods
The @code{BitArray} instance methods are implemented in the @code{Array} class
for efficiency reasons but those methods which are particular to the
@code{BitArray} are documented here.








@deffn {BitValue} BitValue::BitArray
@sp 2
@example
@group
v = vBitValue(ary, ...);

object    ary;
unsigned  ...
int       v;
@end group
@end example
This method is used to obtain the bit (1 or 0) value associated with a
particular element of an instance of the @code{BitArray} class.

The arguments after the @code{ary} argument (each an @code{unsigned})
are used to specify the exact index into each consecutive dimension of
the array.  The number of arguments after the @code{ary} argument
@emph{must} be equal to the number of dimensions (or rank) of array
@code{ary}.  See @code{IndexOrigin::Array} for more information.

Note that this is one of the few generics which doesn't return a Dynace
object.  It returns an @code{int}.
@example
@group
@exdent Example:

object  ary;
int     v;

ary = vNew(BitArray, 2, 5, 4);
v = vBitValue(ary, 1, 2);
/*  v has ary[1][2]  */
@end group
@end example
@sp 1
See also:  @code{ChangeBitValue::BitArray}
@end deffn








@deffn {ChangeBitValue} ChangeBitValue::BitArray
@sp 2
@example
@group
ary = vChangeBitValue(ary, val, ...);

object    ary;
int       val;
unsigned  ...
@end group
@end example
This method is used to change the value of one element of
@code{BitArray} @code{ary}.

@code{val} is the value which the element of the array should be changed
to (a 1 or 0).

The arguments after the @code{val} argument (each an @code{unsigned})
are used to specify the exact index into each consecutive dimension of
the array.  The number of arguments after the @code{val} argument
@emph{must} be equal to the number of dimensions (or rank) of array
@code{ary}.  See @code{IndexOrigin::Array} for more information.

The value returned is the modified array passed.
@example
@group
@exdent Example:

object  ary;

ary = vNew(BitArray, 2, 5, 4);
vChangeBitValue(ary, 1, 1, 2);
/*  ary[1][2] <- 1  */
@end group
@end example
@sp 1
See also:  @code{BitValue::BitArray}
@end deffn








