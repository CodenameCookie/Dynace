@page

@section Thread Class
The @code{Thread} class gives Dynace the ability to create, manage and
execute multiple threads of execution in a timesharing fashion.  See the
manual for a detailed description of threads.


See the examples included with the Dynace system for an illustration of the
use of the thread / pipe / semaphore related classes.


@subsection Thread Class Methods
The class methods associated with the @code{Thread} class provide a means
to create new instances of the @code{Thread} class.  Each instance
represents a unique thread of execution.  The class methods also provide
other functionality which is not related to any particular thread, such
as the ability to initialize the threading system.










@deffn {BlockingGetkey} BlockingGetkey::Thread
@sp 2
@example
@group
c = gBlockingGetkey(Thread)

int     c;
@end group
@end example
This method is used to get a key from the keyboard.  It does it in a way
that allows other threads to continue if there is no key ready.  This
allows the system to continue to operate while waiting for a user entry.

The thread which calls this method will be placed on hold until a key
has been hit.  At that time the thread will be enabled and the method
will return the key hit.

Only one thread may use this method at a time.
@c @example
@c @group
@c @exdent Example:
@c 
@c @end group
@c @end example
@sp 1
See also:  @code{StartThreader}
@end deffn











@deffn {FindStr} FindStr::Thread
@sp 2
@example
@group
t = gFindStr(Thread, name)

char    *name;
object  t;
@end group
@end example
This method is used to obtain a thread object from its associated name.
If found it is returned, otherwise @code{NULL} is returned.  Note
that the name of the thread which started the thread system is ``main''.
If @code{name} is @code{NULL} @code{Find} returns the currently running
thread object.
@c @example
@c @group
@c @exdent Example:
@c 
@c @end group
@c @end example
@sp 1
See also:  @code{NewThread::Thread}
@end deffn









@deffn {NewThread} NewThread::Thread
@sp 2
@example
@group
t = gNewThread(Thread, name, fun, priority, arg, run,
               autoDispose)

char    *name;
int     (*fun)(void *);
int     priority;
void    *arg;
int     run;
int     autoDispose;
object  t;
@end group
@end example
This class method creates instances of the @code{Thread} class.  Each
thread represents a unique thread of execution.

@code{name} is the name of the thread.  This name may be used to access
the thread object via its name at a future point.  @code{name} may
be null if no particular name is desired.

@code{fun} is the C language function which will implement the process
to be run by the new thread.  When @code{fun} is first executed it is
passed the pointer defined by @code{arg} as its only argument.  @code{fun}
may return an integer result which may be accessed by other threads.

@code{priority} sets the initial priority of the thread.  This is a number
between 1 and 30000.  The higher the number the greater the priority.
Threads of higher priority (which are ready to run) always run before
any threads of lesser priority.  Only ready threads of the same priority
share CPU time.  There is a macro named @code{DEFAULT_PRIORITY} which
may be used to create threads which share resources equally.  Creating
higher or lower priority threads may be accomplished relative to the
@code{DEFAULT_PRIORITY} macro.  Note that @emph{all} ready threads of
a higher priority must complete before threads of a lower priority will
receive any CPU time.

@code{arg} provides a simple mechanism to pass data to the thread when
it starts.  When @code{fun} is started it is passed a single argument
@code{arg}.

@code{run} is a flag to tell the system how to initialize the thread.
If @code{run} is 1 the new thread will be created in a ready-to-run
state.  It will be placed on the queue of ready threads and will
automatically start when it gets CPU time.  If the thread is started
at a higher priority than the thread which created the new thread
then the current thread will be timed-out and the new thread will
immediately start.  If @code{run} is set to 0 the new thread will
be created in a hold state and must be released from this state
in order to start running.

@code{autoDispose} is used to control what happens to the thread
object once the thread completes.  If set to 1 the thread will be
automatically disposed when it completes.  If set to 0 the thread
will remain in the system until manually disposed.  This may
be useful if the return value of the thread is desired after it
has completed.  Note that the garbage collector never collects
thread objects.  They must be handled manually.  Note also that
the way the system was implemented completed threads and threads
on hold take up 0 CPU resources.

The value returned (@code{t}) is the thread object created.  If not
kept it may be obtained later by the use of the @code{FindStr::Thread}
method.  This thread object may be used for many purposes to control
a thread.  You may hold, release, change priority, kill or otherwise
query the state of the thread via this object.

Note that the threader must be started prior to the creation of any new
threads.  This may be accomplished by the use of the @code{StartThreader}
method.  
@c @example
@c @group
@c @exdent Example:
@c 
@c @end group
@c @end example
@sp 1
See also:  @code{StartThreader, FindStr::Thread}
@end deffn



@subsection Thread Instance Methods
The instance methods associated with this class are used to manage and
query various aspects related to particular threads.  Each instance
represents a single unique thread.





@deffn {DeepDispose} DeepDispose::Thread
@sp 2
@example
@group
r = gDeepDispose(t)

object  t;
object  r;     /*  NULL  */
@end group
@end example
This method is used to permanently stop the execution of an arbitrary
thread, @code{t} (if it is running) and dispose of it.  @code{t} may be
the current thread, in which case the next thread on the queue will
start.

In order to give the thread a return value use @code{Kill::Thread}
and then this method.

This method is the same as @code{Dispose::Thread}.
@c @example
@c @group
@c @exdent Example:
@c 
@c @end group
@c @end example
@sp 1
See also:  @code{Kill::Thread, Hold::Thread}
@end deffn







@deffn {Dispose} Dispose::Thread
@sp 2
@example
@group
r = gDispose(t)

object  t;
object  r;     /*  NULL  */
@end group
@end example
This method is used to permanently stop the execution of an arbitrary
thread, @code{t} (if it is running) and dispose of it.  @code{t} may be
the current thread, in which case the next thread on the queue will
start.

In order to give the thread a return value use @code{Kill::Thread}
and then this method.

This method is the same as @code{DeepDispose::Thread}.

The value returned is always @code{NULL} and may be used to null out
the variable which contained the object being disposed in order to
avoid future accidental use.
@c @example
@c @group
@c @exdent Example:
@c 
@c @end group
@c @end example
@sp 1
See also:  @code{Kill::Thread, Hold::Thread}
@end deffn













@deffn {ChangePriority} ChangePriority::Thread
@sp 2
@example
@group
t = gChangePriority(t, pri)

object  t;
int     pri;
@end group
@end example
This method is used to change the priority of thread @code{t}.
Thread @code{t} may be the currently running thread.  If thread
@code{t} is changed to a higher priority than the currently running
thread, the currently running will immediately release the CPU to thread
@code{t}.

The priority is a number between 1 and 30000.  The higher the number the
greater the priority.  Threads of higher priority (which are ready to
run) always run before any threads of lesser priority.  Only ready
threads of the same priority share CPU time.  There is a macro named
@code{DEFAULT_PRIORITY} which may be used to create threads which share
resources equally.  Creating higher or lower priority threads may be
accomplished relative to the @code{DEFAULT_PRIORITY} macro.  Note that
@emph{all} ready threads of a higher priority must complete before
threads of a lower priority will receive any CPU time.
@c @example
@c @group
@c @exdent Example:
@c 
@c @end group
@c @end example
@sp 1
See also:  @code{Priority::Thread, NewThread::Thread, State::Thread}
@end deffn











@deffn {Hold} Hold::Thread
@sp 2
@example
@group
t = gHold(t)

object  t;
@end group
@end example
This method is used to place a temporary hold on a thread (@code{t}).
The thread stops execution and waits until released.  Threads on hold
take no CPU time.  Any number of threads may place any number of holds
on any threads, however, an equal number of releases are required
to release a thread which has been held multiple times.  A thread
may put itself on hold.  The value returned is the thread passed.
@c @example
@c @group
@c @exdent Example:
@c 
@c @end group
@c @end example
@sp 1
See also:  @code{Release::Thread, Kill::Thread}
@end deffn











@deffn {IntValue} IntValue::Thread
@sp 2
@example
@group
r = gIntValue(t)

object  t;
int     r;
@end group
@end example
This method is used to obtain the return value of a thread (@code{t})
which has completed or been killed.  If the thread is still running
(or on hold) this method will just return 0.
@c @example
@c @group
@c @exdent Example:
@c 
@c @end group
@c @end example
@sp 1
See also:  @code{State::Thread, WaitFor::Thread, Kill::Thread}
@end deffn















@deffn {Kill} Kill::Thread
@sp 2
@example
@group
t = gKill(t, rtn)

object  t;
int     rtn;
@end group
@end example
This method is used to permanently stop the execution of an arbitrary
thread, @code{t}.  Once this method is called on a thread it cannot
be run again.  @code{t} may be the current thread, in which case the
next thread on the queue will start.  Thread @code{t} is not disposed.
Attempting to @code{Kill} a thread which has already been killed or
completed has no ill effect, @code{rtn} is just ignored.

@code{rtn} represents the return value of the thread, which may be
queried by other threads.  Any holds associated with the thread
will be expunged.  Thread @code{t} is returned.

Note that killed threads take up no CPU time.
@c @example
@c @group
@c @exdent Example:
@c 
@c @end group
@c @end example
@sp 1
See also:  @code{Dispose::Thread, Hold::Thread}
@end deffn








@deffn {Name} Name::Thread
@sp 2
@example
@group
nam = gName(t)

object  t;
char    *nam;
@end group
@end example
This method is used to obtain the name of thread @code{t}.
@c @example
@c @group
@c @exdent Example:
@c 
@c @end group
@c @end example
@sp 1
See also:  @code{NewThread::Thread, State::Thread, Priority::Thread}
@end deffn












@deffn {Priority} Priority::Thread
@sp 2
@example
@group
pri = gPriority(t)

object  t;
int     pri;
@end group
@end example
This method is used to obtain the priority of thread @code{t}.

The priority is a number between 1 and 30000.  The higher the number the
greater the priority.  Threads of higher priority (which are ready to
run) always run before any threads of lesser priority.  Only ready
threads of the same priority share CPU time.  There is a macro named
@code{DEFAULT_PRIORITY} which may be used to create threads which share
resources equally.  Creating higher or lower priority threads may be
accomplished relative to the @code{DEFAULT_PRIORITY} macro.  Note that
@emph{all} ready threads of a higher priority must complete before
threads of a lower priority will receive any CPU time.
@c @example
@c @group
@c @exdent Example:
@c 
@c @end group
@c @end example
@sp 1
See also:  @code{ChangePriority::Thread, NewThread::Thread, State::Thread}
@end deffn









@deffn {Release} Release::Thread
@sp 2
@example
@group
t = gRelease(t, yld)

object  t;
int     yld;
@end group
@end example
This method is used to release a hold placed by the @code{Hold}
method on a thread (@code{t}).  The thread will not really be released
until the number of calls to @code{Release} is equal to the number
of times @code{Hold} was called on thread @code{t}.

@code{yld} is a flag.  It indicates whether the current thread should
yield to the thread being released (if the thread being released has a
higher priority).  If @code{yld} is 1 and the thread being released has
a higher priority than the current thread the system will immediately
transfer the CPU to the thread being released.  If @code{yld} is set to
0 the current thread will be allowed to complete its time slice.  The
value returned is the thread passed.
@c @example
@c @group
@c @exdent Example:
@c 
@c @end group
@c @end example
@sp 1
See also:  @code{Hold::Thread}
@end deffn







@deffn {State} State::Thread
@sp 2
@example
@group
s = gState(t)

object  t;
int     s;
@end group
@end example
This method is used to obtain the current state of a thread (@code{t}).
The valid states are macros defined in @code{dynl.h} and are defined as
follows:
@example
@group
NEW_THREAD             New thread which hasn't started running
                       yet
RUNNING_THREAD         Thread ready to run
HOLD_THREAD            Thread placed on hold
DONE_THREAD            Thread has completed or been killed
WAITING_FOR_THREAD     Thread waiting for another thread to
                       complete
WAITING_FOR_SEMAPHORE  Thread waiting for semaphore
@end group
@end example
@c @sp 1
@c See also:  @code{}
@end deffn
















@deffn {WaitFor} WaitFor::Thread
@sp 2
@example
@group
r = gWaitFor(t)

object  t;
int     r;
@end group
@end example
This method is used to place the current thread on hold until thread
@code{t} completes, gets killed or disposed.  At that point the thread
gets released and @code{WaitFor} returns the return value from thread
@code{t}.  Note that threads waiting for other threads take up no CPU time.
@c @example
@c @group
@c @exdent Example:
@c 
@c @end group
@c @end example
@sp 1
See also:  @code{Kill::Thread, Dispose::Thread, Hold::Thread}
@end deffn







@subsection Thread Macros
The thread system includes a few macros which are used to provide
some additional functionality without incurring much system overhead.

















@deffn {ENABLE_THREADER} ENABLE_THREADER
@sp 2
@example
@group
ENABLE_THREADER;
@end group
@end example
This macro is used to re-enable the threader after the
@code{INHIBIT_THREADER} has been used.  Since consecutive calls to
@code{INHIBIT_THREADER} stack, an equal number of @code{ENABLE_THREADER}
calls must be used.

The two macros, @code{INHIBIT_THREADER} and @code{ENABLE_THREADER},
are used to surround code which you want to be absolutely certain
that a context switch doesn't occur during.  Since these macros nest
it is ok if between a pair a function is called which also uses these
macros.

Note that when the threader is enabled a context switch can only occur
during a generic function call or by the use of the @code{YIELD}
macro.
@example
@group
@exdent Example:

INHIBIT_THREADER;
       .
  (protected code)
       .
ENABLE_THREADER;
@end group
@end example
@sp 1
See also:  @code{INHIBIT_THREADER, StartThreader}
@end deffn


















@deffn {INHIBIT_THREADER} INHIBIT_THREADER
@sp 2
@example
@group
INHIBIT_THREADER;
@end group
@end example
This macro is used to temporarily disable the threader.  Once disabled
no context switches will occur until the threader is re-enabled by the
use of the @code{ENABLE_THREADER} macro.  Consecutive calls to
@code{INHIBIT_THREADER} stack, therefore, an equal number of
@code{ENABLE_THREADER} calls must be used to re-enable the threader.

The two macros, @code{INHIBIT_THREADER} and @code{ENABLE_THREADER},
are used to surround code which you want to be absolutely certain
that a context switch doesn't occur during.  Since these macros nest
it is ok if between a pair a function is called which also uses these
macros.

Note that when the threader is enabled a context switch can only occur
during a generic function call or by the use of the @code{YIELD}
macro.
@example
@group
@exdent Example:

INHIBIT_THREADER;
       .
  (protected code)
       .
ENABLE_THREADER;
@end group
@end example
@sp 1
See also:  @code{ENABLE_THREADER, StartThreader}
@end deffn








@deffn {StartThreader} StartThreader
@sp 2
@example
@group
StartThreader(argc)

int     argc;
@end group
@end example
This macro is used to initialize the threading system.  It must be
called before any other thread operations (except for initializing the
thread class).  It @emph{must} be called from the C function called
@code{main}, and should only be called once.  It is typically
called immediately following the Dynace system initialization.

Once this macro is called all execution is performed by threads.  The
program (@code{main}) which called @code{StartThreader} becomes the
single running thread with the name ``main''.  Any future threads
created at the default priority will share CPU time equally with the
main thread.  There is nothing special about the main thread.

@code{argc} is simply the argument normally passed to C language
@code{main} functions.  The value doesn't matter.  It is used to
obtain the address of the stack.
@c @example
@c @group
@c @exdent Example:
@c 
@c @end group
@c @end example
@sp 1
See also:  @code{NewThread::Thread}
@end deffn







@deffn {YIELD} YIELD
@sp 2
@example
@group
YIELD;
@end group
@end example
The @code{YIELD} macro is used to force a context switch (unless the
threader has been disabled with the @code{INHIBIT_THREADER} macro).

Since Dynace only provides automatic context switching during a generic
function call, if a thread is running a long compute cycle without
any generic function calls it will hog the system until a generic is
called unless the @code{YIELD} macro is used.  This macro may be placed
at regular intervals in the code to be sure that it shares the CPU
time.  The @code{YIELD} macro will only cause an actual context
switch when the threads time is complete, otherwise @code{YIELD}
will do nothing.
@example
@group
@exdent Example:

while (some condition)  @{
        do something;
        YIELD;
@}
@end group
@end example
@sp 1
See also:  @code{StartThreader}
@end deffn










