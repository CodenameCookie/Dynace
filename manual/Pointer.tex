@page

@section Pointer Class
The @code{Pointer} class is used to represent the C language
@code{void *} data type as a Dynace object.





@subsection Pointer Class Methods
The Pointer class has only one class method and it is used to create
new instances of itself.



@deffn {NewWithPtr} NewWithPtr::Pointer
@sp 2
@example
@group
i = gNewWithPtr(Pointer, val);

void    *val;
object  i;
@end group
@end example
This class method creates instances of the Pointer class.  @code{val}
is the initial value of the pointer being represented.  

The value returned is a Dynace instance object which represents the pointer
passed.

Note that the default disposal methods are used by this class since
there are no special storage allocation requirements.
@example
@group
@exdent Example:

object  x;
int     v = 3;

x = gNewWithPtr(Pointer, &v);
@end group
@end example
@sp 1
See also:  @code{PointerValue::Pointer, Dispose::Object}
@end deffn





@subsection Pointer Instance Methods
The instance methods associated with the Pointer class provide a
means of changing and obtaining the value associated with an instance of
the Pointer class.  Methods are also included to help the generic
container classes to quickly access members of this class.






@deffn {ChangeValue} ChangeValue::Pointer
@sp 2
@example
@group
i = gChangeValue(i, val);

object  i;
void    *val;
@end group
@end example
This method is used to change the value associated with an instance of
the Pointer class.  Notice that this method returns the instance
being passed.  @code{val} is the new value.
@example
@group
@exdent Example:

object  x;
int     v1 = 4, v2 = 5;

x = gNewWithPtr(Pointer, &v1);
gChangeValue(x, &v2);
@end group
@end example
@sp 1
See also:  @code{NewWithPtr::Pointer, PointerValue::Pointer}
@end deffn







@deffn {Compare} Compare::Pointer
@sp 2
@example
@group
r = gCompare(i, obj);

object  i;
object  obj;
int     r;
@end group
@end example
This method is used by the generic container classes to determine
the equality of the values represented by @code{i} and @code{obj}. 
@code{r} is -1 if the value represented by @code{i} is less than
the value represented by @code{obj}, 1 if the value of @code{i}
is greater than @code{obj}, and 0 if they are equal.
@c @example
@c @group
@c @exdent Example:
@c
@c @end group
@c @end example
@sp 1
See also:  @code{Hash::Pointer}
@end deffn







@deffn {Hash} Hash::Pointer
@sp 2
@example
@group
val = gHash(i);

object  i;
int     val;
@end group
@end example
This method is used by the generic container classes to obtain hash values
for the object.  @code{val}
is a hash value between 0 and a large integer value.
@c @example
@c @group
@c @exdent Example:
@c
@c @end group
@c @end example
@sp 1
See also:  @code{Compare::Pointer}
@end deffn







@deffn {PointerValue} PointerValue::Pointer
@sp 2
@example
@group
val = gPointerValue(i);

object  i;
void    *val;
@end group
@end example
This method is used to obtain the @code{void *} value associated with an
instance of the Pointer class.  Note that this is one of the few
generics which doesn't return a Dynace object.  It returns a void * pointer.
@example
@group
@exdent Example:

object  x;
int     a = 4, *b;
void    *val;

x = gNewWithPtr(Pointer, &a);
b = (int *) gPointerValue(x);
/*  b now points to a  */
@end group
@end example
@sp 1
See also:  @code{NewWithPtr::Pointer, ChangeValue::Pointer}
@end deffn








@deffn {StringRepValue} StringRepValue::Pointer
@sp 2
@example
@group
s = gStringRepValue(i);

object  i;
object  s;
@end group
@end example
This method is used to generate an instance of the @code{String} class
which represents the value associated with @code{i}.  This is often
used to print or display the value.  It is also used by
@code{PrintValue::Object} and indirectly by @code{Print::Object}
(two methods useful during the debugging phase of a project)
in order to directly print an object's value.
@example
@group
@exdent Example:

object  x;
object  s;

x = gNewWithPtr(Pointer, NULL);
s = gStringRepValue(x);
@end group
@end example
@sp 1
See also:  @code{PrintValue::Object, Print::Object}
@end deffn











