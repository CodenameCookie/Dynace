@page

@section IntegerDictionary Class
@pdfsection{IntegerDictionary Class}
This class combines the functionality of the @code{Set}
and @code{IntegerAssociation} classes in order to store a collection of
arbitrary key/value pairs.

The difference between this class and the @code{Dictionary} class is
that this class only accepts C language @code{int} values as keys.
This class provides an efficient and simple to use means of representing
a commonly needed collection.

This class is a subclass of the @code{Set} class and therefore inherits
all of its functionality.

See the examples included with the Dynace system for an illustration of the
use of the @code{Set}/@code{Dictionary} related classes.


@subsection IntegerDictionary Class Methods
@pdfsubsection{IntegerDictionary Class Methods}
There are no class methods for this class.  It inherits the ability
to create instances of itself through its superclass, @code{Set}.



@subsection IntegerDictionary Instance Methods
@pdfsubsection{IntegerDictionary Instance Methods}
The instance methods associated with this class are used to add, retrieve
and remove key/value pairs from the @code{Dictionary}.  Note that additional
functionality may be obtained through its superclass, @code{Set}.






@pdfsubsubsection {AddInt}
@deffn {AddInt} AddInt::IntegerDictionary
@sp 2
@example
@group
r = gAddInt(i, key, value);

object  i;
int     key
object  value;
object  r;
@end group
@end example
This method is used to add a new key/value pair to the @code{Dictionary}
instance (@code{i}).  If an object with the same key already exists in
the @code{Dictionary} it will be left as is and @code{AddInt} will return
@code{NULL}.  If the key/value objects are added @code{AddInt} will
return the @code{IntegerAssociation} instance created to represent the
key/value pair passed.
@c @example
@c @group
@c @exdent Example:
@c 
@c @end group
@c @end example
@sp 1
See also:  @code{FindInt::IntegerDictionary,}
@hfil @break @hglue .64in      @code{FindValueInt::IntegerDictionary,}
@hfil @break @hglue .64in      @code{ChangeValueWithInt::IntegerDictionary,}
@hfil @break @hglue .64in      @code{RemoveInt::IntegerDictionary}
@end deffn











@pdfsubsubsection {ChangeValueWithInt}
@deffn {ChangeValueWithInt} ChangeValueWithInt::IntegerDictionary
@sp 2
@example
@group
r = gChangeValueWithInt(i, key, value);

object  i;
int     key
object  value;
object  r;
@end group
@end example
This method is used to change the value associated with an existing
key/value pair to the @code{IntegerDictionary} instance (@code{i}).
@code{key} represents the identity of the pre-existing key/value
pair and @code{value} represents the new value to be associated with
the key.

Normally, this method changes the value part of the association
and returns the previous value which is not disposed.  If
@code{key} doesn't identify a pre-existing association, this method
simply returns @code{NULL}.
@c @example
@c @group
@c @exdent Example:
@c 
@c @end group
@c @end example
@sp 1
See also:  @code{FindInt::IntegerDictionary,}
@hfil @break @hglue .64in      @code{FindValueInt::IntegerDictionary,}
@hfil @break @hglue .64in      @code{RemoveInt::IntegerDictionary}
@end deffn











@pdfsubsubsection {DeepDisposeInt}
@deffn {DeepDisposeInt} DeepDisposeInt::IntegerDictionary
@sp 2
@example
@group
r = gDeepDisposeInt(i, key);

object  i;
int     key;
object  r;
@end group
@end example
This method is used to remove and dispose of a key/value pair from a
@code{Dictionary}.  If found and removed @code{i} is returned.  If @code{key}
is not found @code{NULL} is returned.

The value and @code{IntegerAssociation} used to bind the two are all deep
disposed.

This method is the same as @code{DisposeInt::IntegerDictionary}.
@c @example
@c @group
@c @exdent Example:
@c 
@c @end group
@c @end example
@sp 1
See also:  @code{RemoveInt::IntegerDictionary}
@end deffn













@pdfsubsubsection {Dispose}
@deffn {Dispose} Dispose::IntegerDictionary
@sp 2
@example
@group
r = gDispose(i);

object  i;
object  r;     /*  NULL  */
@end group
@end example
This method is used to dispose of an entire @code{Dictionary}.  It does not
dispose of any of the values but does dispose of all the associations
used to represent the pair.  

The value returned is always @code{NULL} and may be used to null out
the variable which contained the object being disposed in order to
avoid future accidental use.
@c @example
@c @group
@c @exdent Example:
@c 
@c @end group
@c @end example
@sp 1
See also:  @code{RemoveInt::IntegerDictionary, DeepDispose::Set,}
@hfil @break @hglue .64in      @code{DisposeAllNodes::IntegerDictionary}
@end deffn













@pdfsubsubsection {DisposeAllNodes}
@deffn {DisposeAllNodes} DisposeAllNodes::IntegerDictionary
@sp 2
@example
@group
i = gDisposeAllNodes(i);

object  i;
@end group
@end example
This method is used to remove all objects in an @code{IntegerDictionary}
instance without disposing of the instance itself.  The objects in the
@code{IntegerDictionary} are simply disassociated from the
@code{IntegerDictionary} instance and are not disposed.

The value returned is always the instance passed.
@c @example
@c @group
@c @exdent Example:
@c 
@c @end group
@c @end example
@sp 1
See also:  @code{DeepDisposeAllNodes::Set, Dispose::IntegerDictionary}
@end deffn




















@pdfsubsubsection {DisposeInt}
@deffn {DisposeInt} DisposeInt::IntegerDictionary
@sp 2
@example
@group
r = gDisposeInt(i, key);

object  i;
int     key;
object  r;
@end group
@end example
This method is used to remove and dispose of a key/value pair from a
@code{Dictionary}.  If found and removed @code{i} is returned.  If @code{key}
is not found @code{NULL} is returned.

The value and @code{IntegerAssociation} used to bind the two are all deep
disposed.

This method is the same as @code{DeepDisposeInt::IntegerDictionary}.
@c @example
@c @group
@c @exdent Example:
@c 
@c @end group
@c @end example
@sp 1
See also:  @code{RemoveInt::IntegerDictionary}
@end deffn













@pdfsubsubsection {FindInt}
@deffn {FindInt} FindInt::IntegerDictionary
@sp 2
@example
@group
r = gFindInt(i, key);

object  i;
int     key;
object  r;
@end group
@end example
This method is used to find the instance of the @code{IntegerAssociation}
class which is used to represent the key/value pair stored under
@code{key} in @code{Dictionary} @code{i}.  If @code{key} is not found
@code{NULL} is returned.
@c @example
@c @group
@c @exdent Example:
@c 
@c @end group
@c @end example
@sp 1
See also:  @code{FindValueInt::IntegerDictionary}
@end deffn










@pdfsubsubsection {FindAddInt}
@deffn {FindAddInt} FindAddInt::IntegerDictionary
@sp 2
@example
@group
r = gFindAddInt(i, key, value);

object  i;
int     key;
object  value;
object  r;
@end group
@end example
This method is used to find and return the instance of the
@code{IntegerAssociation} class used to represent the key/value pair
stored under the key @code{key}.  If it is not found a new
@code{IntegerAssociation} will be added and returned which represent the
key/value pair representing @code{key} and @code{value}.
@c @example
@c @group
@c @exdent Example:
@c 
@c @end group
@c @end example
@sp 1
See also:  @code{FindInt::IntegerDictionary,}
@hfil @break @hglue .64in      @code{FindValueInt::IntegerDictionary}
@end deffn









@pdfsubsubsection {FindValueInt}
@deffn {FindValueInt} FindValueInt::IntegerDictionary
@sp 2
@example
@group
r = gFindValueInt(i, key);

object  i;
int     key;
object  r;
@end group
@end example
This method is used to find the value stored under @code{key} in
@code{Dictionary} @code{i}.  If @code{key} is not found @code{NULL} is
returned.
@c @example
@c @group
@c @exdent Example:
@c 
@c @end group
@c @end example
@sp 1
See also:  @code{FindInt::IntegerDictionary}
@end deffn











@pdfsubsubsection {RemoveInt}
@deffn {RemoveInt} RemoveInt::IntegerDictionary
@sp 2
@example
@group
r = gRemoveInt(i, key);

object  i;
int     key;
object  r;
@end group
@end example
This method is used to remove a key/value pair from a @code{Dictionary}.
If found and removed @code{i} is returned.  If @code{key} is not found
@code{NULL} is returned.

The value is not disposed, however, the @code{IntegerAssociation} used to
bind the two is.
@c @example
@c @group
@c @exdent Example:
@c 
@c @end group
@c @end example
@sp 1
See also:  @code{DeepDisposeInt::IntegerDictionary}
@end deffn





