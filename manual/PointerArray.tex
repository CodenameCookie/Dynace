@page

@section  PointerArray Class
This class, which is a subclass of @code{Array}, is used to represent
arbitrary shaped arrays of arbitrary C language pointers in an efficient
manner.


Much of the functionality of this class is implemented and documented in
the @code{Array} class.  Differences are documented in this section.



@subsection PointerArray Class Methods
The only class method implemented by this class is one used to create
new @code{PointerArray} instances.






@deffn {New} New::PointerArray
@sp 2
@example
@group
ary = vNew(PointerArray, rnk, ...)

unsigned  rnk, ...
object    ary;
@end group
@end example
This class method is used to create a new instance of @code{PointerArray}.

@code{rnk} is the number of dimensions the new array should have.
The remaining arguments (each of type unsigned) indicates the size of
each consecutive dimension.  Note that the number of arguments following
@code{rnk} @emph{must} be the same as the value in @code{rnk}.

@code{ary} is the new array object created and will be initialized to
all @code{NULL}'s.
@example
@group
@exdent Example:

object  ary;

ary = vNew(PointerArray, 2, 5, 4);
/*  ary is a 5x4 matrix  */
@end group
@end example
@c @sp 1
@c See also:  @code{}
@end deffn



@subsection PointerArray Instance Methods
Most instance functionality is obtained and documented in the @code{Array}
class, however, functionality which is particular to this class is documented
in this section.





@deffn {PointerValue} PointerValue::PointerArray
@sp 2
@example
@group
v = vPointerValue(ary, ...);

object    ary;
unsigned  ...
void      *v;
@end group
@end example
This method is used to obtain the pointer value associated with a
particular element of an instance of the @code{PointerArray} class.

The arguments after the @code{ary} argument (each an @code{unsigned})
are used to specify the exact index into each consecutive dimension of
the array.  The number of arguments after the @code{ary} argument
@emph{must} be equal to the number of dimensions (or rank) of array
@code{ary}.  See @code{IndexOrigin::Array} for more information.
@example
@group
@exdent Example:

object  ary;
void    *v;

ary = vNew(PointerArray, 2, 5, 4);
v = vPointerValue(ary, 1, 2);
/*  v has ary[1][2]  */
@end group
@end example
@sp 1
See also:  @code{ChangeValue::PointerArray}
@end deffn








@deffn {ChangeValue} ChangeValue::PointerArray
@sp 2
@example
@group
ary = vChangeValue(ary, val, ...);

object    ary;
void     *val;
unsigned  ...
@end group
@end example
This method is used to change the value of one element of
@code{PointerArray} @code{ary}.

@code{val} is the value which the element of the array should be changed
to.

The arguments after the @code{val} argument (each an @code{unsigned})
are used to specify the exact index into each consecutive dimension of
the array.  The number of arguments after the @code{val} argument
@emph{must} be equal to the number of dimensions (or rank) of array
@code{ary}.  See @code{IndexOrigin::Array} for more information.

The value returned is the modified array passed.
@example
@group
@exdent Example:

object  ary;
int     v = 5;

ary = vNew(PointerArray, 2, 5, 4);
vChangeValue(ary, &v, 1, 2);
/*  ary[1][2] <- &v  */
@end group
@end example
@sp 1
See also:  @code{PointerValue::PointerArray}
@end deffn




